\input{/Users/daniel/github/config/preamble.sty}
\input{/Users/daniel/github/config/thms-eng.sty}

\usepackage[style=authortitle-terse,backend=bibtex]{biblatex}
\addbibresource{bibliography.bib}

\begin{document}

\begin{minipage}{\textwidth}
	\begin{minipage}{1\textwidth}
		K3 surfaces \hfill Daniel González Casanova Azuela
		
		{\small Prof. Misha Verbitsky\hfill\href{https://github.com/danimalabares/k3}{github.com/danimalabares/k3}}
	\end{minipage}
\end{minipage}\vspace{.2cm}\hrule

\vspace{10pt}
{\huge Home assignment 1: Riemann-Roch formula in dimension 1}

(Partial progress)

\begin{manualexercise}{1}
	Prove that the ring $\mathcal{O}_1$ of germs of holomorphic functions on $\mathbb{C}$is a principal ideal ring.
\end{manualexercise}

\begin{proof}[Solution]\leavevmode
	I assume that $\mathcal{O}_1$ is the ring of germs of holomorphic functions about $0\in\mathbb{C}$. I will use \cite{gri}, Chapter 0, section \textit{Weierstrass Theorems and Corollaries}. A \textit{\textbf{Weierstrass polynomial}} in $w$ is a function
	\[w^d+a_1w^{d-1}+\ldots+a_d(z),\qquad a_i(0)=0.\]

\begin{thm}[Weierstrass Division Theorem]\leavevmode
	Let $g(z,w)\in\mathcal{O}_{n-1}[w]$ be a Weierstrass polynomial of degree $k$ in $w$. Then for any $f\in\mathcal{O}_n$ we can write
	\[f=g\cdot h+r\]
	with $r(z,w)$ a polynomial of degree $<k$ in $w$.
\end{thm}
In words, we can express the germ of a holomorphic function around 0 as the the product of a Weierstrass polynomial of degree $k$ times some holomorphic function plus a polynomial of degree $<k$.

Then the proof is just mimicking the \href{https://proofwiki.org/wiki/Polynomial_Forms_over_Field_form_Principal_Ideal_Domain}{proof} that the usual polynomial ring is a principal ideal domian.

Let $J$ be an ideal of $\mathcal{O}_1$ and $g\in J$ be a Weierstrass polynomial of lowest degree. Then by Weierstrass Division Theorem we get $h,r\in\mathcal{O}_1$ such that $f=g\cdot h+r$ but by the choice of $g$ we see $r$ must be zero. This means $J=\left<g\right> $.

\begin{remark}[Based in Bruno's approach]
	The former proof using Weierstrass polynomials is, though correct, a bit of an overkill, since, as explained by Griffiths \& Harris, Weierstrass polynomials are used to generalize the 1-dimensional situation, where by power series expansion we know that a holomorphic function has a unique local representation {\color{8}(why?)}
	\[f(z)=(z-z_0)^nu(z),\qquad u(z_0)\neq 0\]
       So, any function $f\in\mathcal{O}_1$ can be expressed as
 \[f(z)=z^ng(z),\qquad g(0)\neq 0\]
Then we notice that if $f$ is in some ideal $J \subset \mathcal{O}_1$, then the ideal $(z^n)$ must be contaied in $J$ since any $\phi(z)z^n \in(z^n)$ must also belong to $J$ since
\[z^n\phi(z)=\frac{f(z)\phi(z)}{g(z)}\in J\]{\color{8}(why is it ok to divide by $g$? It is only non-zero at 0…)} Then, like in our proof, let $n_0$ the least order of vanishing of non-zero elements of $J$, which cannot be zero for in that case we get $J=\mathcal{O}_1$. Then we get that $J=(z^{n_0})$ since any $f\in J$ can be written as
\[f(z)=z^ng(z)=z^{n_0}z^{n-n_0}g(z).\]
\end{remark}
\end{proof}

\addcontentsline{toc}{section}{1.2 Invariant factors theorem}
\begin{manualexercise}{1.2}[Invariant factors theorem]
	Let $R$ be a principal ideal ring. Prove that any finitely-generated $R$-module is a direct sum of cyclic $R$-modules. Use this result to deduce the Jordan normal form theorem, and to classify the finitely-generated abelian groups.	
\end{manualexercise}

\begin{proof}[Solution]\leavevmode
	I follow \cite{dummit} to prove that
	 \begin{idea4}{Invariant Factors Theorem}\leavevmode
		Let $R$ be a principal ideal \textit{domain} and let $M$ be a finitely generated $R$-module. Then 
		\[M\cong  R^r\oplus R/(a_1)\oplus R/(a_2)\oplus \ldots\oplus R/(a_m)\]
		for some integer $r\geq 0$ and nonzero elements $a_1,a_2,\ldots,a_m$ of $R$ which are not units in $R$ and which satisfy the divisibility relation
		\[a_1|a_2|\ldots|a_m\]
		 \end{idea4}
	which strongly relies on
		\begin{idea3}{Theorem 4}\leavevmode
		Let $R$ be a PID, $M$ a a free $R$-module of finite rank $n$ and $N$ a submodule of $M$. Then $N$ is a free module of rank $m\leq n$ and there exists a basis $y_1,y_2,\ldots,y_n$ of $M$ such that $a_1y_1,a_2y_2,\ldots,a_my_m$ is a basis of $N$, where $a_1,a_2,\ldots,a_m$ are nonzero elements of $R$ with the divisibility relations
		\[a_1|a_2|\ldots|a_m.\]
	\end{idea3}
whose proof is rather involved.

\begin{proof}[Proof of Invariant factors theorem]
		Choose a basis $\{x_i\}_{i=1}^n$ of $M$. Consider the free module $R^n$ along with a basis $\{b_i\}_{i=1}^n$. The homomorphism $\pi:R^n\to M$, $b_i\mapsto x_i$ is surjective, so we get $R^n/\ker \pi\cong M$.

		Apply Theorem 4 for the module $R^n$ and its submodule $\ker \pi$. We get a basis $\{y_i\}_{i=1}^n$ of $R^n$ such that $\{a_iy_i\}_{i=1}^m$ is a basis of $\ker \pi$ for $a_i\in R$ for $i=1,\ldots,m$ such that $a_1|\ldots|a_m$. Then
\[M\cong R^n/\ker \pi=(Ry_1\oplus \ldots\oplus Ry_n\Big/(Ra_1y_1\oplus \ldots\oplus Ra_my_m).\]
It remains to show that the quotient on the right-hand side of last equation is in fact the desired decomposition of $M$.

Consider the map
\[Ry_1\oplus \ldots\oplus Ry_n \quad \longrightarrow \quad R/(a_1)\oplus \ldots\oplus R/(a_m)\oplus R^{n-m}\]
given by
\[(\alpha_1y_1,\ldots,\alpha_ny_n)\quad \longmapsto\quad (\alpha_1+(a_1),\ldots,\alpha_m+(a_m),\alpha_{m+1},\ldots,a_n)\]
The kernel of this map is the set of elements which $\alpha_i\in(a_i)$ for all $i=1,\ldots,m$. Thus we can write the kernel as 
\[Ra_1y_1\oplus \ldots\oplus Ra_my_m.\]
	\end{proof}

	\begin{remark}
		Looks like I didn't use the divisibility condition $a_1|\ldots|a_m$.
	\end{remark}

	Now let's deduce the Jordan normal form theorem using again  \cite{dummit}. For this we fix a vector space $V$ over a field $F$ and a linear transformation $T$. This  makes $V$ into an $F[x]$-module by substitution of the variable $x$ by $T$.

	The \textit{\textbf{invariant factors}} are the elements $a_i$ from the last theorem, which in our present case can be shown to be monic polynomials $a_i(x)$ satisfying $a_1(x)|\ldots|a_n(x)$. Further, these elements are associated to the so-called  \textit{\textbf{elementary divisors}}, which concern another similar formulation of the Invariant factors theorem. The elementary divisors are powers of the irreducible components of the $a_i(x)$.

	If we assume that the  $a_i(x)$ factor into linear polynomials, the elementary divisors can be written as $(x-\lambda)^k$, where $\lambda$ is one of the eigenvalues of $T$ (we consider only one of the eigenvalues at this point since we are constructing \textit{one} of Jordan blocks of $T$ ).

	Using the Invariant factors theorem for elementary divisors, we see that $V$ is the direct sum of finitely many cyclic modules of the form $F[x]/(x-\lambda)^k$. Such quotients have basis $\bar{x}^{k-1},\bar{x}^{k-2},\ldots,\bar{x},1$. Now we observe that the polynomials
	\[(\bar{x}-\lambda)^{k-1},(\bar{x}-\lambda)^{k-2},\ldots,\bar{x}-\lambda,1\]
are also a basis. This follows since expanding the latter in terms of the $\bar{x}_i$ gives a triangular matrix, which makes it invertible, so it's a valid change of coordinates.

Nex we observe that action of multiplication by $x$, which is the same as applying $T$, maps these basis elements in the quotient as follows:
\begin{align*}
	1&\mapsto 1\bar{x}=\lambda\cdot 1+\bar{x}-\lambda\\
	(\bar{x}-\lambda)&\mapsto \bar{x}^2-\lambda\bar{x}=x\lambda\cdot (\bar{x}-\lambda)+(\bar{x}-\lambda)^2\\
	\vdots & \\
	(\bar{x}-\lambda)^{k-2}&\mapsto \lambda\cdot (\bar{x}-\lambda)^{k-2}+(\bar{x}-\lambda)^{k-1}\\
	(\bar{x}-\lambda)^{k-1}&\mapsto \lambda\cdot (\bar{x}-\lambda)^{k-1}+(\bar{x}-\lambda)^k
\end{align*}
(where I'm not sure how to compute the last two). The last expression simplifies further since $(\bar{x}-\lambda)^k=0$ in the quotient, so that multiplication of $x$ by the basis element $( \bar{x}-\lambda)^{k-1}$ is barely multiplication by $\lambda$. In sum, we have constructed the  \textit{\textbf{Jordan block}}
\[J_{\lambda}=\begin{pmatrix} \lambda &1\\
&\lambda\\
&&\ddots&1\\
&&&\lambda\\
&&&&1\\
&&&&\lambda
\end{pmatrix} \]
applying this procedure to the rest of the cyclic components of $V$ we obtain the  matrix representation
 \[\begin{pmatrix} J_1\\
 &J_2\\
 &&\ddots\\
 &&&J_t\end{pmatrix} \]
of $T$.

Finally, the classification of finitely generated abelian groups follows by direct application of the Invariant factors theorem choosing $R=\mathbb{Z}$, since abelian groups are in correspondence with $\mathbb{Z}$-modules.
\end{proof}

\begin{idea1}{Definition 1.2}\leavevmode
	A \textit{\textbf{coherent sheaf}} on a complex manifold $M$ is a sheaf of modules over the sheaf $\mathcal{O}_M$ of holomorphic functions on $M$, which is locally finitely generated and locally finitely presented (that is, the sheaf of relations between its local generators is also locally finitely generated {\color{2}(I don't understand this)}).
\end{idea1}

\addcontentsline{toc}{subsection}{Exercise 1.3 (Coherent sheaf at a point on a curve)}
\begin{idea5}{Exercise 1.3}\leavevmode
	Let $C$ be a  complex curve, and $x\in C$ a smooth point. Prove that any coherent sheaf on $C$ supported in $x$ is isomorphic to $\bigoplus_{i=1}^k \mathcal{O}_C/\mathfrak{m}^{d_i} $.
\end{idea5}

\begin{proof}[Solution]\leavevmode
	First notice that the local ring of regular functions $\mathcal{O}_x$ at a smooth point $x\in C$ is a principal ideal domain. This follows from exercise 1, since any neighbourhood of $x$ may be identified with a neighbourhood of $0\in\mathbb{C}$.

	\begin{remark}
		In more general algebraic geometry the statement also holds, namely, the local ring of regular functions $\mathcal{O}_x$ at a smooth point x in a curve over a closed field $k$ is a principal ideal domain. This is because (see Hartshorne thm  I.5.1) since $x$ is smooth, $\mathcal{O}_x$ is regular, meaning $\mathfrak{m}/\mathfrak{m}^2$ has the same dimension as $\mathcal{O}_x$ as a vector space over the field $\mathcal{O}_x/\mathfrak{m}$. This means that any element in $\mathfrak{m}$ will be generated by any element in $\mathfrak{m}/\mathfrak{m}^2$. 
		%{\color{3}(Remember that the point of considering $ \mathfrak{m}/\mathfrak{m}^2$ is to mod out the singular functions at $x$, which are (exactly?) those in $\mathfrak{m}^2$---because those in  $\mathfrak{m}$ vanish at $x$ and then apply chain rule.)} (Also see \href{https://en.wikipedia.org/wiki/Discrete_valuation_ring}{Discrete Valuation Rings wiki}.)
	\end{remark}

	Now any coherent sheaf $\mathcal{F}$ supported in $x$ is, in particular, finitely generated over $\mathcal{O}_x$, so we may apply Invariant Factors Theorem at any neighbourhood $U  \ni x$ to obtain
	\[\mathcal{F}(U)=\mathcal{O}_x^r\oplus \mathcal{O}_x/(a_1)\oplus \ldots \mathcal{O}_X/(a_m)\]
	It only remains to show that the quotients $\mathcal{O}_x/(a_i)$ are actually $\mathcal{O}_x/\mathfrak{m}^{d_i}$ for some positive integers $d_{i}$. This ammounts to showing that every ideal in $\mathcal{O}_x$ is a power of $\mathfrak{m}$. A straighfoward proof of this fact (provided by Bruno) is based on exercise 1.1: 

	\begin{proof}[Proof of exercise 1.1]
	
The idea is that $\mathcal{O}_1$ is not only a principal ideal ring, but its ideals are in fact of the form $(z^{n_0+k})=(z^{n_0})^k$.

Recall that for any ideal  $J \subset \mathcal{O}_1$ and $f\in\mathcal{O}_1$ we can express $f$ locally as $f(z)=z^kg(z)$ with $g(0)\neq 0$. Let $k_0\neq 0$ be the smallest such number (if $k_0=0$, all $f\in J$ do not vanish at $0$ and we obtain that $J=\mathcal{O}_1$) and $f_0=z^{n_0}g_0$ the corresponding element in $J$. Then $J \subset (z^{k_0})$ since every $f_1(z)=z^{k_1}g_1(z)$ can be expressed as 
\[f_1(z)=z^{k_0}z^{k_1-k_0}(z)g_1(z)=z^{k_0}h(z)\in(z^{k_0})\]
This shows every ideal is of the form $(z^{k_0})$. Finally, every ideal must be contained in the maximal ideal $(z^{n_0})$ (this is a straighforward consequence of Zorn's lemma). Then every ideal is of the form $(z^{k_0})\subset(z^{n_0})$, so $z^{k_0}=z^{n_0+k}$ and we get the desired assertion that $(z^{k_0})=(z^{n_0+k})=(z^{n_0})^k$.
	\end{proof}

	\begin{remark}[by Altan]\leavevmode
	As in the previous remark, the fact that all ideals are powers of the maximal ideal in a local ring at a smooth point in an abstract algebraic variety also holds. This follows from Nakayama's lemma. (See also \href{https://math.stackexchange.com/questions/704613/local-ring-with-principal-maximal-ideal}{StackExchange}: all principal ideals are powers of the maximal ideal when  $ \bigcap_{n \in \mathbb{N}} \mathfrak{m}_x^n=\{0\}$.)
\end{remark}
\end{proof}

\begin{manualexercise}{1.4}[After date]
	Let $C$ be a complex curve, and $V$ an abelian group, freely generated by isomorphism classes of coherent sheaves on $C$. The \textit{\textbf{Gothendieck K-group}} $K_0(C)$ is the quotient of $V$ by its subgroup generated by relations $[F_1]+[F_3]=[F_2]$ for all exact sequences of coherent sheaves $\begin{tikzcd}
		0\arrow[r]&F_1\arrow[r]&F_2\arrow[r]&F_3\arrow[r]&0
	\end{tikzcd}$
	\begin{enumerate}[label=\alph*.]
		\item Let $L$ be a line bundle and $\begin{tikzcd}
			0\arrow[r]&\mathcal{O}_C\arrow[r]&L\arrow[r]&R\arrow[r]&0
		\end{tikzcd}$ be an exact sequence associated with a section $\ell \in H^{0}(C,L)$. Prove that $[L]-[\mathcal{O}_C]=\sum_{i}a_i[x_i]$, where $a_i\in\mathbb{Z}^{>0}$, $[x_i]$ are classes of skyscraper sheaves $\mathcal{O}_C/\mathfrak{m}_{x_i}$, and $\mathfrak{m}_{x_i}$ is the maximal ideal of a point $x_i$.
	
	\item Prove that $K_0(C)$ is generated by $\mathcal{O}_C$ and the classes of skyscraper sheaves $\mathcal{O}_C/\mathfrak{m}_{x}$.
	\end{enumerate}
\end{manualexercise}

\begin{proof}\leavevmode
	\begin{enumerate}[label=\alph*.]
		\item Here I suppose that the associated exact sequence to the section $\ell$ is given by
			\[\begin{tikzcd}[row sep=tiny]
				0\arrow[r]&\mathcal{O}_C\arrow[r,"m"]&L\arrow[r]&R=\operatorname{coker} m=L/\operatorname{img} m\arrow[r]&0\\
				& f\arrow[r,maps to]& f\cdot \ell\arrow[r,maps to]&f\cdot \ell+\operatorname{img} m
			\end{tikzcd}\]
			(See \href{https://stacks.math.columbia.edu/tag/01AL}{Stacks Project}.) Since  $L$ is a line bundle, it is a locally free sheaf. This makes $R$ a finitely generated module (see \href{https://en.wikipedia.org/wiki/Finitely_generated_module}{wiki}: module is finitely generated iff it is a quotient of a free module of finite rank). Apply Invariant factors theorem to obtain  $R=\sum_{i}\mathcal{O}_C/\mathfrak{m}^{d_i}$. In the Grothendieck group we get
			\[ [L]-[\mathcal{O}_C]=\left[\sum_{i}\mathcal{O}_C/\mathfrak{m}^{d_i}\right]= \sum [\mathcal{O}_X/\mathfrak{m}^{d_i}]\]
	\end{enumerate}
\end{proof}

\begin{manualexercise}{1.5}[After date]
	Let $C$ be a compact complex curve, and $F$ a coherent sheaf on  $C$. We define the \textit{\textbf{Euler characteristic}} of $F$ as
	\[\chi(F):=\dim H^{0}(C,F)-H^{1}(C,F)\]
	Prove that $\chi$ defines a group homomorphism $K_0(C)\to \mathbb{Z}$.
\end{manualexercise}

\begin{manualexercise}{1.7}
	Let $X$ be a complex manifold and $\operatorname{Pic}(X)$ the set of equivalence classes of vector bundles, equipped with multiplicative structure induced by the tensor product. The group $\operatorname{Pic}(X)$ is called the \textit{\textbf{Picard group}}of $X$.
	\begin{enumerate}[label=\alph*.]
		\item Prove that the cohomology group $H^{1}(X,\mathcal{O}^*_X)$ is naturally identified with $\operatorname{Pic}(X)$.
\iffalse		\item Consider the \textit{\textbf{exponential exact sequence}}
			\[\begin{tikzcd}
				0\arrow[r]&0\arrow[r]&\mathbb{Z}_X\arrow[r]&\mathcal{O}_X\arrow[r]&0
			\end{tikzcd}\]
			where $\mathbb{Z}_X$ denotes the constant sheaf. The corresponding long exact sequence
			\[\begin{tikzcd}[column sep=small]
				\cdots\arrow[r]&H^{1}(X,\mathcal{O}_X)\arrow[r]&H^{1}(X,\mathcal{O}^*_X)\arrow[r]&H^{2}(X,\mathbb{Z})\arrow[r]&\cdots
			\end{tikzcd}\]
			takes a line bundle $[L]\in\operatorname{Pic}(X)=H^{1}(X,\mathcal{O}^*_X)$ to an element $c_1(L)\in H^{2}(X,\mathbb{Z})$ called the \textit{\textbf{first Chern class of $L$}}. PRove that a non-trivial bundle $L$ with $c_1(L)=0$ on a compact complex curve has no holomorphic sections.

		\item (*) [content…]\fi
	\end{enumerate}
\end{manualexercise}

\begin{proof}[Solution]\leavevmode
	\begin{enumerate}[label=\alph*.]
	\item Recall that a line bundle $L$ may be reconstructed from the gluing functions $g_{\alpha\beta}:U_\alpha \cap U_\beta\to \mathbb{C}^*$, which can be thought as changes of coordinates on each fiber. These functions satisfy the consistency condition that  $g_{\gamma\beta}g_{\beta\gamma}=g_{\gamma\alpha}$ on $U_\alpha\cap U_\beta\cap U_{\gamma}$.

		On the other hand, a cochain in \v Cech cohomology is an assigment of an element $g_{\alpha\beta}\in \mathcal{O}^*_X$ for every intersection $U_\alpha\cap U_\beta$. Elements of $H^{1}(X,\mathcal{O}^*_X)$ are cocycles, meaning that coboundary operator vanishes. For 1-cocycles this is typically written as $(g_{\beta\gamma}-g_{\alpha\gamma}+\gamma_{\alpha\beta})|_{U_\alpha\cap U_\beta\cap U_\gamma}=0$. In multiplicative notation this just says that $g_{\beta\gamma}g^{-1}_{\alpha\gamma}\gamma_{\alpha\beta}=1$ which equivalent to the consistency condition for gluing functions.


		\item 

		\item Hopf surfaces.
			\[\begin{tikzcd}
			\mathbb{C}^2\setminus \mathbb{Z}=\left<\alpha\right> \arrow[d,"\mathbb{C}^* /\left<\alpha\right> "]\\
			\mathbb{C}P^{1}
			\end{tikzcd}\]
		now $H^{2}(\text{Hopf surface} )=0$. Topologically it is $S^1\times S^3$.

		Here's why it won't work on K3 surfaces:
		\begin{align*}
			a&\in H^{0}(L)\\
			c_1(L)&=[\mathcal{D}]\in H^{2}(M,\mathbb{Z})\\
			\mathcal{D}\text{ zero divisor of de ?}\\
			\int_{\mathcal{D}}\omega^{n-1}&=\left<[\mathcal{D}],\omega^{n-1}\right> =0
		\end{align*}
		So this statement tells you that Euler number is the same as its first Chern class.

		That
		\[e(L_1\otimes L_2)=e(L_1)+e(L_2)\]
		follows from taking zero sections on $L_2$, on $L_2$ and then tensor product goes to addition.
	\end{enumerate}
\end{proof}
\addcontentsline{toc}{section}{Exercise 1.8}
\begin{manualexercise}{1.8}
	Let $\mathbb{C}$ be a complex curve and $F$ a coherent sheaf on $\mathbb{C}$.
	\begin{enumerate}[label=\alph*.]
		\item Prove that the restriction of $F$ to a certain open set $U\subset C$ is isomorphic to a vector bundle. Prove that the rank of this vector bundle is independant on the choice of $U$ when $\mathbb{C}$ is irreducible. This number is called the \textit{\textbf{rank}} of $F$.
	\end{enumerate}	
\end{manualexercise}

\begin{proof}[Solution]\leavevmode
	\begin{enumerate}[label=\alph*.]
		\item (From \href{https://math.stackexchange.com/questions/2725664/vector-bundle-locally-free-sheaf}{StackExchange}) I show how to construct a vector bundle from a \textit{locally free} sheaf. Given an open cover $U_i$ such that $\mathcal{F}(U_i)$ is free for all $i$, we just need to find gluing functions $g_{ij}:U_i\cap U_j\to \operatorname{GL}(n,\mathbb{C}) $.

		From the definition of locally free, we have isomorphisms $f_i:\mathcal{F}(U_i)\to \mathcal{O}^n_{U_i}$. Restricting to the intersection $U_i\cap U_j$ we may define the functions
	\[f_{ij}=f_j|_{U_i\cap U_j}\circ f_i|_{U_i\cap U_j}^{-1} :\mathcal{O}^n_{U_i}|_{U_i\cap U_j}\to \mathcal{O}^n_{U_j}|_{U_i\cap U_j}.\]
	The claim in StackExchange is that every such map is induced by a gluing function, but I still cannot see why.

	The question indeed is to show why a coherent sheaf can be locally expressed as a locally free sheaf. We cover the manifold with sheafs that look like exercise 1.3. And that basically says that the sheaf is locally a direct sum of free and a finite number of torsion quotients $\mathcal{O}_X/\mathfrak{m}^{d_i}$.

	So
	\[\begin{tikzcd}
		0\arrow[r]&\text{Torsion} \arrow[r]&\mathcal{F}\arrow[r]&(\mathcal{F}^{**}\arrow[r]&0
	\end{tikzcd}\]
	where $\mathcal{F}^* =\mathcal{H}\operatorname{om}(\mathcal{F},0)$.

	\iffalse\begin{remark}
		(\href{https://math.stackexchange.com/questions/2597356/vector-bundle-associated-to-a-locally-free-sheaf}{StackExchange}) The equivalence of the categories of vector bundles and locally free sheaves is given by the functor
		\[\mathcal{E}\mapsto \operatorname{Spec}(\operatorname{Sym}(\mathcal{E})),\quad E\mapsto \Gamma^\vee_{E/X^\bullet}\]
	\end{remark}\fi
	\end{enumerate}
\end{proof}

\printbibliography

\end{document}
