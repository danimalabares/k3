\input{/Users/daniel/github/config/preamble.sty}%available at github.com/danimalabares/config
\input{/Users/daniel/github/config/thms-eng.sty}%available at github.com/danimalabares/config


\usepackage[style=authortitle-terse,backend=bibtex]{biblatex}
\addbibresource{bibliography.bib}

\setcounter{secnumdepth}{2}

\begin{document}

\begin{minipage}{\textwidth}
	\begin{minipage}{1\textwidth}
		K3 surfaces \hfill Daniel González Casanova Azuela
		
		{\small Prof. Misha Verbitsky\hfill\href{https://github.com/danimalabares/k3}{github.com/danimalabares/k3}}
	\end{minipage}
\end{minipage}\vspace{.2cm}\hrule

\vspace{10pt}
{\huge Home assignment 5: Positive forms and Riemann-Hodge pairing}

\begin{thing4}{Definition 5.1}\leavevmode
	Throughout this handout, $V=\mathbb{R}^{2n}$ is a real vector space, $I\in End(V)$ an operator which satisfies $I^2=-\operatorname{Id}$ ("the \textit{\textbf{complex structure operator}}"), and $\Lambda^{*}(V^*\otimes_\mathbb{R}\mathbb{C}) =\bigoplus \Lambda^{p,q}(V^*) $ the Hodge decomposition of its Grasman algebra. A (real) $(1,1)$-form $ \omega\in\Lambda^{1,1}(V^*)$ is \textit{\textbf{Hermitian}}, or \textit{\textbf{strictily positive}} if $\omega(x,Ix)>0$ for any non-zero  $x\in V$. It is called \textit{\textbf{semi-Hermitian}} or  \textit{\textbf{positive}} if  $\omega(x,Ix)\geq 0$ for any $x\in X$. A bivector $\eta\in\Lambda^{1,1}(V)$ is \textit{\textbf{positive}} if $\eta(v,Iv)\geq 0$ for any non-zero $v\in V^*$.
\end{thing4}

\setcounter{section}{4}
\subsection{Positive $(p,p)$-forms}

\begin{thing13}{Exercise 5.1}\leavevmode
	Let $\operatorname{Pos}\subset \Lambda^{1,1}(V^*)$ be the set of all positive $(1,1)$-forms, and $\operatorname{Pos}^n$ be the set all non-zero volume forms obtained as $n$-th power of elements of $\operatorname{Pos}$. Prove that $\operatorname{Pos}^n$ is connected.
\end{thing13}

\begin{proof}[Solution]\leavevmode

	$\operatorname{Pos}^n\subset\Lambda^{2n}(V^*)\cong \mathbb{R}$. Now volume forms in $\Lambda^{2n}(V^*)$ can be separated into two connected components according to an orientation as follows. Choose any of the two equivalence classes of bases according to positive or negative determinant of change of coordinates; define a form in $\Lambda^{2n}(V^*)$ to be positively oriented if it is positive in any of the bases of the chosen orientation. Since volume forms are nowhere-vanishing, any volume form is either positive or negative.

	\begin{claim}\leavevmode
	$\operatorname{Pos}^n$ must be contained in either of the connected components of the set of volume forms in $\Lambda^{2n}(V^*)$.		
	\end{claim}

	\begin{proof}[Proof of claim]\leavevmode
		It suffices to show that any two volume forms $\omega^n,\eta^n \in\operatorname{Pos}^n$ have the same sign on some basis of $V$.

		Suppose $e_1,\ldots,e_n$ is a basis of the complex vector space induced by $I$, and let $\varepsilon_i(e_j)=\delta_{ij}$ be its dual basis. Then a basis for $V$ is $e_1,Ie_1,\ldots,e_n,Ie_n$, and its dual basis is $\alpha_1:=\varepsilon_1,\alpha_2:=-I\varepsilon_1,\ldots,\alpha_{2n-1}:=\varepsilon_n,\alpha_{2n}:=-I \varepsilon_n$.

		Thus a basis for 2-forms is $\alpha_{ij}=\alpha_i\wedge \alpha_j$ for $i<j$. Then $\omega$ and $\eta$ must be of the form
	\[\omega=\sum a_{ij}\alpha_{ij},\qquad \eta=\sum b_{ij}\alpha_{ij},\qquad a_{ij},b_{ij} \in\mathbb{R}.\]
For the (perhaps trivial) case $n=1$ we conclude since both  $\omega$ and $\eta$ are multiples of $\varepsilon_1\wedge -I\varepsilon_1$, namely 
\[\omega=a_{12}\varepsilon_1\wedge (-I\varepsilon_1),\qquad \eta=b_{12}\varepsilon_1\wedge (-I\varepsilon_1)\]
and notice that $\varepsilon_1\wedge -I\varepsilon_1$ gives 1 when evaluated in the pair  $(e_1,Ie_1)$. Since both $\omega$ and $\eta$ are positive, their constants must be positive and thus they are in the same connected component of $\operatorname{Pos}^n$.

In the case $n=2$ the expression
	\begin{align*}\omega^2&=\left( \sum a_{ij}\alpha_{ij} \right)\wedge \left( \sum a_{k\ell}\alpha_{k\ell} \right)
	\end{align*}
	has a more complicated expansion. My guess is that when we evaluate on the basis $(e_1,Ie_1,e_2,Ie_2)$ the only term that can be non-zero corresponds to the factor
	\[\Big(\varepsilon_1\wedge I\varepsilon_1\wedge \varepsilon_2\wedge I\varepsilon_2\Big)(e_1,Ie_1,e_2,Ie_2)=1\]
The corresponding coefficient is then determined by the values $a_{12}$ and $a_{34}$, which must be positive since $\omega\in\operatorname{Pos}$. Indeed,
\[\omega(e_1,Ie_1)=a_{12},\qquad \omega(e_2,I_2)=a_{34}.\]
The general case is analogous using the equation
	\[\Big(\varepsilon_1\wedge (-I\varepsilon_1)\wedge \ldots \wedge \varepsilon_n\wedge (-I\varepsilon_n)\Big)(e_1,Ie_1,\ldots,e_n,Ie_n)=1\]

\iffalse
	\begin{align*}\omega^2&=\left(a_{12}\alpha_{12} \right)\wedge \left( a_{12}\alpha_{12} \right) \\
		&=\big(a_{12}(\varepsilon_1\wedge -I\varepsilon_1)\big)\wedge \big(a_{12}(\varepsilon_1\wedge -I\varepsilon_1)\big)\\
		&=a_{12}^2\big(\varepsilon_1\wedge (-I\varepsilon_1)\wedge \varepsilon_1\wedge (-I\varepsilon_1)\big)
	\end{align*}
In general,
	\begin{align*}\omega^2&=\left( \sum a_{ij}\alpha_{ij} \right)\wedge \left( \sum a_{k\ell}\alpha_{k\ell} \right) \\
		&=\sum a_{ij}a_{k\ell}\alpha_{ij}\wedge \alpha_{k\ell}
	\end{align*}\


	\end{proof}

	To see this suppose $e_1,\ldots,e_n$ are a basis of the complex vector space induced by $I$, and let $\varepsilon_i(e_j)=\delta_{ij}$ be its dual basis. Then a basis for $V$ is $e_1,Ie_1,\ldots,e_n,Ie_n$.

	just notice that any two volume forms $\omega^n,\eta^n \in\operatorname{Pos}^n$ {\color{6}have the same sign on the basis $e_1,Ie_1,\ldots,e_n,Ie_n$} (here $e_1,\ldots,e_n$ are a basis of the complex vector space induced by $I$). 


	{\color{6}But how to show that? The expression for $\omega^n$ is not  obvious and it contains permutations of the vectors; not only things of the form $\omega(e_i,Ie_i)$…}

$\mathsf{OK}$ so what is an $n$-th form? A linear

\iffalse Indeed,
	\begin{align*}
		\omega^n(e_1,Ie_1,\ldots,e_n,Ie_n)&=\omega(e_1,Ie_1)\wedge \ldots \wedge \omega(e_n,Ie_n)\\
		&=
	\end{align*}\fi\fi
\end{proof}
\end{proof}

\begin{thing8}{Remark 5.1}\leavevmode
	The corresponding orientation on $V$ is called the \textit{\textbf{orientation compatible with the complex structure operator}}.
\end{thing8}

\begin{thing4}{Exercise 5.2}\leavevmode
	Let $\omega\in\Lambda^{1,1}(V^*)$ be a 2-form on $V$, satisfying $\omega(x,Ix)\geq 0$, and $W\subset V$ the set of all vectors $v\in V$ such tat $\omega(v,Iv)=0$ 
	\begin{enumerate}[label=\alph*.]
		\item Prove that $W\subset V$ is $I$-invariant.
		\item Prove that there exists a projection $\Pi:(V,I)\to (V_1,I_1)$ commuting with the complex structure operator, and Hermitian form $\omega_1$ on $V_1$ such tat $\omega(x,y)=\omega_1(\Pi(x),\Pi(y))$
	
\end{enumerate}
\end{thing4}

\begin{proof}[Solution]\leavevmode
	\begin{enumerate}[label=\alph*.]
		\item It is immediate: for $ x\in W$, $Ix$ is in $W$ since  $\omega(Ix,I(Ix))=\omega(Ix,-x)=\omega(x,Ix)=0$.
		\item I think the subspace should be
			\[V_1:=\{v\in V\setminus\{0\}:\omega(v,Iv)>0\}\cup \{0\},\]
			so that the restriction $\omega|_{V_1}$ is immediately Hermitian. {\color{3}But I got stuck in constructing the projection map.}

		Upon consulting \href{https://mathoverflow.net/questions/7709/splitting-a-space-into-positive-and-negative-parts}{StackExchange} I propose the following argument. A substpace is called \textit{\textbf{maximally positive}} it it is positive and is not properly contained in any other positive  subspace.

		The partial order of positive subspaces with inclusion satisfies Zorn lemma condition that any total order has an upper bound (the whole space  $V$). Then there exist maximal positive subspace.

		Any maximal positive subspace $V_+$ and $W$ yield a direct sum decomposition  $V=V_+\oplus W$. That their intersection is  $\{0\}$ is obvious. Also $V=V_+ +W$ since any element must have either positive or zero norm because $\omega$ is positive.

\iffalse
		Consider the map
			\begin{align*}
				\omega^\flat: V &\longrightarrow V^* \\
				v &\longmapsto \omega(v,I\cdot)
			\end{align*}
			Then $\ker \omega^\flat=\{w\in V:\omega(v,Iw)=0\forall v\in V_1\}$.\fi
		\end{enumerate}
\end{proof}

\begin{thing9}{Exercise 5.3}\leavevmode
	Let $g\in\operatorname{Sym}^2V^*$ be an $I$-invariant, non-degenerate, symmetric 2-form on $V$. Such $g$ is called a \textit{\textbf{pseudo-Hermitian metric}}.
	\begin{enumerate}[label=\alph*.]
		\item Prove that the form $\omega(x,y):=g(Ix,y)$ belongs to $\Lambda^{1,1}(V^*)$. This form is called a \textit{\textbf{pseudo-Hermitian $(1,1)$-form}}.

		\item Prove that the signature of $g$ is $(2p,2q)$, where $p+q=n$. In this case we say that the \textit{\textbf{signature}} of the pseudo-Hermitian form $\omega$ is $(p,q)$.
	\end{enumerate}
\end{thing9}

\begin{proof}\leavevmode
	\begin{enumerate}[label=\alph*.]
		\item First let's see that indeed $\omega$ is an anti-symmetric form:
			 \[\omega(x,y)=g(Ix,y)=g(I^2x,Iy)=-g(x,Iy)=g(Iy,x)=\omega(y,x)\]Now let's do a quick review on what it means to be a $(1,1)$-form. I will show that $(1,1)$-forms are exactly those which preserve the complex structure, which finishes the proof.
	\begin{thing3}{Reminder}[Taken from VHS 2024 course slides]\leavevmode
	The \textit{\textbf{Hodge decomposition}} of the Grassman algebra $\Lambda^{\bullet}(V_\mathbb{C})$ of $V_{\mathbb{C}}$ is given in each degree by
	 \[\Lambda^{k}(V)=\bigoplus_{p+q=k} \Lambda^{p,q}(V_\mathbb{C}) \]
	 where
	 \[\Lambda^{p,q}(V_\mathbb{C})=\Lambda^{p}V^{1,0}\otimes \Lambda^{q}V^{0,1}\]
	 where
	 \[V_\mathbb{C}=V\otimes_\mathbb{R}\mathbb{C}=V^{1,0} \oplus V^{0,1}\]
	 is the decomposition of $V_\mathbb{C}$ given by the eigenspaces of the complex structure $I$.
\end{thing3}
According to the reminder above, a $(1,1)$-form is just an element of $\Lambda^{1,1}(V^*)=(V^* )^{1,0}\otimes(V^* )^{0,1}$. Here $(V^*)^{1,0}$ and $(V^* )^{0,1}$ are the $\sqrt{-1} $ and $-\sqrt{-1}$-eigenspaces of the induced complex structure on $V^*$. (Indeed: any eigenvalue $\lambda$ is such that $I\omega=\lambda\omega\implies -\omega=I^2\omega=\lambda I\omega=\lambda^2\omega$.) This means that an element $\omega\in\Lambda^{1,1}(V^*)$ is of the form $\omega=\omega' \otimes \omega'$ for $\omega'\in\Lambda^{1,0}(V^*)$ and $\omega''\in\Lambda^{0,1}(V^*)$. By definition of tensor product (\cite{tu-smooth} chpt 1., sec. 3.6), this means that for $v,w \in V$, $\omega(v,w)=\omega^{1,0}(v) \cdot \omega^{0,1}(w)$. And that implies that a $(1,1)$-form preserves the complex structure since $\omega(Iv,Iw)=\omega'(Iv)\cdot \omega''(Iw)=\sqrt{-1}\omega'(v)\cdot\big(-\sqrt{-1}\omega''(w)\big)=\omega'(v)\cdot \omega''(w)=\omega(v,w)$.

Conversely, any form $\eta$ preserving the complex structure, i.e., $\eta(Ix,Iy)=\eta(x,y)$, must be a $(1,1)$-form. Indeed, if  $\eta \not\in\Lambda^{1,1}(V^*)$ then $\eta\in\Lambda^{1,0}(V^*)$ or $\eta\in\Lambda^{0,1}(V^*)$. But in either case $\eta$ couldn't preserve the complex structure since, e.g. in the first case, $\Lambda^{1,0}(V^*)=\Lambda^{1}V^{1,0} \otimes \Lambda^{0}V^{0,1}= \Lambda^{1}V^{1,0} \otimes \{\text{constants} \}$ so $\eta(Ix,Iy)$ is independant of the second coordinate.

\item Recall that $V=V^{1,0} \oplus V^{0,1}$. The restriction of $g$ to any of the summands is a non-degenerate, symmetric 2-form and as such it must have a signature of, say, $(p,q)$. Now let's see that the form restricted to either of the spaces has the same signature. The restriction of $g$ to each of the subspace has a matrix of the form $B^{\mathbf{T}}GB$ where $G$ is the matrix of $g$ in $V$ and $B$ is a base of either of  $V^{1,0}$ and $V^{0,1}$. Indeed, the matrix $B$ given by a basis of, say, $V^{1,0}$ may be regarded as an embedding of $V^{1,0}$ into $V$, so that for $u,v\in V^{1,0}$ we have
	\[g(Bu,Bv)=(Bu)^{\mathbf{T}}G(Bv)=\tilde{u}^{\mathbf{T}}(B^{\mathbf{T}}GB)\tilde{v}\]
	(This is a general fact, the restriction of a linear transformation to a subspace corresponds to a matrix of this form. See \href{https://math.stackexchange.com/questions/4386914/given-the-matrix-of-an-r4-scalar-product-b-in-its-standard-basis-and-a-subspa/4387198#4387198}{StackExchange}.)

	But mapping a base of $V^{1,0}$ to a base of $V^{0,1}$ is given by complex conjugation, which is a linear isomorphism and thus preserves the signature of the restriced forms.


	
\end{enumerate}
\end{proof}
\iffalse

\section{Riemann-Hodge pairing}

\begin{thing7}{Definition 5.3}\leavevmode
	For the duration of this subsection, fix a Hermitian form $\omega$ on $(V,I)$, and let  $\operatorname{Vol}:=\omega^n\in\Lambda^{n,n}(V^*)$. the \textit{\textbf{Riemann-Hodge pairing}}  on $\Lambda^{k}(V^*)$, $k\leq n$ is the pairing
	\[q(\eta,\eta')=\dfrac{\eta\wedge \eta\wedge \omega^{n-k}}{\operatorname{Vol}}.\]
\end{thing7}

\begin{thing1}{Exercise 5.9}\leavevmode
	Prove that the Riemann-Hodge pairing is non-degenerate.
\end{thing1}

\begin{thing7}{Exercise 5.10}\leavevmode
	Let $x\in V^*$ be a non-zero vector. Prove that $q(x,Ix)>0$.
\end{thing7}

\begin{thing1}{Exercise 5.11}\leavevmode
	Let $V$ be an irreducible real representation of a compact Lie group, and $g$ a non-degenerate bilinear symmetric form on $V$. Prove that $g$ is positive definite or negative definite.
\end{thing1}

\begin{thing4}{Exercise 5.12}\leavevmode
	Let $\mathsf{U}(n) \subset \mathsf{GL}(V)$ denote the group of matrices preserving  $I$ and $\omega$, and $\mathfrak{u}(V) \subset \operatorname{End}(V)$ its Lie algebra.
	\begin{enumerate}[label=\alph*.]
		\item Consider the map $\Lambda^{1,1}(V) \to  \operatorname{End}(V^*)$ taking $\eta\in\Lambda^{1,1}(V)$ to the map $x\mapsto \omega(i_x\eta,\cdot )$, where $i_x\eta\in V$ is the contraction of $\eta$ with $x$. Prove that this map identifies $\Lambda^{1,1}(V)$ and $\mathfrak{u}(V^*)$.
	\end{enumerate}
\end{thing4}

\begin{proof}[Solution]\leavevmode
	\begin{enumerate}[label=\alph*.]
		\item We want to see that the image of the map $\text{{\color{2}$\eta$}} \mapsto \omega(i_x\eta,\cdot )$ is isomorphic to $\mathfrak{u}(V^*)$. Recall fro Riemann Surface course that the Lie algebra of a group if it is obtained from a Lie group $G\subset \mathsf{GL}(V)$ by taking logarithms of a neighbourhood of identity; and conversely a Lie group is obtained from a subspace $W\subset \operatorname{End}(V)$ by taking exponents.
	\end{enumerate}
\end{proof}

\fi
\printbibliography


\end{document}
