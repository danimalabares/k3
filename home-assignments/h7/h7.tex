\input{/Users/daniel/github/config/preamble.sty}%available at github.com/danimalabares/config
\input{/Users/daniel/github/config/thms-eng.sty}%available at github.com/danimalabares/config


\usepackage[style=authortitle-terse,backend=bibtex]{biblatex}
\addbibresource{~/github/config/bibliography.bib}

\setcounter{secnumdepth}{2}

\begin{document}

\begin{minipage}{\textwidth}
	\begin{minipage}{1\textwidth}
		K3 surfaces \hfill Daniel González Casanova Azuela
		
		{\small Prof. Misha Verbitsky\hfill\href{https://github.com/danimalabares/k3}{github.com/danimalabares/k3}}
	\end{minipage}
\end{minipage}\vspace{.2cm}\hrule

\vspace{10pt}
{\huge Home assignment 7: Moser isotopy lemma}
\iffalse
\begin{thing1}{Exercise 7.1}\leavevmode
Let $M$ be a compact manifold, and $V_0,V_1$ two smooth volume forms which satifsy  $ \int_{M}V_0=\int_{M}V_1$. Prove that there exists a diffeomorphism which satisfies $\Phi^*V_0=V_1$.
\end{thing1}

\begin{proof}[Solution]\leavevmode
We must show there is a family $V_t$ of cohomologous volume forms joining $V_0$ and $V_1$. Define
\[V_t=tV_0+(1-t)V_1.\]
Now since $H^n(M)$ is a real one-dimensional vector space, $[V_t]=\alpha[V_0]$ for some real number  $\alpha$.

To show that $\alpha$ must be 1 we need to integrate $[V_t]$, which we may do since $V_t$ is a closed nowhere-vanishing top-form. Closedness is immediate. To see it is nowhere-vanishing just notice that both $ V_0$ and $V_1$ are, and since they integrate to the same number they are both always positive or always negative.

Then
\begin{align*}
	\int_{M}[V_t]&=t\int_{M}[V_0]+(1-t)\int_{M}[V_1]=\int_{M}[V_0]=\int_{M}[V_1]
\end{align*}
which means that $\alpha=1$. Then by Moser's lemma there exists an isotopy $\varphi_t$ with $\varphi_t^*V_t=V_0$. Taking $t=1$ we obtain the result.
\end{proof}

\begin{thing5}{Problem 7.2}\leavevmode
	Let $(M,I)$ be an almost complex manifold, and $\omega_0,\omega_1$ co-homologous symplectic forms which satisfy $\omega_i(x,Ix)>0$ for any non-zero tangent vetor $x$ (such forms are called \textit{\textbf{taming}}).
	\begin{enumerate}[label=\alph*.]
	\item Prove that there exists a diffeomorphism $\Phi$ which satisfies $\Phi^*\omega_0=\omega_1$.
	\item Prove that $|x|^2_{i}:=\omega_i(X,Ix)$ is a Hermitian metric on $M$. Prove that a diffeomorphism that satisfies $\Phi^*\omega_0=\omega_1$ defines an isometry
		\[(M,|x|^2_1)\longrightarrow(M,|x|^2_0)\]
		is $\Phi$ is compatible with $I$, that is, satisfies $d\Phi(Ix)=I(d\Phi(x))$ \item Find an example of $\omega_0,\omega_1\in\Lambda^2(M)$ such that a diffeomorphism compatible with $I$ and satisfying $\Phi^*\omega_0=\omega_1$ does not exist.
	\end{enumerate}
\end{thing5}

\begin{proof}[Solution]\leavevmode
\begin{enumerate}[label=\alph*.]
\item This is immediate from Moser's lemma defining $\omega_t:=t\omega_0+(1+t)\omega_1$.
\item According to Riemann Surfaces course, a hermitian form may be understood as a Riemannian metric $h$ such that $h(\cdot,\cdot)=h(I\cdot,I\cdot)$. This happens for $h(\cdot,\cdot)=\omega(\cdot,I\cdot)$ if we require $\omega(\cdot,\cdot)=\omega(I\cdot,I\cdot)$. Indeed, symmetry holds since
	\[\omega(x,Iy)=\omega(Ix,-y)=\omega(y,Ix)\]
and positive-definiteness is given as a hypothesis.
	

	The condition that $\omega(\cdot,I\cdot)$ is a riemannian metric is called \textit{\textbf{compatibility}} of  $\omega$ and $I$ in  \cite{das}. In this case we can also produce a hermitian metric in the sense of  \href{https://mathworld.wolfram.com/HermitianInnerProduct.html}{WolframMathWorld}, namely, a positive-definite symmetric sesquilinear form. Indeed, define $g(\cdot,\cdot):=\omega(\cdot,I\cdot)$, then the form
	\[h:=g+\sqrt{-1}\omega\]
satisfies the required properties as follows.
	\begin{enumerate}
	\item Additivity, $h(u_1+u_2,v)=h(u_1,v)+h(u_1,v)$, is immediate.
	\item Homegenity on the first argument, $h(\lambda u,v)=\lambda h(u,v)$, is also immediate.

			\item $h(u,v)=\overline{h(v,u)}$ is clear by anti-symmetry of $\Omega$:
			\begin{align*}
				h(u,v)& =g(u,v)+i\omega(u,v) =g(v,u)-i\omega(v,u) =\overline{h(v,u)}
			\end{align*}
	\item The property $h(u,\lambda v)=\bar{\lambda} h(u,v)$ follows easily from (b) and (c) since 
			\begin{align*}
				h(u,\lambda v)& =\overline{h(\lambda v, u)}=\bar{\lambda} \overline{h(v,u)} =\bar{\lambda} h(u,v)
			\end{align*}
	\item Positive-definiteness follows from positive-definiteness of $g$ and antisymmetry of $\omega$.
	\end{enumerate}
	\item The conditions $\Phi^*\omega_0=\omega_1$ and $\Phi_*(I\cdot)=I\Phi_*(\cdot)$ imply that
		\begin{equation}\label{eq:1}(\Phi^*\omega_0)(\cdot,I\cdot)=\omega_0(\Phi_*\cdot,\Phi_*I\cdot)=\omega_0\Phi_*\cdot,I\Phi_*\cdot)=\omega_1(\cdot,I\cdot)\end{equation}
	Whether the metric is taken to be $\omega_i(\cdot,I\cdot)$ or $h_i=g_i+\sqrt{-1}\omega_i$, the result follows. In the latter case only note that
	\begin{align*}\Phi^*h_0&=\Phi^*(g_0+\sqrt{-1}\omega_0)=\Phi^*g_0+\sqrt{-1}\Phi^*\omega_0\\&=\Phi^*\Big(\omega(\cdot,I\cdot)\Big)=g_1+\sqrt{-1}\omega_1=h_1\end{align*}
	by \cref{eq:1}.
	
\item I am intrigued to know the answer here.	
\end{enumerate}
\end{proof}
\fi
\begin{thing8}{Exercise 7.4}
Let $(M,V)$ be a connected manifold equipped with a volume form. Prove that the group of volume-preserving diffeomorphisms acts on $M$ transitively.
\end{thing8}

\begin{proof}[Solution]\leavevmode
I will sketch the proof of  \cite{boothby}, found in \href{https://mathoverflow.net/questions/61994/how-transitive-are-the-actions-of-symplectomorphism-groups}{MathOverflow}.

\iffalse The idea is to split a path joining any two points into small pieces, and find the volume-preserving diffeomorphisms locally.

\begin{enumerate}
\item \textit{\textbf{Strong local transitivity}} on $M$ means that for every point $p\in M$ and neighbourhood $U$ of $p$ there are neighbourhoods $V$ and $W$ of $p$ with $\overline{V}\subset W$ and $\overline{W} \subset U$, $\overline{W}$ compact, and for any $q \in V$ there is a volume-preserving isotopy $\Phi_t$ of the identity map $\Phi_0$ to $\Phi_1$ such that $\Phi_1(p)=q$ and for all  $t$, $\Phi_t$ leaves fixed every point outside $\overline{W}$.
\item If $M$ is strongly locally transitive then it is transitive by volume-preserving diffeomorphisms. \textit{Sketch of proof:} let $\gamma$ be a path from $p$ to $q$. For every $x \in \gamma$ choose $0<\delta'_x<\delta''_x<\varepsilon$  such that the $\delta'$ and $\delta''$ neighbourhoods of $x$ satisfy the conditions of strong local transitivity. Since $C_1$ is compact, there is a finite collection of points $p_1=a_1,a_2,\ldots,a_r=q_1$ with volume-preserving diffeomorphisms mapping each point to the next. The composition of all such diffeomorphisms yields the desired volume-preserving map.
\item It remains to show that any connected manifold with a volume form is strongly locally transitive. Consider a path $\gamma$ from $p$ to $q$.
\fi
	
	%(such a neighbourhood must be chosen correctly to satisfy the conditions of strong locally transivitity, namely there must be two subneighbourhoods $V$ and  $W$ as above).
We wish to define a vector field along $\gamma$ whose flow $\Phi_t$ gives the desired volume preserving diffeomorphism at $t=1$. Consider a diffeomorphism from $I\times B^{n-1}$ to a neighbourhood $U$ of $\gamma$ (this diffeomorphism is constructed from geodesic segments introducing a metric on $M$). The volume form on $U$ may be expressed as \[V=fdt \wedge dx_1 \wedge\ldots\wedge dx_{n-1}\] for a positive function $f$. Now consider the vector field
\[X'=\frac{1}{f}\frac{\partial }{\partial t}.\]
$X'$ may be extended to a vector field $X$ in all of $M$ using a partition of unity. It is immediate that the flow of this vector field takes $p$ to $q$.

To show this flow is volume-preserving recall Cartan's formula that $\mathcal{L}_XV=di_{X}V+\cancelto{0}{i_XdV}=di_XV$ since $V$ is a top-form. Thus, showing volume invariance ammounts to showing $d_XV=0$, which holds since
\[i_{X'}fdt\wedge dx_1 \wedge \ldots\wedge dx_{n-1}=dx_1\wedge \ldots \wedge dx_{n-1}\]
is closed.
%\end{enumerate}
\end{proof}

\begin{thing5}{Exercise 7.5}\leavevmode
	Prove that the group of symplectomorphisms of $(M,\omega)$ acts transitively on $M$, for any connected symplectic manifold $(M,\omega)$.
\end{thing5}

\begin{proof}[Solution]\leavevmode
This case is similar but requires a little more work: we intend to use Darboux coordinate charts, which can only be defined locally. We will split a path joining any two points into small pieces, find the symplectomorphisms locally and the compose them.
\begin{enumerate}
\item \textit{\textbf{Strong local transitivity}} on $M$ means that for every point $p\in M$ and neighbourhood $U$ of $p$ there are neighbourhoods $V$ and $W$ of $p$ with $\overline{V}\subset W$ and $\overline{W} \subset U$, $\overline{W}$ compact, and for any $q \in V$ there is a symplectomorphism isotopy $\Phi_t$ of the identity map $\Phi_0$ to $\Phi_1$ such that $\Phi_1(p)=q$ and for all  $t$, $\Phi_t$ leaves fixed every point outside $\overline{W}$.
\item If $M$ is strongly locally transitive then it is transitive by symplectomorphisms. \textit{Sketch of proof:} let $\gamma$ be a path from $p$ to $q$. For every $x \in \gamma$ choose $0<\delta'_x<\delta''_x<\varepsilon$  such that the $\delta'$ and $\delta''$ neighbourhoods of $x$ satisfy the conditions of strong local transitivity. Since $\gamma_1$ is compact, there is a finite collection of points $p_1=a_1,a_2,\ldots,a_r=q_1$ with volume-preserving diffeomorphisms mapping each point to the next. The composition of all such diffeomorphisms yields the desired volume-preserving map.
\item It remains to show that any connected symplectic manifold is strongly locally transitive. Let $U$ be a Darboux chart of $p$ and take two neighbourhoods $V'$ and $W'$ as in the definition of strong local transitivity. For any  $y \in V'$ we can define the vector field $X$ in $\mathbb{R}^{2n}$ of vectors parallel to $yx_0$, where $x_0$ are the coordinates of $p$. The flow of this vector field takes takes $x_0$ to $y$ at $t=1$.

To show that this vector field is $\omega$-invariant we use again Cartan's formula to see it's enough to show that $di_X\omega=0$. Now if $\dim M=2m$ we have
\[i_X\omega=\sum_{i=1}^m dx^i\wedge dx^{i+m}\left(X,\cdot \right){\color{8}=0}\]
({\color{8}I'm not sure why does it vanish…})
\end{enumerate}


\end{proof}

\vspace{1em}

{\Large Extra exercise: infinite product}\vspace{1em}

\begin{thing4}{Exercise 8}[\cite{zo1}, 3.2.5, p. 148]\leavevmode
An inifinite product $\prod_{k=1}^\infty e_k$ is said to converge if the sequence $\Pi_n=\prod_{k=1}^n e_k$ has a finite \textbf{nonzero} limit $\Pi$. We then set $\Pi=\prod_{k=1}^\infty e_k$. Show that
\begin{enumerate}[label=(\alph*)]
	\item[(b)]  if $\forall n \in N (e_n>0)$, then the infinite product converges $\prod_{n=1}^\infty$ converges if and only if the series  $\sum_{n=1}^\infty \operatorname{log}e_n$ converges.
\end{enumerate}
\end{thing4}

\begin{remark}\leavevmode
The proof is straightforward using elementary properties of the logarithm ($\operatorname{log}(ab)=\operatorname{log}a+\operatorname{log}b$) and the exponential ($\operatorname{exp}(a+b)=\operatorname{exp}a+\operatorname{exp}b$), as well as their continuity. The caveat is that if the product converges to zero, taking logarithms of the partial products gives a series diverging to $-\infty$ (we say that the product \textit{\textbf{diverges to zero}}). The following exercise is an example of this.
\end{remark}

\begin{claim}\leavevmode
\begin{align*}
			\prod_{n\geq 0}\left|  -1+\frac{1}{n+\frac{1}{2}}\right|=0.
				\end{align*}
\end{claim}			
\begin{proof}\leavevmode
By taking exponent of the partial sums of the following series, we see it's enough to show that
\begin{equation}\label{eq:log}\sum_{n \geq 0}\operatorname{log}\left|  -1+\frac{1}{n+\frac{1}{2}}\right|=-\infty.
\end{equation}
To show this first notice that $\left|  -1+\frac{1}{n+1/2}\right|$ is just  $1-\frac{1}{n+1/2}$. We can also quickly notice that $\operatorname{log}\left(1-\frac{1}{n+1/2}\right)$ is a sequence of negative numbers convering to 0.

(It was not immediate to me why the this series should diverge or converge. The following argument was provided by ChatGPT.)

We can prove the series diverges if we find a divergent series that bounds it from above. Such series is $\sum\frac{-1}{n+1/2}$. Indeed, it turns out that for every number $0\leq x<1$,
\[\operatorname{log}(1-x) \leq -x.\]
To see why, define the function $f(x)=\operatorname{log}(1-x)+x$ and differentiate to find
\[\frac{d}{dx}\operatorname{log}(1-x)+x=-\frac{1}{1-x}+1=\frac{1}{1-x}+\frac{1-x}{1-x}=\frac{-x}{1-x}\leq 0,\]
and since also $f(0)=0$, we have that
\[\operatorname{log}(1-x)+x\leq 0,\qquad 0\leq x<1.\]
Finally just recall that $\sum_{n\geq 0}\frac{1}{n}$ diverges (and behaves like $\sum \frac{1}{n+1/2}$ for lage $n$). This confirms \cref{eq:log}.




\end{proof}

\printbibliography




\end{document}
