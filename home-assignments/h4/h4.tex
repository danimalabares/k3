\input{/Users/daniel/github/config/preamble.sty}%available at github.com/danimalabares/config
\input{/Users/daniel/github/config/thms-eng.sty}%available at github.com/danimalabares/config


\usepackage[style=authortitle-terse,backend=bibtex]{biblatex}
\addbibresource{bibliography.bib}

\setcounter{secnumdepth}{0}

\begin{document}

\begin{minipage}{\textwidth}
	\begin{minipage}{1\textwidth}
		K3 surfaces \hfill Daniel González Casanova Azuela
		
		{\small Prof. Misha Verbitsky\hfill\href{https://github.com/danimalabares/k3}{github.com/danimalabares/k3}}
	\end{minipage}
\end{minipage}\vspace{.2cm}\hrule

\vspace{10pt}
{\huge Home assignment 4: quadratic lattices}

\begin{idea5}{Definition 4.1}\leavevmode
	A \textit{\textbf{lattice}} is a finitely generated torsion-free $\mathbb{Z}$-module. \textit{\textbf{Quadratic form}} on a lattice is a function $q:L\to \mathbb{Z}$, $q(\ell)=B(\ell,\ell)$ where $B$ is a bilinear symmetric pairing $B:L\otimes_\mathbb{Z}L\longrightarrow \mathbb{Z}$. \textit{\textbf{Quadratic lattice}} is a lattice equipped with a quadratic form. A quadratic form is \textit{\textbf{indefinite}} if it takes positive and negative values, and \textit{\textbf{unimodular}} if $B$ is non-degenerate and defines an isomorphism $L\xrightarrow{\sim}L^*$.
\end{idea5}

\begin{idea4}{Exercise 4.1}\leavevmode
	Let $(L,q)$ be a quadratic lattice, $L_{\mathbb{Q}}:=L\otimes_{\mathbb{Z}}\mathbb{Q}$, and $L^*$ the set of all $x\in L_{\mathbb{Q}}$ such that $q(x,L)\subset \mathbb{Z}$.
	\begin{enumerate}[label=\alph*.]
		\item Prove that $L^*$ is a lattice of the same rank as $L$ and $L\subset L^*$.
		\item The \textit{\textbf{dscriminant group}} of $L$ is $\operatorname{Disc}L:=L^*/L$. Prove that $L$ is unimodular if and only if $\operatorname{Disc}(L)=\{0\}$.
		\item Let $G$ be an abelian group. Construct a lattice $(L,q)$ such that $\operatorname{Disc}(L)=G$.
	\end{enumerate}
\end{idea4}

\begin{proof}[Solution]\leavevmode
\begin{enumerate}[label=\alph*.]
	\item \begin{idea3}{Answer}\leavevmode
			Let $\{a_i\}$ be a basis of $L$. Recall that the space of linear functionals on $L$ is identified with  $L^*$ via the map $x\mapsto q(x,\cdot )$. Then the functionals given by $a^\vee_i(a_j)= \delta_{ij}$ are a basis of $L^*$.
		\end{idea3}
\vspace{1em}

		$\mathsf{OK}$ its rather stupid but I would like to make the following identification explicit:
		\begin{align*}
			L^* &\longrightarrow L^\star \\
			x &\longmapsto q(x,\cdot )
		\end{align*}
		where $L^\star$  is the set of linear forms on $L$. {\color{1}But the kernel of this map is trivial only when $q$ is nondegenerate.}

		So let us proceed as follows: we will show that there is only one element $a^\vee_i\in L_\mathbb{Q}$ such that $q(a^\vee_i,a_j)=\delta_{ij}$. Then suppose $\tilde{a}^\vee _i$ is another such element. Then $q(a^\vee_i-\tilde{a}^\vee_i,a_j)=0$ so if $q$ is nondegerate I'm done, but if $q$ is not nondegenerate {\color{1}How to prove $a^\vee_i=\tilde{a}^\vee_i$?}

\iffalse
		I want to find a set of generators for elements in $L^*$. Suppose $x\in L^*$. This means, first, that $x$ is a rational linear combination of the generators of  $L$, that is, $x=q_1a_1+\ldots +q_na_n$ where $L=\left<a_1\right> \oplus \ldots \left<a_n\right> $. And also 
		\[q(x,\ell)=q\left( \sum_{i}q_ia_i,\ell \right) =\sum_{i}q_iq(a_i,\ell)\in\mathbb{Z}\]
		where of course $q(a_i,\ell)$ are also integers. So in the end we have that
		\[q(x,\ell)\text{ is a rational linear combination of integers that is an integer} \]
		So maybe characterize $q$. What is it? Isn't $q\longleftrightarrow \big(q(a_i,a_j\big)$? Then I expand further to get
		\[q(x,\ell)=\sum_{i,j}q_i\ell_jq(a_i,a_j)\in\mathbb{Z}\]

	Consider the canonical identification of
		\begin{align*}
			L &\longrightarrow L^\star\\
			x &\longmapsto q(x,\cdot)
		\end{align*}
		where $L^\star$  is the set of linear forms on $L$. I expect to find an identification  $L^*\cong L^\star$. What exactly is $q(x,\ell)?$. Since $x=\sum_{i}q_ie_i$ with $q_i\in\mathbb{Q}$, we\fi

	\item Implication  $\implies $ is trivial. Implication $\impliedby$ is also straightforward since $L^* /L=\{0\}$ means that every element of $L^*$ is in $L$, so $L=L^*$.

	\item 
\end{enumerate}	
\end{proof}

\begin{idea5}{Exercise 4.2}\leavevmode
	Let $(L,q)$ be a quadratic lattice, and $L_1\subset L$ a sublattice.
\end{idea5}

\end{document}
