\input{/Users/daniel/github/config/preamble.sty}

\usepackage[style=authortitle-terse,backend=bibtex]{biblatex}
\addbibresource{bibliography.bib}

\setcounter{secnumdepth}{0}

\begin{document}

\begin{minipage}{\textwidth}
	\begin{minipage}{1\textwidth}
		K3 surfaces \hfill Daniel González Casanova Azuela
		
		{\small Prof. Misha Verbitsky\hfill\href{https://github.com/danimalabares/k3}{github.com/danimalabares/k3}}
	\end{minipage}
\end{minipage}\vspace{.2cm}\hrule

\vspace{10pt}
{\huge Home assignment 2: spectral sequences}

\setcounter{section}{2}
\subsection{The monodromy of Gauss-Manin local system}

\begin{manualdef}{2.1}
	Let $\pi:E\to B$ be a locally trivial fibration with fiber $F$. The family of cohomology of fibers of $\pi$ is locally trivial, {\color{magenta}(what does this mean precisely?)} but it might have \textit{\textbf{the monodromy}}. In other words, the group $\pi_{1}(B)$ naturally acts on the algebra $H^{*}(F)$ by autmorphisms. To obtain this action, take a loop in $B$ and trivialize the family $ \pi$ along small intervals of this loop; this gives an identification of $H^{*}(F)$ with itself, which might be non-trivial.
\end{manualdef}

\begin{remark}[Understanding the monodromy action of cohomology]
	(From \href{https://math.stackexchange.com/questions/2845794/monodromy-on-cohomology}{StackExchange}) Let $f:X\to U$ be a proper surjective submersion and fix $u_0\in U$.

	For any path $\gamma \subset U_j$, there is a canonical diffeomorphism $\varphi_\gamma:f^{-1}(\gamma(0))\to f^{-1}(\gamma(1))$, using $\psi_j$ (by a theorem of Ehresmann, all the fibers of $f$ are diffeomorphic).

	Now, for any loop $\gamma$, split $\gamma$ into paths $\gamma_i\subset U_i$ and you can compose these diffeomorphisms to get a diffeomorphism
	 \[\phi_{\gamma_n}\circ \ldots\circ \varphi_{\gamma_1}:f^{-1}(u_0)\to f^{-1}(\gamma(u_0))\]
	 It induces a map on homology: you can check that it is well defined up to homotopy.
\end{remark}

\begin{manualexercise}{2.1}
	Let $ \phi^* :\mathbb{Z}\to \operatorname{Aut}(H^*(F))$ be an automorphism induced by a homeomorphism $\phi:F\to F$. Construct a locally trivial family over a circle with monodromy in cohomology induced by $\phi^*$. 
\end{manualexercise}

\paragraph{Interpretation} Given an action $\phi^*:\mathbb{Z}=\pi_{1}(S^1) \to \operatorname{Aut}(H^{*}(F))$, construct a fibre bundle such that $\phi^*$ is the monodromy action on cohomology.

\begin{proof}
	Consider the standard torus fibration $T^2\to S^1$. Any path in the circle can be thought of as an number $ n \in\mathbb{Z}$. Perhaps the induced automorphism on cohomology is precisely the map $\mathbb{Z}\ni a \mapsto na\in\mathbb{Z}$. But I'm not looking for an automorphism of $\mathbb{Z}$… I need an automorphism of $H^\bullet(S^1)\cong \mathbb{Z}\oplus \mathbb{Z}$…


\end{proof}

\begin{manualexercise}{last}
	Generators here (horizontal), generators there (vertical, 1,3,5), "Extend generators by Leibniz rule, and then they just kill everyting"
\end{manualexercise}

\end{document}
