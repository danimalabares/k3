\input{/Users/daniel/github/config/preamble.sty}

\usepackage[style=authortitle-terse,backend=bibtex]{biblatex}
\addbibresource{bibliography.bib}

\setcounter{secnumdepth}{0}

\begin{document}

\begin{minipage}{\textwidth}
	\begin{minipage}{1\textwidth}
		K3 surfaces \hfill Daniel González Casanova Azuela
		
		{\small Prof. Misha Verbitsky\hfill\href{https://github.com/danimalabares/k3}{github.com/danimalabares/k3}}
	\end{minipage}
\end{minipage}\vspace{.2cm}\hrule

\vspace{10pt}
{\huge Home assignment 2: spectral sequences}

\setcounter{section}{2}
\subsection{The monodromy of Gauss-Manin local system}

\begin{manualdef}{2.1}
	Let $\pi:E\to B$ be a locally trivial fibration with fiber $F$. The family of cohomology of fibers of $\pi$ is locally trivial, {\color{magenta}(what does this mean precisely?)} but it might have \textit{\textbf{the monodromy}}. In other words, the group $\pi_{1}(B)$ naturally acts on the algebra $H^{*}(F)$ by autmorphisms. To obtain this action, take a loop in $B$ and trivialize the family $ \pi$ along small intervals of this loop; this gives an identification of $H^{*}(F)$ with itself, which might be non-trivial.
\end{manualdef}

\begin{remark}[Understanding the monodromy action of cohomology]
	(From \href{https://math.stackexchange.com/questions/2845794/monodromy-on-cohomology}{StackExchange}) Let $f:X\to U$ be a proper surjective submersion and fix $u_0\in U$.

	For any path $\gamma \subset U_j$, there is a canonical diffeomorphism $\varphi_\gamma:f^{-1}(\gamma(0))\to f^{-1}(\gamma(1))$, using $\psi_j$ (by a theorem of Ehresmann, all the fibers of $f$ are diffeomorphic).

	Now, for any loop $\gamma$, split $\gamma$ into paths $\gamma_i\subset U_i$ and you can compose these diffeomorphisms to get a diffeomorphism
	 \[\phi_{\gamma_n}\circ \ldots\circ \varphi_{\gamma_1}:f^{-1}(u_0)\to f^{-1}(\gamma(u_0))\]
	 It induces a map on homology: you can check that it is well defined up to homotopy.
\end{remark}

\begin{manualexercise}{2.1}
	Let $ \phi^* :\mathbb{Z}\to \operatorname{Aut}(H^*(F))$ be an automorphism induced by a homeomorphism $\phi:F\to F$. Construct a locally trivial family over a circle with monodromy in cohomology induced by $\phi^*$. 
\end{manualexercise}

\paragraph{Interpretation} Given an action $\phi^*:\mathbb{Z}=\pi_{1}(S^1) \to \operatorname{Aut}(H^{*}(F))$, construct a fibre bundle such that $\phi^*$ is the monodromy action on cohomology.

\begin{proof}
	Consider the standard torus fibration $T^2\to S^1$. Any path in the circle can be thought of as an number $ n \in\mathbb{Z}$. Perhaps the induced automorphism on cohomology is precisely the map $\mathbb{Z}\ni a \mapsto na\in\mathbb{Z}$. But I'm not looking for an automorphism of $\mathbb{Z}$… I need an automorphism of $H^\bullet(S^1)\cong \mathbb{Z}\oplus \mathbb{Z}$…


\end{proof}

\section{Leray-Serre spectral sequence}

\addcontentsline{toc}{subsection}{Exercise 2.4}
\begin{manualexercise}{2.4}
	Let $\pi:E\longrightarrow B$ be a fibration with the fiber a torus. Assume the $d_2=0$. Prove that all differentials vanish.
\end{manualexercise}

\begin{proof}
	Since $d_2=0$, we have that $E_3^{p,q}=E_2^{p,q}$. Then 
	\[d_3:H^{p}(B)\otimes H^{q}(T)\longrightarrow H^{p+3}(B)\otimes H^{q-2}(T),\] so the only way it could be non-zero is for $q=2$, which implies that 
	\begin{align*}
		H^p(B)\otimes H^{2}(T)\cong H^{p+3}(B)\otimes H^{0}(T)&\iff H^{p}(B)\cong H^{p+3}(B)
	\end{align*}
But I don't see why this couldn't happen…
\end{proof}

\addcontentsline{toc}{subsection}{Exercise 2.5}
\begin{manualexercise}{2.5}
	Let $\pi:E\to B$ be a fibration with the fiber a torus. Assume that the pullback map $\pi^* :H^2(B)\to H^{2}(E)$ is injective. Prove that all differentials $d_i$ vanish.
\end{manualexercise}

\begin{proof}[Solution]\leavevmode
	Recall that since  we are taking coefficients in a field, 
\end{proof}

\addcontentsline{toc}{subsection}{Exercise 2.8}
\begin{manualexercise}{2.8}
	Let $F=S^k$, that is, $\pi:E\to B$is a sphere bundle. Prove that all differentials $d_{k+1}$ vanish. Construct the \textit{\textbf{Gysin exact sequence}} 
	\[\begin{tikzcd}[column sep=small]
		\cdots\arrow[r]&H^{p}(B)\arrow[r]&H^{p+k+1}(B)\arrow[r,"\pi^*"]&H^{p+k+1}(E)\arrow[r]&H^{p+1}(B)\arrow[r]&\cdots
	\end{tikzcd}\]
\end{manualexercise}

\begin{proof}[Solution]\leavevmode
	(This argument is adapted from the construction of Wang exact sequence found in \href{https://en.wikipedia.org/wiki/Spectral_sequence#Wang_sequence}{Wikipedia}). We have that $E_2^{p,q}=H^{p}(B)\otimes H^{q}(S^k)$ can only be non-zero for $q=0,k$. This means that the only non-zero differentials are of the form
	\[d_{k+1}:E_2^{p,k}\cong H^{p}(B)\longrightarrow E_2^{p+k+1,0}\cong H^{p+k+1}(B)\]
	\[\begin{tikzcd}[column sep=tiny]
	H^{0}(B)\otimes H^{k}(S^k)  \arrow[rrdd,"d_{k+1}"]&H^{1}(B)\otimes H^{k}(S^k)\arrow[rrdd,"d_{k+1}"]&\cdots &H^{k}(B)\otimes H^{k}(S^k)\\
	\vdots &&&\vdots \\
	H^{0}(B)\otimes H^{0}(S^k)&\cdots&H^{k+1}(B)\otimes H^{0}(S^k)&H^{k+2}(B)\otimes H^{0}(S^k) 
\end{tikzcd}\]
which means that $E^{k+1}=E^\infty$. Since $E^{k+1}=\ker d_{k+1}/\operatorname{img} d_{k+1}$, we can write
	\[\begin{tikzcd}[row sep=tiny]
		0\arrow[r]&\ker d_{k+1}\arrow[d,equals]\arrow[r]&H^{p}(B)\arrow[r,"d_{k+1}"]&H^{p+k+1}(B)\arrow[r]&\operatorname{coker} d_{k+1}\arrow[r]\arrow[d,equals]&0\\
			  & \frac{\ker d_{k+1}}{\operatorname{img} d_{k+1}}\arrow[d,equals]&&&\frac{E^{p+k+1,0}}{\operatorname{img} d_{k+1}}\arrow[d,equal]\\
			  & E_{k+1}^{p,k}\arrow[d,equal]&&& E_{k+1}^{p+k+1,0}\arrow[d,equal]\\
		& E_\infty^{p,k}&&& E_\infty^{p+k+1,0}
	\end{tikzcd}\]
	This is the "first half" of the Gysin sequence. For the other half we must compute the $E_{\infty}$ terms. We use the filtration
	\[H^{n}(E)=F^0H^n\supset F^1H^n\supset  \ldots \supset F^nH^n\]
	that we know to satisfy
	\[E^{p,q}_{\infty}\cong \frac{F^pH^{p+q}}{F^{p+1}H^{p+q}}.\]
	We may write (I'm not completely sure why this works)
	\[\begin{tikzcd}[row sep=small]
		0\arrow[r]&E_{\infty}^{p+k+1,0}\arrow[d,equal]\arrow[r]&H^{p+k+1}(E)\arrow[r]&E^{p+1,k}_\infty\arrow[d,equal]\arrow[r]&0\\
			  &\frac{F^{p+k+1}H^{p+k+1}}{F^{p+k+2}H^{p+k+1}}&&\frac{F^{p+1}H^{p+k+1}}{F^{p+2}H^{p+k+1}}
	\end{tikzcd}\]
	Putting this together with the first sequence we computed, we get that
	\[\begin{tikzcd}[column sep=small]\leavevmode \arrow[r]&E^{p,k}_{\infty}\arrow[r]&H^{p}(B)\arrow[r,"d_{k+1}"]&H^{p+k+1}(B)\arrow[r]& E^{p+k+1,0}_\infty\arrow[r]&H^{p+k+1}(E)\arrow[r]& E_{\infty}^{p+1,k}\arrow[r]&\leavevmode\end{tikzcd}\]
	and we simply remove the $E_{\infty}$ terms to get the Gysin sequence
	\[\begin{tikzcd}\leavevmode \arrow[r]&H^{p}(B)\arrow[r,"d_{k+1}"]&H^{p+k+1}(B)\arrow[r]&H^{p+k+1}(E)\arrow[r]&H^{p+1}(B)\arrow[r]&\leavevmode\end{tikzcd}\]
	\begin{remark}
		I still cannot see why the map $H^{p+k+1}(B)\to H^{p+k+1}(E)$ is the map induced by the projection.
	\end{remark}
\end{proof}

\addcontentsline{toc}{subsection}{Exercise 2.11}
\begin{manualexercise}{2.11}
	Let $\pi:E\to B$ be a fibration with $B=S^k$. Prove that all differentials except $d_k$ vanish. Construct an exact sequence
	\[\begin{tikzcd}[column sep=small]
		\cdots\arrow[r]&H^{p+k}(F)\arrow[r,"\tilde{d}_k"]&H^{p}(F)\arrow[r,"\mu"]&H^{p+k}(E)\arrow[r]&H^{p+k+1}(F)\arrow[r]&\cdots
	\end{tikzcd}\]
	where $\mu$ is multiplication by $\pi^*\operatorname{Vol}_{S^k}$ and $\tilde{d}_{k}$ is equal to $d_k$ after the identification $H^{p}(F)=H^{k}(S^k)\otimes H^{p}(F)=E^{k,p}_2$
\end{manualexercise}

\begin{proof}[Solution]\leavevmode
\iffalse The argument to construct the Wang exact sequence
\[\begin{tikzcd}[column sep=small]
	\cdots\arrow[r]&H_{q}(F)\arrow[r]&H_{q}(E)\arrow[r]&H_{q-n}(F)\arrow[r]&H_{q-1}(F)\arrow[r]&H_{q-1}(E)\arrow[r]&H_{q-n-1}(F)\arrow[r]&\cdots
\end{tikzcd}\]
is analogous to exercise 2.8.\fi
Like in Exercise 2.8 we see that the only non-zero differentials are
\[d_k:H^{0}(S^k)\otimes H^{k+p}\longrightarrow H^{k}(S^k)\otimes H^{p+1}(F)\]
since $E_{2}=E_k$ is of the form
\[\begin{tikzcd}[row sep=tiny]
H^{0}(S^k)\otimes H^{k+1}(F)\arrow[rrddd,"d_k"]&\cdots & H^k(S^k)\otimes H^{k+1}(F)\\
	H^{0}(S^k)\otimes H^{k}(F)  \arrow[rrddd,"d_{k}"]&\cdots &H^{k}(S^k)\otimes H^{k}(F)\\
	\vdots &&\vdots \\
	%H^{0}(S^k)\otimes H^{3}(F)&\cdots &H^{k}(S^k)\otimes H^{3}(F)\\
	H^{0}(S^k)\otimes H^{2}(F)&\cdots &H^{k}(S^k)\otimes H^{2}(F)\\
	H^{0}(S^k)\otimes H^{1}(F)&\cdots&H^k(S^k)\otimes H^{1}(F)\\
	H^{0}(S^k)\otimes H^{0}(F)&\cdots&H^{k}(S^k)\otimes H^{0}(F) 
\end{tikzcd}\]
Again like in Exercise 2.8 we obtain a sequence
\[\begin{tikzcd}
	0\arrow[r]&E_{\infty}^{0,q}\arrow[r]& H^{q}(F)\arrow[r,"d_k"]&H^{q-k+1}(F)\arrow[r]& E^{k,q-k+1}_{\infty}\arrow[r]&0
\end{tikzcd}\]
\begin{remark}
	The exercise has the map $H^{p+k}(F)\overset{\tilde{d}_{k}}{\longrightarrow} H^{p}(F)$, but my computations suggest it should be $H^{p+k}(F)\overset{\tilde{d}_k}{\longrightarrow}H^{p+1}(F)$.
\end{remark}
Then we compute the $E_{\infty}$ terms using a filtration
\[H^{n}(E)=F^0H^n\supset F^1H^n\supset  \ldots \supset F^nH^n,\qquad E_\infty^{p,q}=\frac{F^pH^{p+q}}{F^{p+1}H^{p+q}}\]
which yields
\[\begin{tikzcd}
	0\arrow[r]&E^{k-1,q-k+1}_{\infty}\arrow[r]&H^{q}(E)\arrow[r]&E^{0,q}_{\infty}\arrow[r]&0
\end{tikzcd}\]
and then we get
\[\begin{tikzcd}[column sep=small]
	\cdots\arrow[r]&H^{q}(E)\arrow[r]&H^{q}(F)\arrow[r]&H^{q-k+1}(F)\arrow[r]&H^{q+1}(E)\arrow[r]&H^{q+1}(F)\arrow[r]&\cdots
\end{tikzcd}\]
\begin{remark}
	As in Exercise 2.8, I don't know why the map $H^{q-k+1}(F)\to H^{q+1}(E)$ should be multiplication by the volume form of $S^k$.
\end{remark}


\end{proof}

\begin{manualexercise}{last}
	Generators here (horizontal), generators there (vertical, 1,3,5), "Extend generators by Leibniz rule, and then they just kill everyting"
\end{manualexercise}

\end{document}
