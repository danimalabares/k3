\input{/Users/daniel/github/config/preamble.sty}

\begin{document}

{\Huge k3}

\section{Class 1}

The most important invariant of a k3 surface is \href{https://en.wikipedia.org/wiki/Intersection_form_of_a_4-manifold}{intersection form}.

There are three classes of manifolds
\begin{enumerate}
	\item Smooth manifolds
\end{enumerate}

\[\begin{tikzcd}
	\text{smooth manifolds}\arrow[r,"\text{forgetful functor} "]&\text{PL manifold}\arrow[r]&\text{Topological manifolds}   
\end{tikzcd}\]

Donaldson: contiunally many non-equivalent smooth structures on $\mathbb{R}^{4}$. K3 surfaces has countably many smooth structures and only one of them is compatible with complex structure.

\begin{defn}
	Intersection form. Given a quadratic form on a lattice $V_{\mathbb{Z}}=\mathbb{Z}^n$, so 
	 \[q:V_{\mathbb{Z}}\times V_{\mathbb{Z}}\to \mathbb{Z}\]
	 is \textit{\textbf{unimodular}} if 
	 \[V_{\mathbb{Z}}\overset{q}{\longrightarrow} \operatorname{Hom}(V_{\mathbb{Z}},\mathbb{Z})\]
	 is an isomorphism.
\end{defn}

\begin{thm}[Universal coefficients formula]\leavevmode
	\[H_{n-1}(M,\mathbb{Z})=\mathbb{Z}^{b_{n-1}(M)}\oplus T_{n-1}(M)\]
	\[h^n(M,\mathbb{Z})=\mathbb{Z}^{b_{n}(M)}\oplus T_{n-1}(M)\]
\end{thm}

\begin{coro}
	$H^2(X,\mathbb{Z})$ is torsion free if $\pi_{1}(X) =0$ because 
\end{coro}

\begin{defn}
	\textit{\textbf{Signature}} is $m-n$ if $q$ has signature $(m,n)$.
\end{defn}

\begin{thm}[Rokhlm-Wu?]\leavevmode
	Signature is divisible by 16 for simply-connected (something else).
\end{thm}

\begin{remark}
	The methods used in surgery break down in smooth case because strange topological objects like infinite sums of spheres arise.
\end{remark}

\begin{thm}[Freedman, 1982]
	There are as many 4-manifolds as there are intersection forms. $M$ simply connected 4 manifold homotopy class is uniquely determined by intersection dorm. Moreover, for every unimodular form there exists a unique $M$ with this intersection form.
\end{thm}

\begin{thm}[Donaldson, 1986]\leavevmode
	$M$ smooth compact manifold with positive definite odd intersection form $q$. Then
	\[\begin{pmatrix} 1&0&0\\0&1&0\\0&0&0&1 \end{pmatrix} \]
\end{thm}

\begin{defn}
	Bilinear symmetric form is \textit{\textbf{indefinite}} if it is not positive definite nor negative definite.
\end{defn}

\begin{thm}[Classification of unimodular symmetric bilinear forms]\leavevmode
	Odd are diagonalizable, while even are related to special Lie group $E_8$.
\end{thm}

\begin{defn}
	A \textit{\textbf{K3 surface}} is a K\"ahler complex surface $M$ with $b_1=0$ (simply connected) and $c_1(M,\mathbb{Z})=0$.
\end{defn}

Kodaira did what Andr\'e Weil couldn'g classify.

 \begin{thm}
	K3 surfaces have trivial canonical bundle $K_{M}=\Lambda^2(\Omega^1M)$.
\end{thm}

\section{Class 2}

$G$ topological group. \textit{\textbf{Principal $G$ bundle}} is a space with free $G$-action such that the quotient $E/G$ is Housdorff. There are several conditions that make this work. And then you have $\operatorname{Homo t o py}(X,BG)=$equivalence classes of $G$-bundles.
Vector bundles of a manifold are the same as maps from $X$ to $B\operatorname{U}(n)$.

Vector bundles up to stable equivalence are classified basically by Chern classes, so by the cohomology in $H^\bullet(B\operatorname{U})=Q[c_1,c_2,\ldots,c_n$.

Now look at the loop space of $X$. Then $H^{\bullet}(\Omega X)$ is a free graded commutative algebra. Loop space has the interesting property that $\Omega \operatorname{U}=B\operatorname{U}$ and $\Omega B\operatorname{U}=\operatorname{U}$.

\subsection{Bialgebras}

Let $A$ be a superalgebra (graded with antisymmetric product). Then we ask the axiom of coassociativity and that .

\begin{example}
	$G$ group, and $C(G)$ the ring of  $k$-valued functions $C(G\times G)=C(G)\times C(G)$ so
	\begin{align*}
		G\times G &\longrightarrow G \\
		C(G) &\longmapsto C(G)\otimes C(G)
	\end{align*}
	
\end{example}

\subsection{H-spaces}

\begin{defn}
	$H$-space is a space $M$ with a map $\mu:M\times M\\to M$that is homotopy associative,
	\[\begin{tikzcd}
		M\times M\times M\arrow[r,"\mu\times \operatorname{id}"]\arrow[d,"\operatorname{id}\times \mu"]&M\times M\arrow[d,"\mu"]\\
		M\times M\arrow[r,"\mu"]&M
	\end{tikzcd}\]
	which is homotopy commutative. And with homotopy unit.
\end{defn}

So it's like a homotopy algebra?

\begin{example}
	The loop space.
\end{example}

\subsection{Bialgebras of finite type}

\begin{defn}
	A bialgebra $A$ is of \textit{\textbf{finite type}} if it is the direct sum of $A=\bigoplus_{i\geq 0} A^i $ supercommutative and each $A^1$ is finite dimensional.
\end{defn}

\begin{remark}
	Free commutative algebra is polynomial algebra
\end{remark}

\begin{defn}
	$A=\mathbb{C}[x_1,\ldots,x_n,\ldots]\otimes \Lambda^\bullet(a_1,\ldots,a_n,\ldots)$ is a graded commutative free algebra. In the slides: it is $\operatorname{Sym}_{\operatorname{gr}}V^*$ where $V^*$ is a graded vector space.
\end{defn}

\begin{thm}[Hopf]\leavevmode
	A graded commutative bialgebra of finite type over $k$ of 0 characteristic is free graded commutative as a $k$ algebra.
\end{thm}

\subsection{The cohomology algebra of U(n)}

\begin{claim}
	The cohomology algebra $H^{*}(\operatorname{U}(n),\mathbb{Q})$ is a free graded commutative algebra with generators in degrees $1,3,5,\ldots,2n-1$.
\end{claim}

\begin{proof}[Demostra\c c\~ao]
	Induction.  $\operatorname{U}(1)$ is clear because it is a circle. Then do Serre spectral sequence. Differentials vanish on the second page because there's only nonzero groups on even degrees! And we get that $E_2^{p_1}=H^{p}(S^{2n-1})\otimes H^{q}(\operatorname{U}(n-1))$. And then the sequence converges to that of the total space which is $\operatorname{U}(n)$.
\end{proof}

\subsection{Grassman manifolds}

\begin{defn}
	The \textit{\textbf{fundamental bundle}} $B_{\operatorname{fun}}$ is a rank $n$ vector bundle over $\operatorname{Gr}(n,m)$.
\end{defn}

\begin{claim}
	$B$, $B'$ vector bundles of rank $n$, $m-n$, $B\oplus B'$
	\[\varphi:X\to \operatorname{Gr}(m,n)\]
	\[\varphi (x)=B_x\subset B_x\oplus B_x'=\mathbb{K}^m\]
	then $B=\varphi^* B_{\operatorname{fun}}$.
\end{claim}

\begin{thm}\leavevmode
	If you have $B$ as a bundle on a manifold $X$ then $B\oplus B'$ is trivial for some bundle $B'$.
\end{thm}

\begin{proof}[Demostra\c c\~ao]
	Embed the total space in a large enough euclidean space.
\end{proof}

\begin{defn}
	$\operatorname{Gr}(n,\infty)=\operatorname{Gr}(n)$ is $\bigcup_{m=n_1}^\infty\operatorname{Gr}(n,m)=\operatorname{Gr}(n) $
\end{defn}

\begin{coro}
	For every bundle  $B$ of rank $n$ there is a function $\varphi:X\to \operatorname{Gr}(n)$ such that $B=\varphi^*B_{\operatorname{fun}}$.
\end{coro}

Take a bundle $E\to X$ and $G$ acts freely on $E$ so $E$ principal $G$ bundle. Classifying space  $BG$

\begin{thm}[Atiyah-Bott]\leavevmode
	Classifying space is unique up to homotopy equivalence.
\end{thm}

\subsubsection{The fundamental bundle}

In class 4 I finally understood that

\begin{defn}
	The \textit{\textbf{fundamental bundle}} on the Grassmanian  $\operatorname{Gr}(n)$ (the Grassmanian is this space where points are linear spaces) is the vector bundle such that the fiber of one point (which is a vector space) is the vector space that is the point. It's very tautological.
\end{defn}

\begin{thm}[Did we prove this?]\leavevmode
	Let $B$ be a vector bundle of rank $n$ on a cellular space $X$. Then there exists a continuous map $\varphi:X\to \operatorname{Gr}(n)$ such that $ B$ is isomorphic to the pullback $\varphi^*B_{\operatorname{fun}}$ of the fundamental bundle.
\end{thm}

\begin{remark}
	In fact $\operatorname{Gr}(n)$ is the classifying space of vector bundles of rank $n$, in the sense that isomorphism classes of vector bundles of maps $\varphi:X\to \operatorname{Gr}(n)$.
\end{remark}

\subsection{Stiefel spaces}

\begin{defn}
	$\mathbb{K}^\infty$ is the direct limit of $\mathbb{K}^n$ so its just the direct sum $\bigoplus_{i=n}^\infty\mathbb{K} $. Stiefel space is the space of orthonormal $n$-frames.
\end{defn}

If we prove that Stiefel is contractible we obtain our classifying space so let's prove that. We have a fibration

\[\operatorname{U}(n)\hookrightarrow \operatorname{St}(n,\infty)\to \operatorname{Gr}(n,\infty)\]

\begin{thm}
	$\operatorname{St}(n)$ is contractible.
\end{thm}

\begin{proof}[Demostra\c c\~ao]
	\begin{enumerate}[label=\textbf{Step \arabic*}]
		\item Locally trivial fibration with contractible fiber and base $Y\to X$ then $Y$ is contractible, this is so trivial.

		\item Fibration $ \operatorname{St}(n)\to \operatorname{St}(n-1)$ with fiber $S^\infty$ 

		\item Show that $S^\infty$ is contractible.

		\item And then some map $\mathbb{R}$ that is not surjective, and construct homotopy of identity to a constant map.
	\end{enumerate}
\end{proof}

\begin{exercise}
	If $X_{\infty}=\bigcup X_{i} $ is the inductive limit of contractible cellular spaces then it is contractible. Use Whitehead theorem.
\end{exercise}

\begin{thm}[Important]\leavevmode
	$\operatorname{Gr}(\infty)=B\operatorname{U}$.
\end{thm}

\subsection{Stable equivalence}

\begin{defn}
	Vector bundles $V$,  $W$ are stable equivalent if  $V\oplus A\cong W\cong B$ for trivial vector bundles  $A$ and $B$.
\end{defn}

Homotopy classes of equivalent vector bundles are in coorespondance with…

\begin{thm}
	$B\operatorname{U}$ is $H$-space.
\end{thm}

\begin{coro}
	$H^* (B\operatorname{U},\mathbb{Q})$ is a free supercommutative algebra.
\end{coro}

\begin{claim}
	$H^*(B\operatorname{U}$ is a free polynomial algebra generated by classes $c_1,c_2,\ldots$ in all even degrees.
\end{claim}

\section{Class 3}

\subsection{Reminder}

\begin{defn}
	\textit{\textbf{Bialgebra}} is an algebra that is  equipped with comultiplication, counit…
\end{defn}

\begin{remark}
	It is when the dual space also has an algebra structure, but we prefer to use the tensor notation.
\end{remark}

Let $\sum_{i \geq 0}A^i$ with $\dim A^i<\infty$. \textit{\textbf{Free commutative algebra}} is a polynomial algebra.  \textit{\textbf{Free graded commutative algebra}} is
\[\widetilde{\operatorname{Sym}}^\bullet(W^\bullet \oplus V^\bullet):=\operatorname{Sym}^\bullet(W^\bullet)\otimes \Lambda^\bullet(V^\bullet)\]
where
\[W=\bigoplus_{i} W^{\operatorname{even}} \qquad V=\bigoplus_{i}V^{\operatorname{odd}}. \]

\subsection{Hopf algebra}

\begin{defn}
	A bialgebra is a \textit{\textbf{Hopf algebra}} when it is also equipped with an antipode map ($S$) such that the following diagram commutes
\[\begin{tikzcd}
	&H\otimes H\arrow[rr,"S\otimes \operatorname{id}"]&&H\otimes H\arrow[rd,"m"]\\
	H\arrow[ur,"\Delta"]\arrow[dr,swap,"\Delta"]\arrow[rr,"\eta"]&&\mathbb{C}\arrow[rr,"u"]&&H\\
								     &H\otimes H\arrow[rr,"\operatorname{id}\otimes S"]&&H\otimes H\arrow[ur]
\end{tikzcd}\]
[diagram from quantum group minicourse notes]

	\end{defn}

\begin{example}
	The cohomology of the loop space, $H^{\bullet}(\Omega X)$.
\end{example}

\subsection{Primitive elements in a bialgebra}

\begin{defn}
	An element of a bialgebra $x\in A$ is \textit{\textbf{primitive}} if $\Delta(x)=x\otimes 1+1\otimes x$.
\end{defn}

\begin{align*}
	\Delta(xy)&=\Delta(x)\Delta(y)\\
	&=(1\otimes x+x\cdot 1)(y\otimes 1+y\otimes y)\\
	&=1\otimes xy+xy\otimes 1+x\otimes y+y\otimes x.
\end{align*}

\begin{remark}
	We trying to show that Hopf algebras? bialgebras? are generated by primitive elements?
\end{remark}

\begin{defn}
	$A^\bullet$ bialgebra, $\mathcal{P}^\bullet\subset A^\bullet$ space of primitive, and the natural embedding
	\[\operatorname{Sym}_{gr}(\mathcal{P}^\bullet)\to A\]
	We say that $A$ is \textit{\textbf{free up to defree $k$}} if
	\[\bigoplus_{i\leq k} \operatorname{Sym}_{\operatorname{gr}}^i(P)\overset{\psi}{\longrightarrow} A\]
	is an embedding.
\end{defn}

\begin{lemma}
	Let $A^\bullet$ be a bialgebra which is free up to degree $k$. Then $A^\bullet$ is free up to degree $k+1$.
\end{lemma}

\begin{proof}\leavevmode 
	\begin{enumerate}[label=\textbf{Step \arabic*}]
		\item Choose a basis of $P$, $\{x_i\}$. Chose a polynomial condition $Q(x_1,\ldots,x_n)=0$ of degree $k+1$. Write this as
			\[Q=Q_mx_1^m+Q_{m-1}x_1^{m-1}+\ldots+Q_0.\]
			that is
			\[Q=\sum_{i=0}^mQ_ix_1^i\]
			with $Q_i$ invariant somehow. Then we apply comutiplication to obtain
			\[\Delta(Q)=Q\otimes 1+1\otimes Q+R\]
			where $R$ is some sort of reminder with bounded degree:
			\[R\in\mathfrak{U}:=\bigoplus_{i\leq k} \operatorname{Sym}^i_{\operatorname{gr}}(P)\otimes \bigoplus_{i\leq k} \operatorname{Sym}_{\operatorname{gr}}^i(P)  \]
			which follows from a similar computation of that which we did after defining primitive elements.

			\item Project to drop the terms that have $Q\otimes 1+1\otimes Q$:
				\[\Pi:\mathfrak{U}\to x_1\otimes \bigoplus_{i\leq k}   \operatorname{Sym}^i_{\operatorname{gr}}(P)\]
				since the $x_i$ are primitive, i.e. $\Delta(x_i)=x_i\otimes 1+1\otimes x_i$, one has
				\[\Delta(x_1^m)=(x_1\otimes 1+1 \otimes x_1)^m\]
				we get that
				\[\Pi(\Delta(x_1^m))=mx_1\otimes x_1^{m-1}\]
				while on the board it is written that
				\[\Pi(\Delta(x_1^m))=\Pi((x_1\otimes 1+1\otimes x_1)^m)\]

			\item Let $\Pi(R):=x_1\otimes R_0$. Since $Q=0$ in $A$, its component $R_0$ is also equal to 0. So $\Pi(\Delta(Q))=0$. Then
				\begin{align*}
					0&=\Pi \left( \Delta \left( \sum_{m}x_1^m\cdot Q_m \right)  \right)\\
					 & =\sum_{m}x_1\otimes x_1^{m-1}Q_m+\Pi(mx_1\otimes x_1^{m-1}\cdot  \Delta(Q_m))\\
					& =\sum_{m}x_1\otimes x_1^{m-1}Q_m
				\end{align*}
				so that 
				\begin{align*}
					x_1\otimes x_1^{m-1}Q_m&=0\\
					\implies x_1^{m-1}Q_m&=0
				\end{align*}
				So we conclude that
				\[Q_m=0\]
		\end{enumerate}
	\end{proof}

\begin{remark}
	We just proved that for any subalgebra generated by finite elements, we didn't use that it is free.
\end{remark}

\subsection{Algebras with filtration}

\begin{defn}
	A \textit{\textbf{filtration on algebra}} is
	\[A^\bullet  \supset  F_1A^\bullet\supset F_2A^\bullet\supset\ldots\]
	such that
	 \[F_i A^\bullet F_j\subset F_{i+j}A^\bullet\]
\end{defn}

\begin{defn}
	\textit{\textbf{Associated graded}} to a filtered algebra is
\[A^\bullet_{\operatorname{gr}}=\bigoplus_{i=0}^\infty \dfrac{F^1A^\bullet}{F^{i+1}A^\bullet} \]
\[F^0A^\bullet=A^\bullet\]
\end{defn}

\begin{defn}
	$I\subset A$ ideal then \textit{\textbf{$I$-adic filtration}} is the filtration by the degrees of the ideal
	\[A\supset I\supset  I^2 \supset  I^3\ldots\]
\end{defn}

\begin{lemma}
	Choose an $I$-adic filtration. Then $A_{\operatorname{gr}}$ is generated by its first and second graded components $A/I\oplus I/I^2$.
\end{lemma}

\begin{proof}[Demostra\c c\~ao]
	Indeed, $I^k/^{k+1}$ is generated by products of $k$ elements in $(I/I^2)$.
\end{proof}

\begin{defn}
	A \textit{\textbf{augmentation ideal}} in a bialgebra is the kernel of the counit homomorphism $\varepsilon:A\to k$. We denote it by $Z=\ker A$
\end{defn}

\begin{remark}
\[\Delta(x)=1\otimes x+x\otimes 1 \operatorname{mod}Z\otimes Z\]
Why? Because
\begin{align*}
	x&=\varepsilon \otimes \operatorname{id}(\Delta(x))\qquad \text{up to $Z\otimes A$}\\
	\Delta(x)& =1\otimes x\qquad \text{up to $A\otimes X$} \\
	\Delta(x)&=x\otimes 1
\end{align*}
\end{remark}

Ok, now we can prove Hopf theorem.

\begin{thm}[Hopf theorem]\leavevmode
	A finite type bialgebra is generated by primitive elements.

	In slides: Let $A$ be a graded bialgebra of finite type over a field $k$ of characteristic 0. Then $A$ is a free graded commutative $k$-algebra.
\end{thm}

\begin{proof}\leavevmode 
	\begin{enumerate}[label=\textbf{Step \arabic*}]
		\item I think this is the computation above.

		\item $A_{\operatorname{gr}}$ is a bialgebra.

		\item $A_{\operatorname{gr}}$ is multiplicative generated by $Z^1/Z^2$. All elements $Z^1/Z_2$ are primitive, so this algrebra $A_{\operatorname{gr}}$ is generated by primitive elements.

		\item Let $\{x_i\}$ be a basis of primitive elements of $A_{\operatorname{gr}}$. Then lifts of $A$ have no relations because $A_{\operatorname{gr}}$ is already generated by primitive elements. Then there are no relations also for elements in $A^\bullet$ (I think).
	\end{enumerate}
\end{proof}

\subsection{Grassmanians (Reminder)}

$B$ vector bundle of rank $n$ on $X$ then there exists a map (essentialy unique) $\varphi:X\to \operatorname{Gr}(n)$ such that
\[\varphi^* (B_{\operatorname{fun}}=B\]
which makes the Grassmanian a classifying space, and $\operatorname{Gr}(1)=B\operatorname{U}(n)$.

The infinite Grassmanian is important.

\subsection{BU as an H-space (Reminder)}

Bott periodicity identifies the space of loops on $\operatorname{U}$ and $B\operatorname{U}$.

\begin{prop}
	Embed $\mathbb{C}^{\infty}\times \mathbb{C}^\infty$ into $\mathbb{C}^\infty$ taking the basis vectors of the first copy to the even basis vectors and the basis of the second copy to the odd. Then for $L_1\subset \mathbb{C}^\infty$, $L_2\subset \mathbb{C}^\infty$, the map
	\[L,L'\mapsto S(L,L')\]
	defines a structure of $H$-space on the infinite Grassmanian $B\operatorname{U}$.
\end{prop}

\begin{proof}
	Just show that $H$-associatity up to homotopy.
\end{proof}

\begin{coro}
	$H^\bullet(B\operatorname{U},\mathbb{Q})$ is a free supercommutative algebra.
\end{coro}
\begin{proof}
	Follows from Hopf theorem.
\end{proof}

\subsection{Cohomology of BU}

\begin{claim}
	$H^\bullet(B\operatorname{U},\mathbb{Q})$ is a free polynomial algebra generated by classes $c_1,c_2,\ldots$ in all even degrees.
\end{claim}

\begin{proof}[Demostra\c c\~ao]
	Leray-Serre spectral sequence.
\end{proof}

\subsection{Chern classes: axiomatic definition}

\begin{defn}
	\textit{\textbf{Chern classes}} are classes  $c_i(B)\in H^{2i}(X)$ for $i=0,1,2,\ldots$

	\textit{\textbf{Chern classes}} are  $c_i(B)\in H^{2i}(X)$ for a complex vector bundle $B$ over $X$ with axioms

	\begin{enumerate}[label=\alph*.]
		\item $c_0(B)=1$

		\item Functoriality (commutes with bullbacks): for $\varphi:X\to Y$ with $B$ bundle on $Y$, 
			 \[\varphi^*(c_i(B))=c_1(\varphi^*(B))\]

		\item Define \textit{\textbf{total Chern class}}  $c_{*}:=\sum_{i}c_i(B)$ then
			\[c_i(B)\cdot c_i(B')=c_*(B\oplus B')\qquad \text{(Whitney)} \]

		\item $\mathcal{O}(1)$ on $\mathbb{C}P^{n}$,
			\[c_i(\mathcal{O}(1)=1+[H]\]
			where $[H]$ is the fundamental class of a hyperplane section.
	\end{enumerate}
\end{defn}

Suppose we have a class $a\in H^{\bullet}(B\operatorname{U})$. Then for all $B$ on $X$ 
\[\varphi:X\to B\operatorname{U}\]
so
\[B\cong \varphi^* (B_{\operatorname{fun}})\]
and so
\[\varphi^*_B(c_*)=c_*(B).\]





\section{Class 4}

\subsection{Reminder}

For each rank $n$ bundle $B$ on $X$ there exists $\varphi_B:X\to \operatorname{Gr}(n,\infty)=B\operatorname{U}(n)$ such that $\varphi^* _B(B_{\operatorname{f un}}=B$.

The infinite grassmanian is classifying space for (?) stable bundles.

Some more review about $H$-space structure, primitive elements, a comment on last exercise of homework 2.

Chern classes of $\mathcal{O}(1)$ are hyperplane sections: $c_i(\mathcal{O}(1))=1+[H]$.

\subsection{The splitting principle}

\begin{exercise}
	Prove that $BU(1)=\mathbb{C}P^\infty$.
\end{exercise}

\begin{proof}[Solution]\leavevmode
	Hopf fibration on $S^\infty$? It's easier, take $n=1$, it's just by  definition.
\end{proof}

\begin{defn}
	The \textit{\textbf{fundamental bundle}} on $BU(1)^n$ has fiber
	\[\ell_1\oplus \ell_2\oplus \ldots\ell_n\]
	where $\ell_i\in BU(1)$ are product $\ell_1\times \ell_2\times \ldots\times \ell_n$.
\end{defn}

\begin{remark}
	Chern classes of $B_{\operatorname{f un}}$ are uniquely determined by axioms, because every factor has Chern classes, and fibers are just sums, and pullbacks preserve sums…
	\[B_{\operatorname{f un}}=\bigoplus_{i} \pi_i \mathcal{O}(1) \]
	where
	\[pi_i:BU(1)^n\to BU(1)\]
	is a projection.
\end{remark}

\begin{remark}
	$H^{\bullet}(BU(1))^n=\mathbb{Z}[z_1,\ldots,zn$ 
	Here at least I remember that the cohomology of $\mathbb{C}P^{\infty}$ is just polynomials so it looks reasonable that the $n$-th power is polynomials in more cariables.
\end{remark}

\begin{thm}[Splitting principle]\leavevmode
	Let $\varphi_{\operatorname{f un}}:BU(1)^n\to BU$, the \textit{\textbf{fundamental map}}, it induces embedding on cohomology up to degree $2n$. For all primer generator $\sigma_i\in H^{2}(BU)$, $\varphi_{\operatorname{f un}}(\sigma_1)=\lambda \sum_{i}z_i^k$ with $\lambda\neq 0$.

	So
	\[\begin{tikzcd}
		BU(1)^n \arrow[r]& BU\\
		X\arrow[u]\arrow[ur]
	\end{tikzcd}\]
	
\end{thm}

\begin{remark}
	\href{https://en.wikipedia.org/wiki/Classifying_space_for_U(n)}{Wiki} Thus, the set of isomorphism classes of circle bundles over a manifold $M$ are in one-to-one correspondence with the homotopy classes of maps from $M$ to $\mathbb{C}P^{\infty}$
\end{remark}

\begin{thm}
	Chern classes are unique (uniquely determined by axioms).
\end{thm}

\begin{proof}\leavevmode 
	\begin{enumerate}[label=\textbf{Step \arabic*}]
		\item Every bundle is obtained as pullback of the fundamental bundle. So for $A\in H^{\bullet}(BU)$ and $B$ bundle on $X$, $A(B)=\varphi^*_{B}(A)\subset H^{\bullet}(X)$ so $c_i(B)$ are obtained as pullbacks of $ c$ in the fundamental bundle.

		\item \[BU(1)^\infty \overset{\varphi_{\operatorname{f un}}}{\longrightarrow}BU\]
			pullback of fundamental bundle is fundamental. (This map is defined from the former by induction).
			\[\varphi^*_{\operatorname{fu n}}(c_i(B_{\operatorname{f un}})=c_i(B_{\operatorname{f un}}\text{ on $BU$} )\]
			The Chern classes of the fundamental bundle are already known. Since $\varphi^*_{\operatorname{f un}}$ is injective by the splitting principle we are done.
	\end{enumerate}
\end{proof}

\subsection{Primitive generators of $H^{*}(BU)$}

Recall the $H$-space multiplication:
\begin{align*}
	BU\times BU  &\longrightarrow BU \\
	L_1\times L_2 &\longmapsto L_1\oplus L_2
\end{align*}
and the comultiplication
\[\Delta :H^{\bullet}(BU)\to H^{\bullet}(BU)\]
Generators of $H^{\bullet}(BU)$ are $c_{h_1},c_{h_2},\ldots$ with $c_{h_i}\in H^{2i}(BU)$ and we have the comultiplication $\Delta(c_{h_i})=c_{h_i}\otimes 1+1\otimes c_{h_i}$.

\begin{remark}
	\[\varphi=(\varphi_1,\varphi_2):X\to BU\times BU\]
	and we can compose so we have
	\[\varphi\circ \mu:X\to BU\]
	what does this map do?
	\begin{align*}
		\varphi\circ \mu: X &\longrightarrow BU \\
		\varphi^* (B_{\operatorname{fun}} &\longmapsto B_1\\
		(\varphi\circ \mu)^* (B_{\operatorname{ fu n}})&=B_1\oplus B_2
	\end{align*}
	So then we have
	\begin{align*}
		\varphi^*&:H^{\bullet}(BU)\otimes H^{\bullet}(BU)\to H^{\bullet}(X)\\
		\Delta &:H^{\bullet}(BU)\to H^{\bullet}(BU)\otimes H^{\bullet}(BU)\\
		\Delta \circ \varphi^*&:H^{\bullet}(BU)\to H^{\bullet}(X)
	\end{align*}
\end{remark}

\begin{coro}
		For every $x \in H^{\bullet}(BU)$
		\[X(B_1\oplus B_2)=\Delta(x)(B_1,B_2)\]
	\end{coro}

	\begin{coro}
		If $x\in H^{*}(BU)$ is primitive, then $x(B_1\oplus B_2)=x(B_1)\oplus X(B_2)$.
	\end{coro}

	\begin{proof}
		$\Delta(x)=x\otimes 1+1\otimes x$ so $\Delta(x)$ evaluated on $(B_1,B_2)$
	\end{proof}

	\begin{remark}
	We will construct the full Chern class $c_*(B)$ as a pullback of a class $C\in H^{*}(BU)$.
\end{remark}

\begin{remark}
Then take exponential. Let $\chi_i\in H^{2i}(BU)$ be a primitive generator. Use Hopf theorem to see that it is unique by a constant. Snce $\chi_i(B_1\oplus B_2)=\chi_i(B_1)+\chi_i(B_2)$, the class $C=e^{\sum_{i}a_i\chi_i}=1+\ldots+\frac{\chi_n}{n!}+\ldots$ satisfies the Whitney formula.

To construct Chern classes satisfying the axioms it remains to arrange the coefficients $a_i$ in such a way that $C(\mathcal{O}(1))=1+[H]$ I think this means hyperplane section.
\end{remark}

\begin{lemma}
An embedding
	\[BU(1)\overset{\varphi}{\hookrightarrow}BU\]
	with $\chi_i\in H^{2i}(BU)$ primitive generator. Then $\varphi^*(\chi_i)\neq 0$
\end{lemma}

\begin{proof}
	$H^{\bullet}(BU)=$ symmetric polynomials in $H^{i}(BU(1))^n$, $\varphi_{\operatorname{fun}}(x_N)=x\sum_{i=1}^nz_i^k$ so $\varphi (x_k)=\lambda x_1^k$.
\end{proof}

\begin{remark}
	$\varphi^* (c_i(B_{\operatorname{fun}})=c_i(\mathcal{O}(1)=1+[H]$
\end{remark}

\begin{thm}
	Choose generators $\chi_i\in H^{2}(BU)$ primitive. Then $\varphi^* (\sum_{i}\chi_i=\log(1+[H])$ where the logarith is a formal power series, namely $\sum_{i=1}^\infty\frac{H^n}{n!}(-1)^n$.

	That means $c(B_{\operatorname{fun}})=\operatorname{exp}\left(\sum_{\chi_i}\right)$.
\end{thm}
	
	
\section{Class 5}


\end{document}


