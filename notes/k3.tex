\input{/Users/daniel/github/config/preamble.sty}
\input{/Users/daniel/github/config/thms-eng.sty}

\begin{document}

\begin{minipage}{\textwidth}
	\begin{minipage}{1\textwidth}
		K3 surfaces \hfill Daniel González Casanova Azuela
		
		{\small Prof. Misha Verbitsky\hfill\href{https://github.com/danimalabares/k3}{github.com/danimalabares/k3}}
	\end{minipage}
\end{minipage}\vspace{.2cm}\hrule

\vspace{1em}
{\Huge Lecture notes on K3 surfaces}

\tableofcontents

\section{Class 1}

The most important invariant of a k3 surface is \href{https://en.wikipedia.org/wiki/Intersection_form_of_a_4-manifold}{intersection form}.

There are three classes of manifolds
\begin{enumerate}
	\item Smooth manifolds
\end{enumerate}

\[\begin{tikzcd}
	\text{smooth manifolds}\arrow[r,"\text{forgetful functor} "]&\text{PL manifold}\arrow[r]&\text{Topological manifolds}   
\end{tikzcd}\]

Donaldson: contiunally many non-equivalent smooth structures on $\mathbb{R}^{4}$. K3 surfaces has countably many smooth structures and only one of them is compatible with complex structure.

\begin{defn}
	Intersection form. Given a quadratic form on a lattice $V_{\mathbb{Z}}=\mathbb{Z}^n$, so 
	 \[q:V_{\mathbb{Z}}\times V_{\mathbb{Z}}\to \mathbb{Z}\]
	 is \textit{\textbf{unimodular}} if 
	 \[V_{\mathbb{Z}}\overset{q}{\longrightarrow} \operatorname{Hom}(V_{\mathbb{Z}},\mathbb{Z})\]
	 is an isomorphism.
\end{defn}

\begin{thm}[Universal coefficients formula]\leavevmode
	\[H_{n-1}(M,\mathbb{Z})=\mathbb{Z}^{b_{n-1}(M)}\oplus T_{n-1}(M)\]
	\[h^n(M,\mathbb{Z})=\mathbb{Z}^{b_{n}(M)}\oplus T_{n-1}(M)\]
\end{thm}

\begin{coro}
	$H^2(X,\mathbb{Z})$ is torsion free if $\pi_{1}(X) =0$ because 
\end{coro}

\begin{defn}
	\textit{\textbf{Signature}} is $m-n$ if $q$ has signature $(m,n)$.
\end{defn}

\begin{thm}[Rokhlm-Wu?]\leavevmode
	Signature is divisible by 16 for simply-connected (something else).
\end{thm}

\begin{remark}
	The methods used in surgery break down in smooth case because strange topological objects like infinite sums of spheres arise.
\end{remark}

\begin{thm}[Freedman, 1982]
	There are as many 4-manifolds as there are intersection forms. $M$ simply connected 4 manifold homotopy class is uniquely determined by intersection dorm. Moreover, for every unimodular form there exists a unique $M$ with this intersection form.
\end{thm}

\begin{thm}[Donaldson, 1986]\leavevmode
	$M$ smooth compact manifold with positive definite odd intersection form $q$. Then
	\[\begin{pmatrix} 1&0&0\\0&1&0\\0&0&0&1 \end{pmatrix} \]
\end{thm}

\begin{defn}
	Bilinear symmetric form is \textit{\textbf{indefinite}} if it is not positive definite nor negative definite.
\end{defn}

\begin{thm}[Classification of unimodular symmetric bilinear forms]\leavevmode
	Odd are diagonalizable, while even are related to special Lie group $E_8$.
\end{thm}

\begin{defn}
	A \textit{\textbf{K3 surface}} is a K\"ahler complex surface $M$ with $b_1=0$ (simply connected) and $c_1(M,\mathbb{Z})=0$.
\end{defn}

Kodaira did what Andr\'e Weil couldn'g classify.

 \begin{thm}
	K3 surfaces have trivial canonical bundle $K_{M}=\Lambda^2(\Omega^1M)$.
\end{thm}

\section{Class 2}

$G$ topological group. \textit{\textbf{Principal $G$ bundle}} is a space with free $G$-action such that the quotient $E/G$ is Housdorff. There are several conditions that make this work. And then you have $\operatorname{Homo t o py}(X,BG)=$equivalence classes of $G$-bundles.
Vector bundles of a manifold are the same as maps from $X$ to $B\operatorname{U}(n)$.

Vector bundles up to stable equivalence are classified basically by Chern classes, so by the cohomology in $H^\bullet(B\operatorname{U})=Q[c_1,c_2,\ldots,c_n$.

Now look at the loop space of $X$. Then $H^{\bullet}(\Omega X)$ is a free graded commutative algebra. Loop space has the interesting property that $\Omega \operatorname{U}=B\operatorname{U}$ and $\Omega B\operatorname{U}=\operatorname{U}$.

\subsection{Bialgebras}

Let $A$ be a superalgebra (graded with antisymmetric product). Then we ask the axiom of coassociativity and that .

\begin{example}
	$G$ group, and $C(G)$ the ring of  $k$-valued functions $C(G\times G)=C(G)\times C(G)$ so
	\begin{align*}
		G\times G &\longrightarrow G \\
		C(G) &\longmapsto C(G)\otimes C(G)
	\end{align*}
	
\end{example}

\subsection{H-spaces}

\begin{defn}
	$H$-space is a space $M$ with a map $\mu:M\times M\\to M$that is homotopy associative,
	\[\begin{tikzcd}
		M\times M\times M\arrow[r,"\mu\times \operatorname{id}"]\arrow[d,"\operatorname{id}\times \mu"]&M\times M\arrow[d,"\mu"]\\
		M\times M\arrow[r,"\mu"]&M
	\end{tikzcd}\]
	which is homotopy commutative. And with homotopy unit.
\end{defn}

So it's like a homotopy algebra?

\begin{example}
	The loop space.
\end{example}

\subsection{Bialgebras of finite type}

\begin{defn}
	A bialgebra $A$ is of \textit{\textbf{finite type}} if it is the direct sum of $A=\bigoplus_{i\geq 0} A^i $ supercommutative and each $A^1$ is finite dimensional.
\end{defn}

\begin{remark}
	Free commutative algebra is polynomial algebra
\end{remark}

\begin{defn}
	$A=\mathbb{C}[x_1,\ldots,x_n,\ldots]\otimes \Lambda^\bullet(a_1,\ldots,a_n,\ldots)$ is a graded commutative free algebra. In the slides: it is $\operatorname{Sym}_{\operatorname{gr}}V^*$ where $V^*$ is a graded vector space.
\end{defn}

\begin{thm}[Hopf]\leavevmode
	A graded commutative bialgebra of finite type over $k$ of 0 characteristic is free graded commutative as a $k$ algebra.
\end{thm}

\subsection{The cohomology algebra of U(n)}

\begin{claim}
	The cohomology algebra $H^{*}(\operatorname{U}(n),\mathbb{Q})$ is a free graded commutative algebra with generators in degrees $1,3,5,\ldots,2n-1$.
\end{claim}

\begin{proof}[Demostra\c c\~ao]
	Induction.  $\operatorname{U}(1)$ is clear because it is a circle. Then do Serre spectral sequence. Differentials vanish on the second page because there's only nonzero groups on even degrees! And we get that $E_2^{p_1}=H^{p}(S^{2n-1})\otimes H^{q}(\operatorname{U}(n-1))$. And then the sequence converges to that of the total space which is $\operatorname{U}(n)$.
\end{proof}

\subsection{Grassman manifolds}

\begin{defn}
	The \textit{\textbf{fundamental bundle}} $B_{\operatorname{fun}}$ is a rank $n$ vector bundle over $\operatorname{Gr}(n,m)$.
\end{defn}

\begin{claim}
	$B$, $B'$ vector bundles of rank $n$, $m-n$, $B\oplus B'$
	\[\varphi:X\to \operatorname{Gr}(m,n)\]
	\[\varphi (x)=B_x\subset B_x\oplus B_x'=\mathbb{K}^m\]
	then $B=\varphi^* B_{\operatorname{fun}}$.
\end{claim}

\begin{thm}\leavevmode
	If you have $B$ as a bundle on a manifold $X$ then $B\oplus B'$ is trivial for some bundle $B'$.
\end{thm}

\begin{proof}[Demostra\c c\~ao]
	Embed the total space in a large enough euclidean space.
\end{proof}

\begin{defn}
	$\operatorname{Gr}(n,\infty)=\operatorname{Gr}(n)$ is $\bigcup_{m=n_1}^\infty\operatorname{Gr}(n,m)=\operatorname{Gr}(n) $
\end{defn}

\begin{coro}
	For every bundle  $B$ of rank $n$ there is a function $\varphi:X\to \operatorname{Gr}(n)$ such that $B=\varphi^*B_{\operatorname{fun}}$.
\end{coro}

Take a bundle $E\to X$ and $G$ acts freely on $E$ so $E$ principal $G$ bundle. Classifying space  $BG$

\begin{thm}[Atiyah-Bott]\leavevmode
	Classifying space is unique up to homotopy equivalence.
\end{thm}

\subsubsection{The fundamental bundle}

In class 4 I finally understood that

\begin{defn}
	The \textit{\textbf{fundamental bundle}} on the Grassmanian  $\operatorname{Gr}(n)$ (the Grassmanian is this space where points are linear spaces) is the vector bundle such that the fiber of one point (which is a vector space) is the vector space that is the point. It's very tautological.
\end{defn}

\begin{thm}[Did we prove this?]\leavevmode
	Let $B$ be a vector bundle of rank $n$ on a cellular space $X$. Then there exists a continuous map $\varphi:X\to \operatorname{Gr}(n)$ such that $ B$ is isomorphic to the pullback $\varphi^*B_{\operatorname{fun}}$ of the fundamental bundle.
\end{thm}

\begin{remark}
	In fact $\operatorname{Gr}(n)$ is the classifying space of vector bundles of rank $n$, in the sense that isomorphism classes of vector bundles of maps $\varphi:X\to \operatorname{Gr}(n)$.
\end{remark}

\subsubsection{The canonical bundle}

When doing homework 3 I found this very nice on Hatcher, Vector bundles and K3:

\begin{defn}
	The \textit{\textbf{canonical bundle}} $p:E\to \mathbb{R}P^{n}$ has as its total space $E$ the subspace of $\mathbb{R}P^{n}\times \mathbb{R}^{n+1}$ consisting of pairs $(\ell,v$ with $v\in\ell$ and $p(\ell,v)=\ell$. There is also an infinite-dimensional projective space $\mathbb{R}P^{\infty}$ which is the union of the finite-dimensional projective spaces $\mathbb{R}P^{n}$ under the inclusions $\mathbb{R}P^{n}\hookrightarrow \mathbb{R}P^{n+1}$ coming from natural inclusions $\mathbb{R}^{n+1}\hookrightarrow \mathbb{R}^{n+2}$. The inclusions $\mathbb{R}P^{n}\hookrightarrow \mathbb{R}P^{n+1}$ induce corresponding inclusions of canonical line bundles, and the union of all these is a canonical line bundle over $\mathbb{R}P^{\infty}$.

	A natural generalization is the Grassmanian $\mathcal{Gr}(k,n)$ along with a canonical $k$-dimensional vector bundle over it consisting of pairs $(\ell,v)$ where $\ell$ is a poin the Grassmanian and  $v$ is a vector in $\ell$.
\end{defn}

\subsubsection{What this classification space should mean}

Remember that
\begin{defn}[Representable functor]
	Let $\mathcal{C}$ be a category. A functor $F:\mathcal{C}^{\operatorname{op}}\to \operatorname{Sets}$ is called \textit{\textbf{representable}} if there exists an object $B=B_{F}$ in $\mathcal{C}$ with the property that there is a \textit{natural} isomorphism of functors
	\[\varphi:\mathcal{C}(-,B_{F})\to F\]
	where $\mathcal{C}(-,B_{F})$ is the set of arrows from $-$ to $B_{F}$.

	One ussually expresses the naturality condition for a map $f:X\to Y$ with the following diagram:

	\[\begin{tikzcd}
		\mathcal{C}(X,B)\arrow[r,"\varphi_{X}"]\arrow[d,"f^{*}"]&F(X)\arrow[d,"f^{*}"]\\
		\mathcal{C}(Y,B)\arrow[r,"\varphi_{Y}"]&F(Y)
	\end{tikzcd}\]
\end{defn}

And in homotopy theory I have studied that

\begin{thm}[Brown representability theorem]
	Let $F$ be a contravariant functor from the homotopy category of parallel connected CW-complexes to pointed sets. If $F$ satisfies conditions (i) and (ii) above (for any pointed connected CW-complexes $X_{i}$, $A$, $B$, $C$), then $F$ is representable.
\end{thm}

\begin{remark}[So what is a classifying space?]
	The theorem says that there is a space $B=B_{F}$ (itself a pointed CW-complex) for which there is a natural isomorphism
	\[\varphi :[X,B_{F}]_{*}\overset{\cong }{\longrightarrow}F(X)\]
	for any pointed CW-complex $X$. This space $B_{F}$ is called a \textit{\textbf{classifying space}} for $F$. Recall also that when such $\varphi$ exists, it is completely determined by a \textit{generic} element $\gamma\in F(B_{F})$.

	The classifying space together with the genereic element is unique up to homotopy.
\end{remark}

\begin{remark}
	$H^{n}(-,G)$ is represented by  $K(G,n)$ together with a chosen element in $H^{n}(K(G,n),G)$
\end{remark}

But anyway. We see in \href{https://en.wikipedia.org/wiki/Classifying_space_for_U(n)#Case_of_line_bundles}{wiki} that for the case of homework 3 bundle  $S^1\to S^\infty\to \mathbb{C}P^\infty$ we get that the base space $BU(1)=\mathbb{C}P^\infty$, Thus, the set of isomorphism classes of circle bundles over a manifold $M$ are in one-to-one correspondence with the homotopy classes of maps from $M$ to $\mathbb{C}P^\infty$.

So what is the functor that we are representing? I think is K. Because the maps are isomorphic to  $K(S^1)$…? Circle bundles?

\subsection{Stiefel spaces}

\begin{defn}
	$\mathbb{K}^\infty$ is the direct limit of $\mathbb{K}^n$ so its just the direct sum $\bigoplus_{i=n}^\infty\mathbb{K} $. Stiefel space is the space of orthonormal $n$-frames.
\end{defn}

If we prove that Stiefel is contractible we obtain our classifying space so let's prove that. We have a fibration

\[\operatorname{U}(n)\hookrightarrow \operatorname{St}(n,\infty)\to \operatorname{Gr}(n,\infty)\]

\begin{thm}
	$\operatorname{St}(n)$ is contractible.
\end{thm}

\begin{proof}[Demostra\c c\~ao]
	\begin{enumerate}[label=\textbf{Step \arabic*}]
		\item Locally trivial fibration with contractible fiber and base $Y\to X$ then $Y$ is contractible, this is so trivial.

		\item Fibration $ \operatorname{St}(n)\to \operatorname{St}(n-1)$ with fiber $S^\infty$ 

		\item Show that $S^\infty$ is contractible.

		\item And then some map $\mathbb{R}$ that is not surjective, and construct homotopy of identity to a constant map.
	\end{enumerate}
\end{proof}

\begin{exercise}
	If $X_{\infty}=\bigcup X_{i} $ is the inductive limit of contractible cellular spaces then it is contractible. Use Whitehead theorem.
\end{exercise}

\begin{thm}[Important]\leavevmode
	$\operatorname{Gr}(\infty)=B\operatorname{U}$.
\end{thm}

\subsection{Stable equivalence}

\begin{defn}
	Vector bundles $V$,  $W$ are stable equivalent if  $V\oplus A\cong W\cong B$ for trivial vector bundles  $A$ and $B$.
\end{defn}

Homotopy classes of equivalent vector bundles are in coorespondance with…

\begin{thm}
	$B\operatorname{U}$ is $H$-space.
\end{thm}

\begin{coro}
	$H^* (B\operatorname{U},\mathbb{Q})$ is a free supercommutative algebra.
\end{coro}

\begin{claim}
	$H^*(B\operatorname{U}$ is a free polynomial algebra generated by classes $c_1,c_2,\ldots$ in all even degrees.
\end{claim}

\section{Class 3}

\subsection{Reminder}

\begin{defn}
	\textit{\textbf{Bialgebra}} is an algebra that is  equipped with comultiplication, counit…
\end{defn}

\begin{remark}
	It is when the dual space also has an algebra structure, but we prefer to use the tensor notation.
\end{remark}

Let $\sum_{i \geq 0}A^i$ with $\dim A^i<\infty$. \textit{\textbf{Free commutative algebra}} is a polynomial algebra.  \textit{\textbf{Free graded commutative algebra}} is
\[\widetilde{\operatorname{Sym}}^\bullet(W^\bullet \oplus V^\bullet):=\operatorname{Sym}^\bullet(W^\bullet)\otimes \Lambda^\bullet(V^\bullet)\]
where
\[W=\bigoplus_{i} W^{\operatorname{even}} \qquad V=\bigoplus_{i}V^{\operatorname{odd}}. \]

\subsection{Hopf algebra}

\begin{defn}
	A bialgebra is a \textit{\textbf{Hopf algebra}} when it is also equipped with an antipode map ($S$) such that the following diagram commutes
\[\begin{tikzcd}
	&H\otimes H\arrow[rr,"S\otimes \operatorname{id}"]&&H\otimes H\arrow[rd,"m"]\\
	H\arrow[ur,"\Delta"]\arrow[dr,swap,"\Delta"]\arrow[rr,"\eta"]&&\mathbb{C}\arrow[rr,"u"]&&H\\
								     &H\otimes H\arrow[rr,"\operatorname{id}\otimes S"]&&H\otimes H\arrow[ur]
\end{tikzcd}\]
[diagram from quantum group minicourse notes]

	\end{defn}

\begin{example}
	The cohomology of the loop space, $H^{\bullet}(\Omega X)$.
\end{example}

\subsection{Primitive elements in a bialgebra}

\begin{defn}
	An element of a bialgebra $x\in A$ is \textit{\textbf{primitive}} if $\Delta(x)=x\otimes 1+1\otimes x$.
\end{defn}

\begin{align*}
	\Delta(xy)&=\Delta(x)\Delta(y)\\
	&=(1\otimes x+x\cdot 1)(y\otimes 1+y\otimes y)\\
	&=1\otimes xy+xy\otimes 1+x\otimes y+y\otimes x.
\end{align*}

\begin{remark}
	We trying to show that Hopf algebras? bialgebras? are generated by primitive elements?
\end{remark}

\begin{defn}
	$A^\bullet$ bialgebra, $\mathcal{P}^\bullet\subset A^\bullet$ space of primitive, and the natural embedding
	\[\operatorname{Sym}_{gr}(\mathcal{P}^\bullet)\to A\]
	We say that $A$ is \textit{\textbf{free up to defree $k$}} if
	\[\bigoplus_{i\leq k} \operatorname{Sym}_{\operatorname{gr}}^i(P)\overset{\psi}{\longrightarrow} A\]
	is an embedding.
\end{defn}

\begin{lemma}
	Let $A^\bullet$ be a bialgebra which is free up to degree $k$. Then $A^\bullet$ is free up to degree $k+1$.
\end{lemma}

\begin{proof}\leavevmode 
	\begin{enumerate}[label=\textbf{Step \arabic*}]
		\item Choose a basis of $P$, $\{x_i\}$. Chose a polynomial condition $Q(x_1,\ldots,x_n)=0$ of degree $k+1$. Write this as
			\[Q=Q_mx_1^m+Q_{m-1}x_1^{m-1}+\ldots+Q_0.\]
			that is
			\[Q=\sum_{i=0}^mQ_ix_1^i\]
			with $Q_i$ invariant somehow. Then we apply comutiplication to obtain
			\[\Delta(Q)=Q\otimes 1+1\otimes Q+R\]
			where $R$ is some sort of reminder with bounded degree:
			\[R\in\mathfrak{U}:=\bigoplus_{i\leq k} \operatorname{Sym}^i_{\operatorname{gr}}(P)\otimes \bigoplus_{i\leq k} \operatorname{Sym}_{\operatorname{gr}}^i(P)  \]
			which follows from a similar computation of that which we did after defining primitive elements.

			\item Project to drop the terms that have $Q\otimes 1+1\otimes Q$:
				\[\Pi:\mathfrak{U}\to x_1\otimes \bigoplus_{i\leq k}   \operatorname{Sym}^i_{\operatorname{gr}}(P)\]
				since the $x_i$ are primitive, i.e. $\Delta(x_i)=x_i\otimes 1+1\otimes x_i$, one has
				\[\Delta(x_1^m)=(x_1\otimes 1+1 \otimes x_1)^m\]
				we get that
				\[\Pi(\Delta(x_1^m))=mx_1\otimes x_1^{m-1}\]
				while on the board it is written that
				\[\Pi(\Delta(x_1^m))=\Pi((x_1\otimes 1+1\otimes x_1)^m)\]

			\item Let $\Pi(R):=x_1\otimes R_0$. Since $Q=0$ in $A$, its component $R_0$ is also equal to 0. So $\Pi(\Delta(Q))=0$. Then
				\begin{align*}
					0&=\Pi \left( \Delta \left( \sum_{m}x_1^m\cdot Q_m \right)  \right)\\
					 & =\sum_{m}x_1\otimes x_1^{m-1}Q_m+\Pi(mx_1\otimes x_1^{m-1}\cdot  \Delta(Q_m))\\
					& =\sum_{m}x_1\otimes x_1^{m-1}Q_m
				\end{align*}
				so that 
				\begin{align*}
					x_1\otimes x_1^{m-1}Q_m&=0\\
					\implies x_1^{m-1}Q_m&=0
				\end{align*}
				So we conclude that
				\[Q_m=0\]
		\end{enumerate}
	\end{proof}

\begin{remark}
	We just proved that for any subalgebra generated by finite elements, we didn't use that it is free.
\end{remark}

\subsection{Algebras with filtration}

\begin{defn}
	A \textit{\textbf{filtration on algebra}} is
	\[A^\bullet  \supset  F_1A^\bullet\supset F_2A^\bullet\supset\ldots\]
	such that
	 \[F_i A^\bullet F_j\subset F_{i+j}A^\bullet\]
\end{defn}

\begin{defn}
	\textit{\textbf{Associated graded}} to a filtered algebra is
\[A^\bullet_{\operatorname{gr}}=\bigoplus_{i=0}^\infty \dfrac{F^1A^\bullet}{F^{i+1}A^\bullet} \]
\[F^0A^\bullet=A^\bullet\]
\end{defn}

\begin{defn}
	$I\subset A$ ideal then \textit{\textbf{$I$-adic filtration}} is the filtration by the degrees of the ideal
	\[A\supset I\supset  I^2 \supset  I^3\ldots\]
\end{defn}

\begin{lemma}
	Choose an $I$-adic filtration. Then $A_{\operatorname{gr}}$ is generated by its first and second graded components $A/I\oplus I/I^2$.
\end{lemma}

\begin{proof}[Demostra\c c\~ao]
	Indeed, $I^k/^{k+1}$ is generated by products of $k$ elements in $(I/I^2)$.
\end{proof}

\begin{defn}
	A \textit{\textbf{augmentation ideal}} in a bialgebra is the kernel of the counit homomorphism $\varepsilon:A\to k$. We denote it by $Z=\ker A$
\end{defn}

\begin{remark}
\[\Delta(x)=1\otimes x+x\otimes 1 \operatorname{mod}Z\otimes Z\]
Why? Because
\begin{align*}
	x&=\varepsilon \otimes \operatorname{id}(\Delta(x))\qquad \text{up to $Z\otimes A$}\\
	\Delta(x)& =1\otimes x\qquad \text{up to $A\otimes X$} \\
	\Delta(x)&=x\otimes 1
\end{align*}
\end{remark}

Ok, now we can prove Hopf theorem.

\begin{thm}[Hopf theorem]\leavevmode
	A finite type bialgebra is generated by primitive elements.

	In slides: Let $A$ be a graded bialgebra of finite type over a field $k$ of characteristic 0. Then $A$ is a free graded commutative $k$-algebra.
\end{thm}

\begin{proof}\leavevmode 
	\begin{enumerate}[label=\textbf{Step \arabic*}]
		\item I think this is the computation above.

		\item $A_{\operatorname{gr}}$ is a bialgebra.

		\item $A_{\operatorname{gr}}$ is multiplicative generated by $Z^1/Z^2$. All elements $Z^1/Z_2$ are primitive, so this algrebra $A_{\operatorname{gr}}$ is generated by primitive elements.

		\item Let $\{x_i\}$ be a basis of primitive elements of $A_{\operatorname{gr}}$. Then lifts of $A$ have no relations because $A_{\operatorname{gr}}$ is already generated by primitive elements. Then there are no relations also for elements in $A^\bullet$ (I think).
	\end{enumerate}
\end{proof}

\subsection{Grassmanians (Reminder)}

$B$ vector bundle of rank $n$ on $X$ then there exists a map (essentialy unique) $\varphi:X\to \operatorname{Gr}(n)$ such that
\[\varphi^* (B_{\operatorname{fun}}=B\]
which makes the Grassmanian a classifying space, and $\operatorname{Gr}(1)=B\operatorname{U}(n)$.

The infinite Grassmanian is important.

\subsection{BU as an H-space (Reminder)}

Bott periodicity identifies the space of loops on $\operatorname{U}$ and $B\operatorname{U}$.

\begin{prop}
	Embed $\mathbb{C}^{\infty}\times \mathbb{C}^\infty$ into $\mathbb{C}^\infty$ taking the basis vectors of the first copy to the even basis vectors and the basis of the second copy to the odd. Then for $L_1\subset \mathbb{C}^\infty$, $L_2\subset \mathbb{C}^\infty$, the map
	\[L,L'\mapsto S(L,L')\]
	defines a structure of $H$-space on the infinite Grassmanian $B\operatorname{U}$.
\end{prop}

\begin{proof}
	Just show that $H$-associatity up to homotopy.
\end{proof}

\begin{coro}
	$H^\bullet(B\operatorname{U},\mathbb{Q})$ is a free supercommutative algebra.
\end{coro}
\begin{proof}
	Follows from Hopf theorem.
\end{proof}

\subsection{Cohomology of BU}

\begin{claim}
	$H^\bullet(B\operatorname{U},\mathbb{Q})$ is a free polynomial algebra generated by classes $c_1,c_2,\ldots$ in all even degrees.
\end{claim}

\begin{proof}[Demostra\c c\~ao]
	Leray-Serre spectral sequence.
\end{proof}

\subsection{Chern classes: axiomatic definition}

\begin{defn}
	\textit{\textbf{Chern classes}} are classes  $c_i(B)\in H^{2i}(X)$ for $i=0,1,2,\ldots$

	\textit{\textbf{Chern classes}} are  $c_i(B)\in H^{2i}(X)$ for a complex vector bundle $B$ over $X$ with axioms

	\begin{enumerate}[label=\alph*.]
		\item $c_0(B)=1$

		\item Functoriality (commutes with bullbacks): for $\varphi:X\to Y$ with $B$ bundle on $Y$, 
			 \[\varphi^*(c_i(B))=c_1(\varphi^*(B))\]

		\item Define \textit{\textbf{total Chern class}}  $c_{*}:=\sum_{i}c_i(B)$ then
			\[c_i(B)\cdot c_i(B')=c_*(B\oplus B')\qquad \text{(Whitney)} \]

		\item $\mathcal{O}(1)$ on $\mathbb{C}P^{n}$,
			\[c_i(\mathcal{O}(1)=1+[H]\]
			where $[H]$ is the fundamental class of a hyperplane section.
	\end{enumerate}
\end{defn}

\begin{remark}[Once and for all]
	$\mathcal{O}(-1)$ is the \textit{\textbf{tautological line bundle}} on a Grassmanian, defined as  $\{(\ell,v)\in\mathcal{Gr}(k,n)\times \mathbb{C}^n:v\in\ell\}$.

	$\mathcal{O}(1)$ is the \textit{\textbf{hyperplane bundle}} which is the dual of that so $\{(\ell,v^* )\in\mathcal{Gr}(k,n)\times (\mathbb{C}^n)^*:v^*\in\ell^*\}$
\end{remark}

Suppose we have a class $a\in H^{\bullet}(B\operatorname{U})$. Then for all $B$ on $X$ 
\[\varphi:X\to B\operatorname{U}\]
so
\[B\cong \varphi^* (B_{\operatorname{fun}})\]
and so
\[\varphi^*_B(c_*)=c_*(B).\]





\section{Class 4}

\subsection{Reminder}

For each rank $n$ bundle $B$ on $X$ there exists $\varphi_B:X\to \operatorname{Gr}(n,\infty)=B\operatorname{U}(n)$ such that $\varphi^* _B(B_{\operatorname{f un}}=B$.

The infinite grassmanian is classifying space for (?) stable bundles.

Some more review about $H$-space structure, primitive elements, a comment on last exercise of homework 2.

Chern classes of $\mathcal{O}(1)$ are hyperplane sections: $c_i(\mathcal{O}(1))=1+[H]$.

\subsection{The splitting principle}

\begin{exercise}
	Prove that $BU(1)=\mathbb{C}P^\infty$.
\end{exercise}

\begin{proof}[Solution]\leavevmode
	Hopf fibration on $S^\infty$? It's easier, take $n=1$, it's just by  definition.
\end{proof}

\begin{defn}
	The \textit{\textbf{fundamental bundle}} on $BU(1)^n$ has fiber
	\[\ell_1\oplus \ell_2\oplus \ldots\ell_n\]
	where $\ell_i\in BU(1)$ are product $\ell_1\times \ell_2\times \ldots\times \ell_n$.
\end{defn}

\begin{remark}
	Chern classes of $B_{\operatorname{f un}}$ are uniquely determined by axioms, because every factor has Chern classes, and fibers are just sums, and pullbacks preserve sums…
	\[B_{\operatorname{f un}}=\bigoplus_{i} \pi_i \mathcal{O}(1) \]
	where
	\[pi_i:BU(1)^n\to BU(1)\]
	is a projection.
\end{remark}

\begin{remark}
	$H^{\bullet}(BU(1))^n=\mathbb{Z}[z_1,\ldots,zn$ 
	Here at least I remember that the cohomology of $\mathbb{C}P^{\infty}$ is just polynomials so it looks reasonable that the $n$-th power is polynomials in more cariables.
\end{remark}

\begin{thm}[Splitting principle]\leavevmode
	Let $\varphi_{\operatorname{f un}}:BU(1)^n\to BU$, the \textit{\textbf{fundamental map}}, it induces embedding on cohomology up to degree $2n$. For all primer generator $\sigma_i\in H^{2}(BU)$, $\varphi_{\operatorname{f un}}(\sigma_1)=\lambda \sum_{i}z_i^k$ with $\lambda\neq 0$.

	So
	\[\begin{tikzcd}
		BU(1)^n \arrow[r]& BU\\
		X\arrow[u]\arrow[ur]
	\end{tikzcd}\]
	
\end{thm}

\begin{remark}
	\href{https://en.wikipedia.org/wiki/Classifying_space_for_U(n)}{Wiki} Thus, the set of isomorphism classes of circle bundles over a manifold $M$ are in one-to-one correspondence with the homotopy classes of maps from $M$ to $\mathbb{C}P^{\infty}$
\end{remark}

\begin{thm}
	Chern classes are unique (uniquely determined by axioms).
\end{thm}

\begin{proof}\leavevmode 
	\begin{enumerate}[label=\textbf{Step \arabic*}]
		\item Every bundle is obtained as pullback of the fundamental bundle. So for $A\in H^{\bullet}(BU)$ and $B$ bundle on $X$, $A(B)=\varphi^*_{B}(A)\subset H^{\bullet}(X)$ so $c_i(B)$ are obtained as pullbacks of $ c$ in the fundamental bundle.

		\item \[BU(1)^\infty \overset{\varphi_{\operatorname{f un}}}{\longrightarrow}BU\]
			pullback of fundamental bundle is fundamental. (This map is defined from the former by induction).
			\[\varphi^*_{\operatorname{fu n}}(c_i(B_{\operatorname{f un}})=c_i(B_{\operatorname{f un}}\text{ on $BU$} )\]
			The Chern classes of the fundamental bundle are already known. Since $\varphi^*_{\operatorname{f un}}$ is injective by the splitting principle we are done.
	\end{enumerate}
\end{proof}

\subsection{Primitive generators of $H^{*}(BU)$}

Recall the $H$-space multiplication:
\begin{align*}
	BU\times BU  &\longrightarrow BU \\
	L_1\times L_2 &\longmapsto L_1\oplus L_2
\end{align*}
and the comultiplication
\[\Delta :H^{\bullet}(BU)\to H^{\bullet}(BU)\]
Generators of $H^{\bullet}(BU)$ are $c_{h_1},c_{h_2},\ldots$ with $c_{h_i}\in H^{2i}(BU)$ and we have the comultiplication $\Delta(c_{h_i})=c_{h_i}\otimes 1+1\otimes c_{h_i}$.

\begin{remark}
	\[\varphi=(\varphi_1,\varphi_2):X\to BU\times BU\]
	and we can compose so we have
	\[\varphi\circ \mu:X\to BU\]
	what does this map do?
	\begin{align*}
		\varphi\circ \mu: X &\longrightarrow BU \\
		\varphi^* (B_{\operatorname{fun}} &\longmapsto B_1\\
		(\varphi\circ \mu)^* (B_{\operatorname{ fu n}})&=B_1\oplus B_2
	\end{align*}
	So then we have
	\begin{align*}
		\varphi^*&:H^{\bullet}(BU)\otimes H^{\bullet}(BU)\to H^{\bullet}(X)\\
		\Delta &:H^{\bullet}(BU)\to H^{\bullet}(BU)\otimes H^{\bullet}(BU)\\
		\Delta \circ \varphi^*&:H^{\bullet}(BU)\to H^{\bullet}(X)
	\end{align*}
\end{remark}

\begin{coro}
		For every $x \in H^{\bullet}(BU)$
		\[X(B_1\oplus B_2)=\Delta(x)(B_1,B_2)\]
	\end{coro}

	\begin{coro}
		If $x\in H^{*}(BU)$ is primitive, then $x(B_1\oplus B_2)=x(B_1)\oplus X(B_2)$.
	\end{coro}

	\begin{proof}
		$\Delta(x)=x\otimes 1+1\otimes x$ so $\Delta(x)$ evaluated on $(B_1,B_2)$
	\end{proof}

	\begin{remark}
	We will construct the full Chern class $c_*(B)$ as a pullback of a class $C\in H^{*}(BU)$.
\end{remark}

\begin{remark}
Then take exponential. Let $\chi_i\in H^{2i}(BU)$ be a primitive generator. Use Hopf theorem to see that it is unique by a constant. Snce $\chi_i(B_1\oplus B_2)=\chi_i(B_1)+\chi_i(B_2)$, the class $C=e^{\sum_{i}a_i\chi_i}=1+\ldots+\frac{\chi_n}{n!}+\ldots$ satisfies the Whitney formula.

To construct Chern classes satisfying the axioms it remains to arrange the coefficients $a_i$ in such a way that $C(\mathcal{O}(1))=1+[H]$ I think this means hyperplane section.
\end{remark}

\begin{lemma}
An embedding
	\[BU(1)\overset{\varphi}{\hookrightarrow}BU\]
	with $\chi_i\in H^{2i}(BU)$ primitive generator. Then $\varphi^*(\chi_i)\neq 0$
\end{lemma}

\begin{proof}
	$H^{\bullet}(BU)=$ symmetric polynomials in $H^{i}(BU(1))^n$, $\varphi_{\operatorname{fun}}(x_N)=x\sum_{i=1}^nz_i^k$ so $\varphi (x_k)=\lambda x_1^k$.
\end{proof}

\begin{remark}
	$\varphi^* (c_i(B_{\operatorname{fun}})=c_i(\mathcal{O}(1)=1+[H]$
\end{remark}

\begin{thm}
	Choose generators $\chi_i\in H^{2}(BU)$ primitive. Then $\varphi^* (\sum_{i}\chi_i=\log(1+[H])$ where the logarith is a formal power series, namely $\sum_{i=1}^\infty\frac{H^n}{n!}(-1)^n$.

	That means $c(B_{\operatorname{fun}})=\operatorname{exp}\left(\sum_{\chi_i}\right)$.
\end{thm}
	
	
\section{Class 5}

We want to study the space of line bundles on a surface.

\subsection{K-group}

\begin{defn}
	Let $V$ be the set of equivalence classes of vector bundles on $X$. Consider the free module generated by $V$ (it's just $V$ copies of $Z$):
	\[\mathbb{Z}\left<V\right> =\bigoplus_{V} \mathbb{Z} \]
	And now consider
	\[\frac{\mathbb{Z}\left<V\right> }{[F_1]-[F_1]-[F_3]}\]
	for all exact sequences
	\[\begin{tikzcd}
		0\arrow[r]&F_1\arrow[r]&F_2\arrow[r]&F_3\arrow[r]&0
	\end{tikzcd}\]
	Equivalently, the relation is $[F_1]+[F_3]=[F_2]$.
\end{defn}

\begin{remark}
	We may give an $H$-structure to the set of homotopy classes of maps $X\to BU$ as follows
	$\varphi_1,\varphi_2:X\to B\operatorname{U}$ 
	\[B_1=\varphi^* (B_{\operatorname{fun}})\]
	define the $H$-product
	\[\varphi:=\varphi_1\circ \varphi_2\]
	such that
	\[\varphi^* (B_{\operatorname{fun}}=B_1\oplus B_2\]
\end{remark}

And then we have an isomorphism (that we are not going to use):
\[K\left(X \right) \overset{\operatorname{hom}}{\longrightarrow}\text{group of homotopy classes of maps from $X$ to $BU$} \]
This is because every bundle on $X$ is the pullback of the fundamental bundle by some map. We need to check that the image of trivial bundle is trivial map (homotopic to constant?) and that it preserves the product.

\begin{remark}
	The important thing of today is that that sum corresponds to addition
\end{remark}

\begin{remark}
	I guess I should first understand how is it that every bundle is the pullback of the fundamental bundle.
\end{remark}

So for example for injectivity we need to show that if a map $\varphi$ pulls back the fundamental bundle to the trivial bundle then $\varphi$ is homotopic to identity. This is not obvious though.

The point is that that map is a bijection.

\begin{claim}
	Chern classes are defined on $K(X)$ and satisfy Whitney formula (meaning Chern classes they pass to the quotient, right?)
\end{claim}

\begin{proof}
	Let $B$ be a bundle on $X$ so that $B=\varphi^* (B_{\operatorname{fun}}$. We showed last time that there is a $c_{\cdot}\in H^{0}(BU)$ such that $c_{\cdot}(B)=\varphi^* (c_{\cdot}$. In fact we proved that $c_{\cdot}= \operatorname{exp}(\text{additive} )$, but its actually Chern character, $c_{\cdot}=\operatorname{exp}(\operatorname{Ch}_{\cdot})$, in fact $\operatorname{Ch}_{\cdot}(B_1+B_2\operatorname{Ch}(B_1)+\operatorname{Ch}_{\cdot}(B_2)$.
\end{proof}

\subsection{Coherent sheaves}

\begin{defn}
	Let $M$ be a complex manifold and $\mathcal{O}_{M}$ its structure sheaf (of holomorphic functions). A \textit{\textbf{coherent sheaf}} is a sheaf of $\mathcal{O}_{M}$-modules, locally isomorphic to a quotient of a free sheaf $\mathcal{O}^n_{M}$ by a finitely generated  $\mathcal{O}_{M}$-invariant subsheaf.

	A \textit{\textbf{coherent sheaf}} on a projective manifold. A \textit{\textbf{projective manifold}} is  $\operatorname{Proj}(A^\bullet) $ where $A^\bullet$ is a graded ring. \textit{\textbf{Coherent sheafes}} are sheaves of graded $A^\bullet$-modules.
\end{defn}

\begin{exercise}
	Let $M$ be a projective manifold. Prove that any coherent sheaf $F$ has a (projective) resolution
	\[\begin{tikzcd}[column sep=small]
		0\arrow[r]&B_{n}\arrow[r]&B_{n-1}\arrow[r]&\cdots \arrow[r]&B_0\arrow[r]&F\arrow[r]&0
	\end{tikzcd}\]
	where $B_i$ are vector bundles. This is called the \textit{\textbf{syzygy resolution}}
\end{exercise}

\begin{proof}[Solution]\leavevmode
	Every module has a projective resolution called \textit{\textbf{Koszul resolution}}. So what is Koszul resolution. First you have a resolution of a maximal ideal. For a maximal ideal it is clear since … (Herieta? and) Eisenbud or even Bourbaki Homological algebra.
\end{proof}

\subsection{Coherent sheaves and their Chern classes}


So there's actually two K-groups. One is generated by bundles and the other by sheaves. For bundles, it is an algebra. For sheaves, it is a module over the other one. For Groethendick one was $K^\bullet$ and the other $K_{\bullet}$ but we don't know which is which.

\begin{remark}
	After this is done, it's possible to prove that the $K$-group of coherent sheaves on a projective manifold is equal to the K-group generated by holomorphic vector bundles.
\end{remark}

\begin{defn}
	The \textit{\textbf{Chern class}} of a coherent sheaf is the Chern class of the corresponding element in the K-group.
\end{defn}

\begin{remark}[about singularities, see slides]	
Suppose we do resolution of a manifold and pullback a bundle
\[\begin{tikzcd}
\tilde{M}\arrow[d,"\pi"]\\
M
\end{tikzcd}\qquad \begin{tikzcd}
\pi^*F\arrow[d]\\
F
\end{tikzcd}\]
\end{remark}

\subsection{Euler characteristic of a coherent sheaf}

\begin{defn}
	Let $F$ be a coherent sheaf. Its \textit{\textbf{Euler characteristic}} is
	\[\chi(F)=\sum_{i}(-1)^i \dim H^{i}(F)\]
	{\color{magenta}But what is that cohomology? What is the space?}
\end{defn}

\begin{claim}
	For any exact sequence
	\[\begin{tikzcd}
		0\arrow[r]&F_1\arrow[r]&F_2\arrow[r]&F_3\arrow[r]&0
	\end{tikzcd}\]
	we have
	\[\chi(F_2)=\chi(F_1)+\chi(F_3)\]
\end{claim}

\begin{proof}
	Should be possible…
\end{proof}

Then
\[\chi:K(M)\to \mathbb{Z}\]
is a homomorphism.

\subsection{Chern character}


OK so last class we defined an homomorphism called $\chi$ that was additive. Now let's call it
\[\operatorname{c}_{\cdot}=\operatorname{exp}(\operatorname{Ch}_{\cdot})\]
and it was additive
\[\operatorname{Ch}_{\cdot}(B_1\oplus B_2)=\operatorname{Ch}_{\cdot}(B_1)+\operatorname{Ch}_{\cdot}(B_2)\]

So the textbook definition is that \textit{\textbf{Chern character}} on line bundles is
 \[\operatorname{exp}(c_{\bullet}(L))\]
 So $c_1$ is additive and if you pass to the exponent it will be multiplicative:
 \begin{align*}
 	c_1(L_1\otimes L_2)&=c_1(L_1)+c_1(L_2)\\
	\operatorname{Ch}_{\cdot}(L_2\otimes L_2)&=\operatorname{Ch}_{\cdot}(L_1)\cdot \operatorname{Ch}_{\cdot}(L_2)
 \end{align*}

\subsection{Riemann-Roch-Hirzebruch theorem}

\begin{thm}[RRH]\leavevmode
	Let $F$ be a coherent sheaf on a complex compact manifold $M$. Then $\chi(F)$ can be expressed through Chern classes of $F$ and $M$ as follows:
	\[\chi(F)=\int_{X}\operatorname{Ch}_{\cdot}(F)\wedge \operatorname{Td}_{\cdot}(TM),\]
	where $\operatorname{Td}_{\cdot}(TM)$ mdenotes the \textit{\textbf{total Todd class of the tangent bundle $TM$}}, which is a sum of Chern classes.
	\[\operatorname{Td}_{\cdot}=1+\frac{c_1}{2}+\frac{c_1^2+c_1}{12}+\frac{c_1c_2}{24}+\frac{-c_1^4+4c_1^2c_2+c_1c_3+3c_2^2-c_4}{720}+\ldots\]
\end{thm}

\subsection{K-group for complex curves}

\begin{lemma}
	$K$-group for complex curves is generated by line bundles.
\end{lemma}

\begin{proof}\leavevmode 

	\begin{enumerate}[label=\textbf{Step \arabic*}]
		\item For each $F$ coherent sheaf, $L^n\otimes F$ has a section. So there is a monomorphism $L^{-N}\hookrightarrow F$.

		\item The consider the localization to produce a short exact sequence
			\[\begin{tikzcd}
				0\arrow[r]&F_1\arrow[r]&F_2\arrow[r]&F_3\arrow[r]&0
			\end{tikzcd}\]
			since $F=\bigoplus_{i} F_i $ for $F_i=\mathcal{O}_{M}/\mathfrak{m}_{X}^{a_i}$ so
			\[\begin{tikzcd}
				0\arrow[r]&(\mathfrak{m}_{X}^{a_1}\arrow[r]&\mathcal{O}_{X}\arrow[r]&F_1\arrow[r]&0
			\end{tikzcd}\]
	\end{enumerate}
\end{proof}

\subsection{Riemann-Roch for complex curves}

\begin{thm}[Riemann-Roch for complex curves]\leavevmode
	Let $F$ be a coherent sheaf on a compact complex curve of genus $g$. Then
	\[\chi(F)=c_1(F)+\operatorname{rk}(F)(1-g)\]
\end{thm}

\begin{proof}\leavevmode 
 We want to see
			\[c_1(L)=\operatorname{deg}(L)\]

	\begin{enumerate}[label=\textbf{Step \arabic*}]
		\item It suffices to prove for line bundles by the lemma.

		\item For degree 0 its easy beacuse $c_1(k_{x})=1$. For structure sheaf $\mathcal{O}_X$ we have rank is 1.

		\item Now let $ L$ be a line bundle. We have
			\[\begin{tikzcd}
				0\arrow[r]&\mathcal{O}_M\arrow[r]&L\arrow[r]&F\arrow[r]&0
			\end{tikzcd}\]
			\[\begin{tikzcd}
				0\arrow[r]&L\arrow[r]&\mathcal{O}_M\arrow[r]&F\otimes L=F\arrow[r]&0
			\end{tikzcd}\]
			\[\begin{tikzcd}
				0\arrow[r]&\mathcal{O}_{M}\arrow[r]&L_1^N\otimes L\arrow[r]&F\arrow[r]&0
			\end{tikzcd}\]
			\[\begin{tikzcd}
				0\arrow[r]&L_1^{-N}\arrow[r]&L\arrow[r]&F\arrow[r]&0
			\end{tikzcd}\]
		and the point is that many things "have sections". What does it mean to have sections.
	\end{enumerate}
\end{proof}

\subsection{Riemann-Roch-Hirzebruch for line bundles on complex surfaces}

\begin{defn}
	A \textit{\textbf{complex surface}} is a compact complex manifold of dimension 2.
\end{defn}

\paragraph{Notation}
\[(L_1,L_2)=c_1(L_1)\wedge c_1(L_2)\]
and if $D$ is a divisor we write (the \textit{\textbf{degree of a divisor}})
\[(D,L)=\operatorname{deg}_DL=\int_{M}[D]\wedge c_1(L)\]

\begin{thm}[RRH for surfaces]\leavevmode
	$L$ line bundle on surface and $K_{X}=\Omega^{2}(X)$ its canonical bundle. Then
	\begin{equation*}\label{eq:rrh}
\chi(L)=\chi(\mathcal{O}_{X})+\frac{(L-K_X,L)}{2}
	\end{equation*}
	where $(A,B)$ denotes the intersection form applied to cohomology classes on $X$.
\end{thm}

\begin{proof}\leavevmode 
	
	\begin{enumerate}[label=\textbf{Step \arabic*}]
		\item Let $D$ a smooth curve of genus $g$ and $L_1,L_2$ line bundles that fit in an exact sequence
			\[\begin{tikzcd}
				0\arrow[r]&L_2\arrow[r]&L_2\arrow[r]&L_2|_{D}\arrow[r]&0
			\end{tikzcd}\]
Then we use Rieman-Roch for curves gives
\[\chi(L_1)=\chi(L_2)+(L_2,D)+(1-g)\]

\item Let $N D$ denote the normal bundle on $D$. The adjunction formula gives $K_D=K_X|_{D}\otimes KD$. Since $g-1=\operatorname{deg}K_D/2$, we obtain $1-g=-(K_X+D,D)/2$.

\item The next step goes as before, with  Rieman-Roch in one dimension. Let $\chi'(L)$ be the RHS of \cref{eq:rrh}, namely  $\chi'(L)=\chi(\mathcal{O}_{X})+\frac{L-K_X,L)}{2}$. In step 1 we have $c_1(L_2)=c_1(L_1)+D$. Then
	\begin{align*}
		\chi'(L_2)-\chi'(L_1)&=\frac{1}{2}\left[ (L_2-K_X,L_2)-(L_2-K_X-D,L_2-D) \right] \\
		&=(L_2,D)-(K_X+D,D)/2
	\end{align*}

	\item Comparing Step 2 and Step 3, we get
		\[\chi'(L_2)-\chi'(L_1)=\chi(L_2)-\chi(L_1)\]
		Therefore, \cref{eq:rrh} is equivalent for $L_2$ and for  $L_1$. We just need to manipulate bundles to reduce a bundle to… by building exact sequences.

	\item So suppose you have a smooth section of a bundle. Take an ample bundle $A$ and do
		\[\begin{tikzcd}
			0\arrow[r]&\mathcal{O}_X\arrow[r]&A^N\arrow[r]&A^N|_{D}\arrow[r]&0
		\end{tikzcd}\]
		\[\begin{tikzcd}
			0\arrow[r]&L\arrow[r]&A^N\otimes L\arrow[r]&A^N\otimes L|_{D}\arrow[r]&0
		\end{tikzcd}\]
	and then by step 4 we just need to deal with $A^N\otimes L$.

	\item It's very ample, it has many sections, including some that are smooth. Now we just assume $L$ is $A^N\otimes L$. So
		\[\begin{tikzcd}
			0\arrow[r]&\mathcal{O}_X\arrow[r]&L\arrow[r]&L|_{D}\arrow[r]&0
		\end{tikzcd}\]
		so for bundles that have smooth sections the statement is free.
	\end{enumerate}
\end{proof}

\subsection{Applying the general formula to the curve case}

We have
\[\operatorname{Ch}_{\cdot}(L)=1+c_1(L)+\frac{c_1^2(L)}{2}\]
\[\operatorname{Td}_{\cdot}(L)=1+\frac{c_1(TM)}{2}+\frac{c_1^2(M)+c_2}{12}\]
Now
\[\chi(L)-\chi(\mathcal{O})=-\frac{(K_1(L),K)}{2}+\frac{c_1(L)^2}{2}=\frac{(L,K-L)}{2}\]





\section{Class 6: Local Torelli theorem and its applications}

\subsection{Exponential exact sequence}

The exponential exact sequence is
\[\begin{tikzcd}
	0\arrow[r]&\mathbb{Z}_M\arrow[r]&\mathcal{O}_M\arrow[r]&\mathcal{O}_M^*\arrow[r]&0
\end{tikzcd}\]
and it gives a long exact sequence
\[\begin{tikzcd}[column sep=small]
	\cdots\arrow[r]&H^{1}(\mathcal{O}_M)\arrow[r]&H^{1}(\mathcal{O}^*_M)=\operatorname{Pic}\arrow[r,"c_1"]&H^{2}(M,\mathbb{Z})\arrow[r,"\alpha"]&H^{2}(\mathcal{O}_M)\arrow[r]&\arrow[r]&\cdots
\end{tikzcd}\]
$\alpha$ is just forgetful map, a projection, to the  $H^{0,2}(M)$ part of a form

The group $H^{2}(\mathcal{O}_M)$ is identified with $H^{0,2}(M)$ which is Dolbeault cohomology, hence the kernel of $\alpha$ is $H^{2}(M,\mathbb{Z})\cap H^{1,1}(M)$.

\begin{prop}
	$c_1$ holomorphic line bundle on compact K\"ahler manifold belongs to intersection $H^{2}(M,\mathbb{Z})\cap H^{1,1}(M)$ and every element of this group can be realised as $c_1(L)$.
\end{prop}

\subsection{K3 surfaces are holomorphically symplectic}

\begin{defn}
	A \textit{\textbf{complex surface}} is a compact, complex manifold of complex dimension 2.
\end{defn}

\begin{defn}
	A \textit{\textbf{K3 surface}} is a (K\"ahler, can drop this assumption) complex surface $M$ with $b_1=0$ and $c_1(M,\mathbb{Z})=0$
\end{defn}

\begin{remark}
	The hypothesis that $c_1=0$ implies that $c_1(K_M)=0$ and thus $K_M=0_M$ (it is trivial). This is beacuase  $H^{1}(\mathcal{O}_M)=0$, which follows from Hodge theory.
\end{remark}

\subsection{Hodge diamond of a K3 surface}

\[\begin{array}{c c c c c }
	&&1\\
	&0&&0\\
	1&&20&&1\\
	&0&&0\\
	&&1
\end{array}\]
since the cohomology groups
\[\begin{array}{c c c c c }
	&&\mathbb{C}\\
	&0&&0\\
	\text{sections of }K_X= H^{2,0}=\mathbb{C}&&?&&\text{Hodge (Serre?) duality}\implies  H^{0,2}=\mathbb{C}\\
	&0&&0\\
	&&H^{1,1}=\mathbb{C}
\end{array}\]
For the missing one, we comput $\chi(\mathcal{O}_M)$ using Riemann-Roch, which gives $c_2$ and from that we comput $b_2$.

\subsection{Geometric structures (the story of Teichm\"uler space)}

\begin{defn}
	\textit{\textbf{Geometric structure}} on a manifold is reduction of structure group to  $G\subset \operatorname{GL}()$.
\end{defn}

\subsection{Fr\'echet spaces}

\begin{defn}
	A \textit{\textbf{seminorm}} on a vector space $V$ is a function $\nu:V\to \mathbb{R}^{\geq 0}$ such that
	\begin{itemize}
	\item $v(\lambda x)=|\lambda| \nu(x)$ 
	\item triangle inequality.
	\end{itemize}
\end{defn}

\begin{defn}
	Define a topology using a family of seminorms generated by the open balls of all seminorms.
\end{defn}

\begin{defn}
	$V$ infinite dimensional vector space, $\nu_{\alpha}$ collection of seminorms. Sequence of vectors $z_i$ is \textit{\textbf{Cauchy}} if  $z_i$ is Cauchy for each $\nu_j$. If all Cauchy sequences converge it is called \textit{\textbf{Fr\'echet space}}.
\end{defn}

We can also define Fr\'echet space to with the distance $d(x,y)=\sum_{k=1}\frac{1}{2^k}\max(\nu_k(x-y),1)$ 

\begin{defn}
	The \textit{\textbf{topology $C^k$}} on a Riemannian manifold on the space $C^\infty_{c}(M)$ is
	\[|\varphi|_{C^k}:=\operatorname{sup}\sum_{i=0}^k|\nabla^i\varphi|\]
	where $\nabla^i$ is the iterated connection $\nabla^i:\mathcal{C}^\infty(M)\to \Lambda^{1}(M)^{\otimes i}$
\end{defn}

\begin{defn}
	Of tensor field, section of $TM^{\otimes i}\otimes T^*M^{\otimes j}$.
\end{defn}

\subsection{$C^0$ topology on group of diffeomorphisms}

\paragraph{Idea} To interpret diffeomorphisms as sections of a bundle. 

\begin{defn}
	On  $\operatorname{Dif}(M)$, riemannian manifold,
	\[d(f_1,f_2)=\operatorname{sup}_{x\in M}d(f_1(x),f_2(x))\]
\end{defn}

\subsection{$\mathcal{C}^\infty$-topology on group of diffeomorphisms}

It has more sets (is stronger) than the $C^0$ topology, 


\begin{defn}
	Fix $\mathcal{U}$ small neightbourdhoos of $\operatorname{id}$ in $\operatorname{Dif}(M)$. Choose an atlas of $U_i\subset V_i$ such that $U_i$ is relatively compact. There exists a neighbourhood of identity in $\operatorname{Dif}$ such that diffeomorphisms (sufficiently close to identity) they map $\tau (U_i)\subset V_i$. To find this neighbourhood use that closure of $U_i$ is compact in $V_i$.

	Now define the $C^\infty$ topology on $\mathcal{U}$ as $C^\infty$ convergence on maps from $U_i\subset \mathbb{R}^n$ to $V_i\subset \mathbb{R}^{n}$ using usual derivatives.
\end{defn}

Anyways, the idea is that we only need a \textit{uniform structure}  which is a partially ordered set to define Cauchy sequences.

\subsection{Teichm\"uler space of geometric structures}

Let $\mathcal{C}$ be the set of all geometric structures of a given type equipped with $C^\infty$ topology. The \textit{\textbf{Teichm\"uller space}} is  $\mathcal{C}/\operatorname{Diff}_0$, where $\operatorname{Diff}_{0}$ is the connected component of the identity. The group $\operatorname{Diff}(M)/ \operatorname{Diff}_0(M)$ is the \textit{\textbf{mapping class group}}, we are not going to use it.

\subsection{Teichm\"uler space of symplectic structures}

$\operatorname{Symp}\subset \Gamma(\Lambda^{2}(M)$. It is not Housdorff and we don't even know how much Housdorff it is. Maybe for four dimensional manifolds,…

\subsection{Moser's theorem}

\begin{thm}[Moser, 1965]\leavevmode
	The Teichm\"uler space is a manifold, and the preiod map
		\begin{align*}
		\operatorname{Per}:\operatorname{Teich}_{s}&\longrightarrow H^{2}(M,\mathbb{R})\\
		w &\longmapsto [w]
		\end{align*}
\end{thm}

It is very beautiful but semi-elementary if you know Moser's lema.

\subsection{The kernel of a differential form}

If $\Omega$ is a differential form on $M$, its \textit{\textbf{kernel}} is the space of all vectors  $X\in TM$ such that $i_X(\Omega)=0$.

 \begin{prop}
	 $[\ker \Omega,\ker \Omega]\subset \ker \Omega$.
\end{prop}

\begin{coro}
	If  $(M,I)$ almost complex and  $\Omega\in\Lambda^{2,0}(M)$ non-degenerate and closed, then $I$ is integrable.
\end{coro}

\begin{proof}
	$T^{0,1}=\ker \Omega$.
\end{proof}

\subsection{C-symplectic structures}

\begin{defn}
	$\Omega\in\Lambda^{2}(M,\mathbb{C})$, $M$ $4n$-dimensional manifold. Suppose that  $\Omega^{n+1}=0$ and $\Omega^n\wedge \overline{\Omega}^n$ is nowhere zero. Then $\ker \Omega\oplus \overline{\ker \Omega}=TM\otimes \mathbb{C}$.

	This is a \textit{\textbf{C-symplectic}} manifold.
\end{defn}

These manifolds have a nice Teichm\"uler space.

\begin{thm}[Moser-Koebe?]\leavevmode
	$(M,I_+,\Omega_+($ family of  $C$-symplectic forms, $[\Omega_t]=$constant, $H^{0,1}(M_t)=0$ then all $\Omega_t$ are related by a diffeomorphism. (This is Moser's trick!)
\end{thm}

Notice that for $n=1$ we have that the condition $\Omega^2=0$ and $\Omega\wedge \overline{\Omega}$ volume mean


\begin{thm}\leavevmode
	$\operatorname{CTeich}$ Teichm\"uler space of C-symplectic structures on K3 surface. Consider the
	\[\operatorname{Per}:\Omega\to [\Omega]\in H^{2}(M,\mathbb{C})\]
	Then the image $\operatorname{Per}(\operatorname{Teich})=\{pQ]:\int u\wedge  u=0, \int u\wedge  \bar{u}>0\}$ is a quadric.

	This is a local diffeomorphism.
	
\end{thm}

\subsection{The period space of complex structures}

Now take $\operatorname{CTeich}/\mathbb{C}^*$ because the Teichm\"uler space of complex structures has a free $\mathbb{C}^*$ action.


\begin{prop}[Local Torelli theorem for complex structures]
	Teichm\"uler space of complex strcutres on K3.
	\[\mathbb{P}\operatorname{er}=\{v\in\mathbb{P}H^{2}(M,\mathbb{C}):(v,v)=0,(v,\bar{v})>0\}\]
	so
	\[\frac{\operatorname{CTeich}}{\mathbb{C}^*}\longrightarrow \frac{Q}{\mathbb{C}^*}\subset \mathbb{P}H^{2}(M,\mathbb{C})\]
\end{prop}

\subsection{The period space of complex structures is a Grassmanian}

Lets define
\[\operatorname{Gr}_{++}(H^{2}(M,\mathbb{R}))=\text{positively oriented 2-planes in }H^{2}(M,\mathbb{R}) \]
Where positively oriented means the form is ? Then
\begin{align*}
\mathbb{P}\operatorname{er}&=\operatorname{Gr}_{++}(H^{2})=\dfrac{\operatorname{SO}(3,1\mathcal{o})}{\operatorname{SO}(2) \times \operatorname{SO}(1,1\mathcal{o})}
\end{align*}

\section{Class 7: smooth quartics}

\subsection{Reminder on local Torelli theorem}

\[\operatorname{CTeich}=\text{Teichmüller space of hol sympl structures} \]
\begin{align*}
	\operatorname{CTeich} &\longrightarrow H^{2}(M,\mathbb{C}) \\
	\Omega &\longmapsto [\Omega ]
\end{align*}

Then this map is a local differomorphism to the period space.

\subsection{Hodge index theorem (without slides)}

\begin{thm}[Hodge index theorem]\leavevmode
	Consider a signature of intersection form on complex K\"ahlerler surface is positive on real part of $\operatorname{Re}H^{2,0}(M)$, $(1,0)$ on  $H^{1,1}(M,\mathbb{R})$, negative on 
	\begin{align*}
		\ker L:H^{1,1}(M,\mathbb{R})  &\longrightarrow H^{4}(M) =\mathbb{R}\\
		X &\longmapsto [X\wedge \omega ]
	\end{align*}
	with $\omega$ Kähler form.
	
\end{thm}

\begin{proof}
	$\Omega^{2,0}$ 1 dimensional, $\operatorname{Re}\Omega^{2,0}$ 2-dimensional (at most) in $\Lambda^2(M,\mathbb{R})$.
	\begin{align*}
		\Omega&=\omega_1+r-1\omega_2\\
		\Omega \wedge  \overline{\Omega}&=\omega_1^2+\omega_2^2>0\\
		\Omega \wedge \Omega &=0=\omega_1^2=\omega^2_2\\
		\iff \omega_1\wedge \omega_2&=0\\
		\omega_1^2&=\omega_2^2\\
		\implies \omega_1\perp \omega_2\\
		\omega_1^2&=\omega_2^2>0
	\end{align*}

	Then 
	\[\Lambda^{1,1}=\ker L\oplus \omega\]
	That is a 4-dimensional bundle that is given by multiples of the K\"ahler form plus the primitive part. Then
	\[\ker^\perp=\left<\operatorname{Re} \Omega,\operatorname{Im}\Omega,\omega\right> \]
	Then consider Hodge star operator $*:\Lambda^2\to \Lambda^2$, $*^2=1$ and its complementary, this interchanges eigenvalues, negative positive, …
\end{proof}

\begin{coro}
	Signature of K3 surface is $(3,19)$
\end{coro}

\subsection{The period space of complex structures is Grassmanian}

\begin{claim}
\[\mathbb{P}\operatorname{er}=\dfrac{\operatorname{SO}(3,b_2-3)}{\operatorname{SO}(1,b_2-3) \times \operatorname{SO}(2)}=\operatorname{Gr}_{++}(h^{2}(M,\mathbb{R})\]
\end{claim}

\begin{remark}
	$(V,q)$ real vector space signature $q$ is $(m,n)$, $m\geq 2$ then
\[\operatorname{Gr}_{++}(V,q)=\{\ell\in\mathbb{P}V_{\mathbb{C}}:q(\ell,\ell)=0,q(\ell,\bar{\ell})>0\} \]
Recall that
$T_p\operatorname{Gr}_{++}=\operatorname{Hom}(?)$
\end{remark}

\begin{proof}[Of the claim]\leavevmode 
	\begin{enumerate}[label=\textbf{Step \arabic*}]
		\item $\ell\in\mathbb{P}\operatorname{er}$
		\begin{align*}
			q(\operatorname{Re}(\Omega),\operatorname{Im}(\Omega))&=0\\
			q(\operatorname{Re}(\Omega),\operatorname{Re}(\Omega))=q(\operatorname{Im}(\Omega),\operatorname{Im}(\Omega))>0
		\end{align*}
		What is going on
		\begin{align*}
			\omega_1&=\operatorname{Re}(\Omega),\qquad \omega_2=\operatorname{Im}(\Omega),\qquad \Omega \in\ell\\
			q(\omega_1+\sqrt{-1} \omega_2,\omega_1+\sqrt{-1}\omega_2&=0\\
			&=q(\omega_1,\omega_1)-q(\omega_2,\omega_2)+\sqrt{-1} 2q(\omega_1,\omega_2)
		\end{align*}
		and also 
		\begin{align*}
			q(\Omega,\overline{\Omega})&>0\\
			=q(\omega_1-\sqrt{-1}\omega_2,\omega_1+\sqrt{-1}\omega_2)&=q(\omega_1,\omega_1)+q(\omega_2,\omega_2)>0
		\end{align*}
		so we have obtained from a line in Period a positive definite plane

		\item $p\in\operatorname{Gr}_{++}$. Project, obtain a quadric form on $\mathbb{C}^2$.
			There exist two lines in $P_{\mathbb{C}}, $ $ \ell,\overline{ \ell}$ such tat
			\[q(\ell,\ell)=0, \qquad q(\bar{\ell},\bar{\ell})=0  \]
			\[q(x,y)=xy\]
	\end{enumerate}
\end{proof}

\begin{coro}
	$U\subset \operatorname{Teich}$, $V\subset H^{2}(M,\mathbb{R})$ set of all nonzero $(1,1)$-classes on $H^{2}(M,I)$ for some $I\in U$ Then $V\subset H^{2}(M,\mathbb{R})$ is open.
\end{coro}

\begin{proof}
Idea: take a 2 dimensional space and move it a bit everywhere, consider orthogonal complement. Deform $P$ by taking a $y$ and considering its orthogonal complement. $y$ is chosen close to $x$.

\[\begin{tikzcd}
	X\in P^\perp \text{is a class of type (1,1)}\\
	y \text{near }x\\
	P \text{project to }y^\perp \\
	\text{so you have an open set in Grassmanian} 
\end{tikzcd}\]
\[U_x\overset{\phi}{\longrightarrow}\operatorname{Gr}_{++}\qquad \qquad \phi^{-1}(U)\]


	\begin{enumerate}[label=\textbf{Step \arabic*}]
		\item 	Take a complex structure $I\in\operatorname{Teich}$, $P\subset H^{2}(M,\mathbb{R})$, then $H^{11}(M,I)=P^\perp$.

			\item Teichmüler is locally diffeomorphic to $\operatorname{Gr}_{++}$. Suffices to show in a nieghbourhood $U_1 \ni P$ in $\operatorname{Gr}_{++}$ that $\bigcup_{P_1\in U_1}P_1^+ $ is open.

		\item $y\in H^{2}(M,\mathbb{R})$, $y\in U_x$, nonzero in a neighb of $x\in P^\perp$. $P_y$ projection from $P$ to $y^\perp$
	\end{enumerate}
\end{proof}

\subsection{Intersection form on a K3 surface}

\begin{lemma}[Of linear algebra]
	Consider bilinear symmetric form on $V_\mathbb{Z}$
	\begin{align*}
		\pi:V_R\setminus 0  &\longrightarrow \mathbb{P}V_Q \\
		 &\longmapsto 
	\end{align*}
	where $R$ is the set of odd vectors and $Q$ rational vectors. Then  $p(\text{odd vectors} )$ is dense on $\mathbb{P}V_Q$.
	
\end{lemma}

\begin{proof}\leavevmode 
	\begin{enumerate}[label=\textbf{Step \arabic*}]
		\item Construct a sequence of odd vectors converging to any element $s\in V_{\mathbb{Z}}\setminus 0$.
			\[\lim_n\pi(r_0+2ns)=\pi(s)\]
	\end{enumerate}
\end{proof}

\begin{thm}
	Intersection form of K3 is even.
\end{thm}

\begin{proof}
	\begin{enumerate}[label=\textbf{Step \arabic*}]
		\item suppose it is odd. Coro 1 lema 1 imply complex structure $I$ and odd vector $r\in H^{1}(M,I)$. The point is that the set of vectors of type 11 is open negithboudood of any class. But then the odd classes are dense.

		\item Involves Riemann-Roch. For each class $r\in H^{11}(M)\cap H^{2}(M,\mathbb{Z})$ we have a holomorphic line bundle by the exponential sequence, $c_1|L|$, $L$ is hol line bundle:
			\[\begin{tikzcd}
				0\arrow[r]&\mathbb{Z}_M\arrow[r,"\sqrt{-1}2\pi"]&\mathcal{O}_M\arrow[r,"\operatorname{exp}"]&\mathcal{O}^*_M\arrow[r]&0
			\end{tikzcd}\]
			and
			\[\begin{tikzcd}[column sep=small]
				\cdots\arrow[r]&0=H^{1}(\mathcal{O}_M)\arrow[r]&\operatorname{Pic}= H^{1}(\mathcal{O}^*_M)\arrow[r]&H^{2}(M,\mathbb{Z})\arrow[r,"\operatorname{proj}"]&H^{02}(M)=H^{2}(\mathcal{O}_M)\arrow[r]&\arrow[r]&\cdots
			\end{tikzcd}\]
			Now Riemann-Roch:
			\[\chi(L)=\chi(\mathcal{O}_M-\frac{L(K-L)}{2}=2-(L,L)/2\]
			because canonical bundle is trivial, so that cannot be odd.
	\end{enumerate}
\end{proof}

\subsection{Smooth quartics}

\begin{defn}
	A \textit{\textbf{smooth quartic}} is a smooth quartic hypersurface in  $\mathbb{P}^3$. So a solution of a quartic equation, ie. polynomial of degree 4.
\end{defn}

\begin{remark}
	Adjunction formula. Canonical bundle of quartic is canonical bundle of $\mathbb{C}P^{3}$ restrictesd to quartic times normal bundle:
	\[K_Q=K_{\mathbb{C}P^{3}}|_{Q}\otimes N(Q)\]
	But $N(Q)$ is degree four so it is just $\mathcal{O}(4)=N(Q)$.

	and canonical bundle $K_{\mathbb{C}P^{3}}$ of $\mathbb{C}P^{3}$ is $\mathcal{O}(-4)$ by Euler formula.

	So $K_Q=0_Q$---quartic has trivial canonical bundle (it is Calabi-Yau).
\end{remark}

\[V:\mathbb{C}P^{3}\hookrightarrow \mathbb{C}P^{34}\]
What is this map. It is associated to $\mathcal{O}(4)$, with the line system
\[\mathbb{C}P^{34}=\mathbb{P}H^{0}(\mathcal{O}4)^*\]
and it is called Veronese map.

\begin{claim}
	Smooth quartic is a hyperplane section of $V(\mathbb{C}P^{3})$.
\end{claim}

So any hyperplane on $\mathbb{C}P^{34}$ is a hyuperplane on $\mathbb{C}P^{3}$.

…So the zeroes of this restriction are quadrics.

\paragraph{The point} is that quartics are (in correspondence with) hyperplane sections.

All quartics are sections of Veronese.

\subsection{Smooth quartics and Lefschetz hyperplance section theorem}

\begin{thm}[Lefshetz hyperplane]\leavevmode
	$ \pi(H\cap V)=\pi_1(V)$
\end{thm}

\section{Class 8: smooth quartics}

\subsection{Lefschetz again}

\begin{thm}[Lefschetz hyperplane]\leavevmode
	If $X \subset \mathbb{P}^n$ and $H=\mathbb{C}P^{n-1}$ a hyperplane in $\mathbb{C}P^{n}$ and $X\cap H$ (transversal justo to be safe), $X\cap H\to  X$ isomorphism on homotopy group $\pi_i$ for  $i<\dim X$, ie.
	$\pi_{i}(Z\cap H)\overset{\cong }{\longrightarrow}\pi_{i}(Z)$
\end{thm}

\begin{proof}
	Will discuss later but make a cellular decomposition that puts cells of certain dimension in the intersection.
\end{proof}

\begin{coro}
	$\pi_{1}(\text{smooth quartic} ) =\pi_{1}(\mathbb{C}P^{3}) =0$.
\end{coro}

\begin{coro}
	Smooth quartic is K3.
\end{coro}

\subsection{Smooth submersions}

\begin{defn}
	\textit{\textbf{smooth submersion}} is a map  $\pi:M\to M'$ such that $d\pi$ is surjective everywhere.
\end{defn}

\begin{remark}
	Submersions are just products: each point has a neighbourhood that looks like a product and submersion is projection on one factor.
\end{remark}

\begin{thm}[Ehresemann fibration theorem]\leavevmode
	Let $\pi:M\to M'$ be a smooth submersion of compact manifolds. Then $\pi$ is locally trivial fibration.
\end{thm}

\begin{proof}
	It is a vector bundle because
	\[\begin{tikzcd}
		0\arrow[r]&T_\pi M\arrow[r]&TM\arrow[r,"d\pi"]&\pi^*TM'\arrow[r]&0
	\end{tikzcd}\]
	where $T_\pi M$ is the vertical subbundle ie. $\ker \pi$

	\textit{\textbf{Ehresemann connection}} is a decomposition  $T_{\text{horizontal} }\oplus T_{\text{vertica} }=TM$. Then there is a projection $d\pi:T_\text{hor}\to TM' $ and an associated curve. This gives the diffeomorphism that says all fibers are diffeomorphic. (see slides)
\end{proof}

\subsection{Space of smooth quartics}

Let $V=\mathbb{C}^{35}=\operatorname{Sym}^4\mathbb{C}^4$ be the set of homogeneous degree 4 polynomials in 4 variables. Interpret $P\in V$ as a quartic equation in $W=\mathbb{C}^{4}$

\begin{claim}
	Let $Z\subset \mathbb{P}V\times \mathbb{C}P^{3}$ be the set $\{(P\in\mathbb{P}V,w\in\mathbb{C}P^{3}:P(w)=0\}$. Then $Z$ is smooth and irreducible.
\end{claim}

\begin{proof}
	\begin{enumerate}[label=\textbf{Step \arabic*}]
		\item So we have a point in a hyperplane and the hyperplane is in $\mathbb{C}P^{3}$ Veronese (ie. embedded, it is a quartic $Q$). So we have $x\in\ell$, $\ell$ hyperplane section. And then let $\tilde{Z}\subset \mathbb{C}^4\times \mathbb{C}^{35}$ be "the corresponding set of vectors". So $Z=\tilde{Z}/\mathbb{C}^*\times \mathbb{C}^*$. Clearly it suffices to show that $\tilde{Z}$ is smooth (?).

		\item Take the derivative of $(P+tQ)(w)$, it is not zero and so  $\tilde{Z}$ is smooth.

		\item Use Sard's lemma or Bertini theorem + Lefschetz hyperplane to show $F$, the general fiber of the projection of  $Z$ to  $\mathbb{C}P^{34}$, is connected and hence $Z$ is irreducible.

			How to use LEfschetz?
\[V(\mathbb{C}P^{3})\subset \mathbb{C}P^{34}\]
\[Q=V(\mathbb{C}P^{3})\cap H\]
and put all the cells in one half and then the other half just remains connected.
	\end{enumerate}
\end{proof}

\begin{question}
	Is there a better way to show that a general smooth quartic in $\mathbb{C}P^{3}$ is connected (without Lefschetz)?

	Use that all quartics are equivalent (outside discriminant) and then just use $X_1^4$. (…?)
\end{question}

\subsection{ Smooth quartics are diffeomorphics}

\begin{coro}
	Smooth quartics are differomorphic.
\end{coro}

\begin{proof}
	\[\begin{tikzcd}
	Z\arrow[d,"Q"]\\
	\mathbb{C}P^{34}\supset\mathcal{D}
	\end{tikzcd}\]
	We want to prove that the fibers are diffeomorphic. We need to remove the non-smooth fibers of this map. The critical values are the (…) is called \textit{\textbf{discriminant}}.  So $\mathcal{D}$ is the set of al singular quartics. And $\mathbb{C}P^{34}\setminus \mathcal{D}$ is connected.

\begin{exercise}
	Complement to proper subvariety is connected. Take two points and try to join them. Vanya: they intersect $\mathcal{D}$ in a finite ammount of points. Misha: every path can be deformed to a path that avoids $\mathcal{D}$ by Sard's theorem because $\mathcal{D}$ has codimension 2.
\end{exercise}
Then all fibers of 
\[\begin{tikzcd}
Z\setminus \pi^* (\mathcal{D})\arrow[d,"\pi"]\\
\mathbb{C}P^{34}\setminus \mathcal{D}
\end{tikzcd}\]
$\pi$ are diffeomorphic because $\pi$ is a proper smooth submersion.
\end{proof}

\begin{remark}
	The same arguemtns shows that smooth hypersurfaces of degree $d$ in $\mathbb{C}P^{n}$ are diffeomorphic.
\end{remark}

\subsection{Ample bundles}

\begin{defn}
If you have $\varphi:X\to \mathbb{C}P^{n}$ projective complex, then 
\[\varphi^*(\mathcal{O}(1))\]
is called \textit{\textbf{very ample}} and  $L$ is \textit{\textbf{ample}} if $L^{\otimes n}$ is very ample for some $n>0$.
\end{defn}

Kähler classes are classes of Kähler forms.

\begin{thm}[Kodaira]\leavevmode
	$L$ is very ample iff $c_1(L)$ is a Kähler class.
	
\end{thm}

\paragraph{Objective} All K3 are diffeomorphic. Need to prove that quartics are dense in the universal family of K3 over its Teichmüler space. Then we can deform a bit complex structure, deformatino doesn't change topology.


We need to identify the quartics among all K3 surfaces $M$ containing $x\in\operatorname{Pic}(M)$ such that $x\cap x=4$ in other words 

You have a quartic $Q$, the generator of picard is $\mathcal{O}(1)$. Its self intersection is 4, to see if consider $[H\cap Q]$, intersection with hyperplane.

And we want that to be very ample. $c_1(L)\cap c_1(L)=4$.

\begin{remark}[Dani]\leavevmode
	Not every K3 is quartic but every K3 is very close to a quartic in Teichmüler.
\end{remark}

\begin{remark}
	Pellisky? did it with Kummer (surfaces?), which much harder.
\end{remark}

\subsection{Very ample bundles}
Interpret very ampleness as vanishing of cohomology groups.

\begin{claim}[The Following AreEquival ]\leavevmode
	\begin{enumerate}[label=(\roman*)]
		\item $\phi_L:X\to \mathbb{C}P^{n}=\mathbb{P}H^{0}(X,L)$ is injective and holomorphic.
		\item $\forall x,y$ exists section $\gamma \in H^{0}(X,L)$ with $\mathcal{D}=$zero $\gamma$, $X\in\mathcal{D}, y \not\in\mathcal{D}$.
	\end{enumerate}
\end{claim}

This is equivalent to
\[\begin{tikzcd}
H^{0}(X,L)\arrow[d]\\
H^{0}(X,\frac{L}{(\mathfrak{m}_x\cap \mathfrak{m}_y})\otimes L)
\end{tikzcd}\]
And that thing is some skyscraper things but is reall only $\mathbb{C}^2$.

\begin{remark}
	$\mathfrak{m}_x$ max ideal of $x$. The 1-jets of functions in $x$ is $\mathcal{O}_X/\mathfrak{m}_x^2$. Then the natural map
	\[\phi_LX:\to \mathbb{P}H^{0}(L)\]
	is non-zero for all $X$. There exists section $\gamma$ 1-jet $x$ is non-zero $\gamma(x)=0$.
\end{remark}

\begin{defn}
	$f,g\in\mathcal{O}_{X,x}$ germs of functions at $x$ (or even sections of line bundle in neighbourhood of a point). $f$ has same  $k$-jet as $g$ if $f-g$ has zero order $k+1$. Same Taylor series up to $k+1$. 

	0jet is Taylor. 1-jet is Taylor plus differential. And the good thing is that $\mathfrak{m}/\mathfrak{m}^2=T_xM^*$. Kernel of differential is kernel on 1-jets.
\end{defn}

\subsection{Alternative description of very ampleness}

\begin{coro}
	$L$ bundle on compact complex manifold $X$. equivalent:
	\begin{enumerate}[label=(\roman*)]
		\item $L$ very ample.
		\item \[ H^{0}(L)\to H^{0}(L)/(\mathfrak{m}_x\cap \mathfrak{m}_y)\otimes L)\] (skyscraper sheaves isomorphic to $\mathbb{C}$) is surjective and also
				\[H^{0}(L)\to H^{0}(L/(L\otimes \mathfrak{m}_x^2)\]
				is surjective too. On right hand side of the second one in fact that it cotangent space (some comment about thinking of this like some coordinate system).
	\end{enumerate}
\end{coro}

Now think of this short exact sequence of coherent sheaves
\[\begin{tikzcd}
	0\arrow[r]&L\otimes (\mathfrak{m}_x\cap\mathfrak{m}_y)\arrow[r]&L\arrow[r]&L/(\mathfrak{m}_x\cap\mathfrak{m}_y)\otimes L\arrow[r]&0
\end{tikzcd}\]
and the last sheaf is just finite dimensional space. This gives a long exact sequence.
\[\begin{tikzcd}[column sep=small]
	\cdots\arrow[r]&H^{0}(L)\arrow[r]&H^{0}(L)/\mathfrak{m}_x\cap \mathfrak{m}_y\arrow[r]&H^{1}(L\otimes (\mathfrak{m}_x\cap \mathfrak{m}_y))\arrow[r]&\arrow[r]&\arrow[r]&\cdots
\end{tikzcd}\]
If cohomology of that last one vanishes and also $H^{1}(L\otimes \mathfrak{m}_x^2)$ vanishes you are very ample. So its a vector bundle.

\begin{remark}
	So, for curves very amplness is very easy to check. Because every module is sum of rings by list 1 we have that the last one on the sequence is a vector bundle, finitely generated coherent, and then Kodaira says that vanishes.
\end{remark}

Canonical bundle is very ample unless curve is elliptic?

\section{Class 9: Nakai-Moishezon theorem}
Today it will be mostly algebraic geometry, so no K3.

\subsection{Ample bundles}
Looks like here's another definition:

\begin{defn}\leavevmode
	$L$ is \textit{\textbf{very ample over $X$}}  if
	\[X\hookrightarrow \mathbb{P}H^{0}(x,L)^*\]
	and $L$ is  \textit{\textbf{ample}} if $L^{\otimes N}$ is very ample for some $N>0$.
\end{defn}

\begin{remark}
	$\overline{L}$ holomorphic line bundle on compact complex, then $L$ is ample on a convex complex cone if and only if $\operatorname{deg}(L)>0$.
\end{remark}

\subsection{Very ample bundles}


\begin{claim}
	$x\neq y\in X$ then
	\[\varphi:H^{0}(L)\longrightarrow H^{0}(L/(\mathfrak{m}_x\cap \mathfrak{m}_y)\]
	is surjective, and the standard map
	\[\varphi_L\longrightarrow \mathbb{P}H^{0}(X,L)^*\]
\end{claim}

Two sections that have different derivative (1-jet) have different images, then the derivative is non-zero (derivative is injective). Uses inverse function theorem?

\begin{proof}\leavevmode
From (surjectivity?) of $\varphi$ we get that
	\[H^{0}(L)\longrightarrow H^{0}(L/\mathfrak{m}_x^2)\]
	Then $\varphi$ is isom to its image. (Because $\varphi=\varphi^*\Big|_{\mathfrak{m}_x/\mathfrak{m}^2}$)
\end{proof}

\subsection{Very ample bundles again}

\begin{coro}\leavevmode
	If $H^{1}(L/(\mathfrak{m}_x\cap \mathfrak{m}_y))=0$ and $H^{1}(L\otimes \mathfrak{m}_y^2)=0$ for all $x,y$ then  $L$ is very ample.
\end{coro}

\begin{proof}\leavevmode
	\[\begin{tikzcd}
		0\arrow[r]&H^{0}(L)\arrow[r]&H^{0}(L/\mathfrak{m}_x^2)\arrow[r]&H^{1}(\mathfrak{m}_x^2\otimes L)\arrow[r]&0
	\end{tikzcd}\]
\end{proof}

\subsection{Very ample bundles on a curve}

\begin{thm}[Kodaira-Nakano vanishing]\leavevmode
	$L$ holomorphic line bundle on compact complex curve, $L\otimes K^{-1}_M$ ample. Then $H^{i}(L)=0$ for all $i>0$
\end{thm}
Now we can do something about curves.

\begin{coro}\leavevmode
	Let $L$ be a line bundle on a compact complex curve  $C$ of genus $g$, and $\operatorname{deg}L>2g$.
\end{coro}

\begin{proof}\leavevmode
		\begin{enumerate}[label=\textbf{Step \arabic*}]
		\item We need to prove that
	\begin{align*}
		H^{1}(L\otimes(\mathfrak{m}_x\cap \mathfrak{m}_y))=0\qquad \qquad H^{1}(L\otimes \mathfrak{m}_x^2)
	\end{align*}
	{\color{2}Slides: }The sheaves $L\otimes\mathcal{O}_X\otimes(\mathfrak{m}_x\cap \mathfrak{m}_y)$ are line bundles of $\operatorname{deg}L-2$. {\color{4}Board:} $L\otimes \mathfrak{m}^2$ and $L\otimes(\mathfrak{m}_x\cap \mathfrak{m}_y)$ are line bundles of degree $\operatorname{deg}L-2$.

	So what is it anyway?
	 \[L\otimes \mathfrak{m}_x^2=L(-2x)\qquad \qquad L\otimes \mathfrak{m}_x\cap \mathfrak{m}_y=L(-x-y)\]

	 \item The degree of the canonical bundle is $2g-2$. {\color{6}By Kodaira}, $L_1$ ample if and only if $\operatorname{deg}(L_1\otimes K_X^{-1})>0$ iff $\operatorname{deg}L_1>2g-2$.  $(L\otimes \mathfrak{m}_x^2)\otimes K^{-1}_X$ ample and equals $H^{1}(L\otimes \mathfrak{m}_x^2)=0$ by Kodaira-Nakano, $L_1=L\otimes \mathfrak{m}_x$, or $L_1=L\otimes(\mathfrak{m}_x\cap \mathfrak{m}_y$
	\end{enumerate}
\end{proof}


\subsection{Canonical map for a complex curve}

\begin{defn}\leavevmode
	Let $L$ be a line bundle on $X$. A point $p \in X$ is called a \textit{\textbf{base point}} if all sections of $L$ vanish in $p$.
\end{defn}

\begin{defn}\leavevmode
	Assume $K_X$ has no base points. Then
	\[\psi_{K_X}:X\longrightarrow \mathbb{P}H^{0}(X,K_X)\]
	is called the \textit{\textbf{canonical map}}.
\end{defn}

\begin{thm}\leavevmode
	$C$ curve, $g(L)\geq 2$. $K$ canonical bundle. Then  $H^{0}(K)$ has no common zeroes, the canonical map $\psi:C\longrightarrow \mathbb{P}H^{0}(K)^*$ is embedded or is a two-sheeted (2-to-1) ramified covering to $ \psi(C)=\mathbb{C}P^{1}$.
\end{thm}

\begin{remark}
	In the second case $C$ is called a \textit{\textbf{hyperelliptic curve}}. In step 3, we will prove that any curve admitting a two sheeted ramified covering to  $\mathbb{C}P^{1}$ is hyperelliptic.

	This gives you two theorems about moduli spaces. Two curves are isomorphic if these subvarietes are conjugated by linear map, so gives you a point in Hilbert scheme. So hyperelliptics are the same as $\mathbb{C}P^{1}$… ah, but it has to be separated, $(\mathbb{C}P^{1})^n$, and without the diagonals, and 
	\[\Big((\mathbb{C}P^{1})^n\setminus \text{diagonals}\Big)\Big/\operatorname{PGL}(2,\mathbb{C})\times \{2^{2g}\} \]
	The power of 2 term are possible choices of ramification.
\end{remark}

\begin{proof}\leavevmode
	\begin{enumerate}[label=\textbf{Step \arabic*}]
		\item First we need to show there are no common zeroes. That is, sections of $K$ have no common zeroes. Let $ p \in\mathbb{C}$ and let $k_p=\mathcal{O}_C/\mathfrak{m}_p$. Consider 
			\[\begin{tikzcd}
			0\arrow[r]&K(-p)\arrow[r]&K\arrow[r]&k_{p}\arrow[r]&0
		\end{tikzcd}\]
The corresponding long exact sqeuence says
\[\begin{tikzcd}
	H^{0}(K_C)\arrow[r]&H^{0}(k_p)\arrow[r]&H^{0}(K(-p))
\end{tikzcd}\]
so surjectivity is equivalent to $p  \not \in $ common zeroes.

But $H^{1}(K(-p))=0$ because $H^{1}(K(-p))=H^{0}(\mathcal{O}(p))^*$ by Serre duality. So we only need to show that there are no section is the latter degree 1 bundle. If there exists $\gamma \in H^{0}(\mathcal{O}(p))$ meromorphic function with a single pole on $p$,  we obtain a holomorphic function $f:C\longrightarrow \mathbb{C}P^{1}$ of degree 1, meaning $C=\mathbb{C}P^{1}$.

\item $\psi$ non-injective, that it glues together $p$ and $q$. Then $H^{1}(K(-p-q))\neq 0$.And by Serre duality $H^{0}(\mathcal{O}(p+q)=H^{1}(K(-p-q))\neq 0$. Then there exists a meromorphic function $f$ with poles on $p,q$ and  $f:C\longrightarrow \mathbb{C}P^{1}$ of degree 2.

\item So you have a map with two preimages. So consider a map that exchanges the preimages.

		Now we will show that $\psi(C)=\mathbb{C}P^{1}$ admits a two sheeted ramified covering to $\mathbb{C}P^{1}$. Let $\tau:C\longrightarrow C$ be the involution exchanging the sheets of the covering. It is holomorphic because it has only Riemann-extendible singularities. And it acts on $H^{0}(K_C)$ with eigenvalues $\pm 1$. But $H^{0}(\Omega^{1}(\mathbb{C}P^{1}) )=0$ so $\tau|_{H^{0}(K_C)}=-\operatorname{id}$. Therefore we see that $\psi$ glue $p$ and  $q$.
	\end{enumerate}
\end{proof}

\subsection{Finite morphisms}

\begin{defn}\leavevmode
	Let $f:X\longrightarrow Y$ a morphism of varieties (or schemes). $f$ is \textit{\textbf{finite}} if for every $U\subset Y$ open, $\mathcal{O}_{f^{-1}(U)}$ is finitely generated as an $H^{0}(\mathcal{O}_U)$-module.

	{\color{5}Board:} if the ring $f^*\mathcal{O}_U$ is finitely generated as an $\mathcal{O}_V$-module where $V=f^{-1}(U)$.
\end{defn}

\begin{thm}\leavevmode
	$f:X\longrightarrow Y$ proper and the preimage of any point is finite.
\end{thm}

\begin{proof}\leavevmode
	Hartshorne exerice III 11.2. And also past courses and EGA.
\end{proof}

\subsection{Amplennes and cohomology}

\begin{thm}\leavevmode
	$L$ is ample iff $\forall $ coherent $F$ there exists $d>0$ such that  $H^{i}(F\otimes L^{\otimes k})=0$ for all $i>0$ and  $k\geq $.
\end{thm}

\begin{proof}\leavevmode
	Hartshorne.
\end{proof}

\begin{thm}\leavevmode
$f$ finite functor then pusforward is acyclic exact functor.

$f:X\longrightarrow Y$ finite map, $F$ coherent sheaf on $X$. Then
\[H^{i}(f^{-1}(U),F)=H^{i}(U,f_* F)\]
for any open set $U\subset Y$; in other words, $R^if_* F=0$ for all $i>0$.
\end{thm}

\begin{proof}\leavevmode
	The Rising Sea, thm 18.7.5.
\end{proof}

\begin{idea4}{Corollary 1}\leavevmode
	$L$ line bundle on a complex variety $X$ such that the standard map $f:X\longrightarrow \mathbb{P}H^{0}(X,L)^*$ is finite. Then $L$ is ample.
\end{idea4}

\begin{proof}\leavevmode
	(Proof is simple but unfortunately uses complicated theorems.)

	Let $Y=f(X)$ and $F$ a coherent sheaf on $X$. We have that $L=f^*(\mathcal{O}(1))$. And then there is the formula of base-change (maybe) which says (and it works for any other sheaf instead of $\mathcal{O}(1)$) that $f_*(F\otimes_{\mathcal{O}_X}L^{\otimes k}=f_* F\otimes_{\mathcal{O}_Y}\mathcal{O}(k)$. Ok and then we can compute cohomology:
	\begin{align*}
		H^{i}(X,F\otimes L^{\otimes n})&=H^{i}(Y,\underbrace{F\otimes \mathcal{O}(n)}_{=0}
	\end{align*}
	so $L$ is ample.
\end{proof}

What is this for? We want to explore some invariants on K3.

\subsection{Nakai-Moshezon theorem}

It's a very nice criterion for ampleness. It can be generalized to Kähler, but we won't do that.

\begin{thm}[Nakai-Moshezon]\leavevmode
	Let $L$ be a line bundle on a projective variety $X$. Suppose for all subvarieties $Y\subset X$
	\[\int_{Y}c_1(L)^d>0,\qquad d=\dim Y\]
	then $L$ is ample.
\end{thm}

\begin{proof}\leavevmode It's seven steps.
	\begin{enumerate}[label=\textbf{Step \arabic*}]
		\item Let's do induction on $\dim X$. For $\dim X=1$ it is clear. Assume that $L$ is ample on all proper $X_1\subsetneq X$. The next step is the {\color{3}most difficult step}: show that $H^{0}(X,L^{\otimes n})\neq 0$ for $k\gg 0$

		\item We need a very ample bundle. Let $L_1$ be a very ample bundle with a sufficiently big $c_1$ such that $c_1(L\otimes L_1\otimes K^{-1}_X$ is Kähler ($\overset{?}{=}$ ample). Then $H^{1}(L_1\otimes L)=0$ for all $i>0$. Let  $H$ be a smooth zero divisor of $L_1$ such that $\mathcal{O}(H)=L_n$. Now consider the short exact sequence
			\[\begin{tikzcd}
				0\arrow[r]&L\arrow[r]&L\otimes \mathcal{O}(H)\arrow[r]&L\otimes \mathcal{O}(H)|_{H}\arrow[r]&0
			\end{tikzcd}\]
			I took a section of $L_1$, wrote this exact sequence and the one on the right is ample by assumption. Then we can replace $L$ by a sufficiently big power $L^{\otimes d}$, we may assume that $L^{\otimes d}|_{H}$ is ample. (We want to show $L_1$? is Kähler. If you are an algebraic geometer then maybe you'd say something like any bundle is ample if you multiply be a sufficiently large power…) {\color{4}I think} here we tensor multiply to get
			\[\begin{tikzcd}
				0\arrow[r]&\mathcal{O}(-H)\arrow[r]&\mathcal{O}\arrow[r]&?\arrow[r]&0
			\end{tikzcd}\]
			Anyway, we get
			\[\begin{tikzcd}[column sep=small]
				\cdots\arrow[r]&H^{i-1}(L^{\otimes d}\otimes \mathcal{O}(n)|_{H})\arrow[r]&H^{1}(L^{\otimes d})\arrow[r]&H^{i}(L^{\otimes d}\otimes L_1)\overset{?, \text{board} }{=}H^{1}(L^{\otimes d}\otimes \mathcal{O}(H))\arrow[r]&\cdots
			\end{tikzcd}\]
			then there exists $d\gg 0$ such that  $H^{i}(L^{d+j})=0$ for all $j\geq 0$ and $i>1$.

			(We didn't prove  $L$ is ample on  $X$. The idea is that the curvature of $\mathcal{O}(n)\otimes L^{\otimes d}$ is strictly positive ($\overset{?}{=}$ has sections. Perhaps the use of $d$ can be avoided if we choose $H$ properly…)


			Again, we have 
			\[\begin{tikzcd}[column sep=small]
				\cdots\arrow[r]&H^{i-1}(L^{\otimes d}\otimes \mathcal{O}(n)|_{H})\arrow[r]&H^{1}(L^{\otimes d})\arrow[r]&H^{i}(L^{\otimes d}\otimes \mathcal{O}(n))\arrow[r]&\cdots
			\end{tikzcd}\]
and then
\begin{align*}
	K_H&=K_M|_{H}\otimes \mathcal{O}(n)\\
	K_H^{-1}&=K_M|_{H}\otimes \mathcal{O}(-n)
\end{align*}

\begin{idea5}{We proved}\leavevmode
	there's only two possible non zero cohomology which is $H^{1}$ and $H^0$.

	That is, there exists $d\gg 0$ such that $H^{i}(L^{\otimes j})=0$ for $i>1$,  $j>d$
\end{idea5}

\item Looks like by Riemann-Roch, $\chi(kL)$ is a polynomial of  $k$ of degree $n$ given by Todd.
	\begin{align*}
	\int_{X}\frac{kc_1(L)^n}{n!}&=\lim_{k \to \infty} \chi(kL)=\infty\\
\implies \lim_{k \to \infty} \operatorname{Ch}_{\cdot}(H^{0}(kL))=\infty
	\end{align*}

\item Replace $L$ by $L^{\otimes k}$. Can assume $\dim H^{0}(L)>d$. Then
	\[\begin{tikzcd}
		0\arrow[r]&(K-L)L\arrow[r]&kL\arrow[r]&kL|_D\arrow[r]&0
	\end{tikzcd}\]
	Now by inductive assumption $L|_{D}$ ample. So $H^{i}(kL)=0$ for all $k>d$ and  $i>0$. Then we get the long exact sequence
	 \[\begin{tikzcd}[column sep=small]
		0\arrow[r]&H^{0}((k-1)L)\arrow[r]&H^{0}(kL)\arrow[r]&H^{0}(kL|_{D})\arrow[r]&H^{1}((k-1)L)\arrow[r]&H^{1}(kL)\arrow[r]&0
	\end{tikzcd}\]
	Now the function $k\mapsto \dim H^{1}(kL)$ is monotonous non-increasing, so it must stabilize. Therefore for $k\gg 0$ we get an surjection $H^{0}(kL)\longrightarrow H^{0}(L|_{D})$.

	\item The birrational map $\Phi:X\longrightarrow \mathbb{P}H^{0}(kL)$. Every $x\in X$ is connected in $D $ for some section of $L$ to every $ y\in D$ (?). {\color{3}Slides:} $D$ can be chosen as $\Phi^*(H)$ where $\Phi^*$ is proper preiage and $H$ a hyperplane section in $\mathbb{P}H^{0}(L)^*$, therefore for any two points $x,y\in X$ we may choose $D$ containint these two points.

		Then there exists a section of  $kL$ non-zero on  $y$, implying that  $\Phi$ is holomorphic. So there is a section of $kL$ separating these points (vanishing in one and non-vanishing in the other). For each point, there is a section that is not zero.

	 \item The bundle $kL$ is ample by Corollary 1.
\end{enumerate}
\end{proof}

\begin{remark}
	See the book \textit{Positivity in algebraic geometry}.
\end{remark}




\section{Class 10: surfaces with Picard rank 1}

\subsection{Intuition and review}

Today we'll see how to construct quadrics. Any K3 with rank 1 and Picard rank 4 is a quartic.

\begin{idea4}{Mumford}\leavevmode
	offered money to anyone who could give a K3 with Picard rank 1. Took 30 years.
\end{idea4}

So far we have:

\begin{thm}[Kodaira]\leavevmode
	$L$ ample $\iff$ $c_1(L)$ Kähler.
\end{thm}

\begin{thm}[Kodaira-Nakano vanishing]\leavevmode
	$L\otimes K^{-1}_M\implies H^{i}(L)=0\forall i>0$.
\end{thm}

\begin{thm}[?]\leavevmode
	$C$ compact complex smooth curve then $K_C$ is globally generated  $\phi:C\longrightarrow \mathbb{P}H^{0}(K_C)$ is 2:1 smooth cover or embedding.
\end{thm}

\begin{thm}[result, see last class]\leavevmode
	Is ample iff kills cohomology
\end{thm}

\begin{idea2}{Corollary 1}\leavevmode
	$X\longrightarrow \mathbb{P}H^{0}(X,L)^*$ is finite then $L$ is ample.
\end{idea2}

\begin{idea5}{Intuition}[of what's about to happen]\leavevmode
	Consider a curve with self-intersection (a hand-drawn line/string that intersects itself). Then we will resolve that singularity by "streching the string" or "separating the branches at the intersection point".
\end{idea5}

\subsection{Singular curve in a K3 surface}

\begin{claim}\leavevmode
	Let $C\subset M$ be a curve in a singular complex surface. Then there exists a surface $\tilde{M}\xrightarrow{\pi}M$ obtained by succesive blow-ups of $M$ such that the proper preimage $\tilde{C}$ of $C$ is smooth. (The \textit{\textbf{proper preimage}} is taking the points in the curve and not the exeptional divisord.)
\end{claim}

\begin{defn}\leavevmode
	Multiplicity of a singular point is dimension? of cohomology. \end{defn}

\begin{proof}\leavevmode
	Take the single blow-up of $C$ in a singular point. It has smaller multiplicity; We are saying that $\tilde{C}$ has strictly smaller multiplicity. For this we have thought of
	\[\begin{tikzcd}
	\tilde{M}\arrow[d,"\pi"]\\
	M\subset C
	\end{tikzcd}\]
	and the exceptional divisor is $E$ in $\pi^{-1}(C)=\tilde{C}+E$. So to compute mutliplicity we have done
	\[\tilde{L}\cap(\tilde{C} \times E)=L\cap C\]
	\[\implies \tilde{L}\cap \tilde{C}<L\cap C\]
\end{proof}

\begin{remark}\leavevmode
	So we can resolve singularities and that's what happens with multiplicity.
\end{remark}

\subsection{Singular curve in a K3 surface (corollary 2)}

So let $M$ be a K3 surface and $C$ a singular curve in K3. Then take the pullback bundle of $C$ and call it $L=\mathcal{O}(\pi^{-1}(C))=\pi^*(\mathcal{O}(C))$. Remember that $\pi^{-1}(C)=\tilde{C}+E$. Then $\mathcal{O}(\tilde{C})\otimes\mathcal{O}(E)=L$, so the \textit{normal bundle} is $N(\tilde{C})=L\otimes \mathcal{O}(-E)$ and $K_{\tilde{M}}=\mathcal{O}(E)$. So we can compute the canonical bundle: $K_{\tilde{M}}\otimes N_{\tilde{\mathbb{C}}}=K_{\tilde{C}}$, $\mathcal{O}(E)\otimes L\otimes \mathcal{O}(-E)=L$.

\begin{idea6}{Corollary 2}\leavevmode
$C$ curve of genus >0 in K3 surface $M$. Then $\mathcal{O}(C)|_{C}$ is globally generated.
\end{idea6}

\begin{proof}\leavevmode
	Need to show 
\end{proof}

\subsection{Picard rank 1 is usually very ample}

\begin{thm}\leavevmode
	$M$ K3 with $\operatorname{Pic}(M) =\mathbb{Z}$ and $L$ a generator of $\operatorname{Pic}(M)$ such that $(L,L)\geq 0$. Then $L$ or $L^*$ is globally generated.
\end{thm}

\begin{proof}\leavevmode
	\begin{enumerate}[label=\textbf{Step \arabic*}]
		\item We use Riemann-Roch (like everytime when you get a surface). We get that $h^{0}(L)-h^{1}(L)+h^{2}(L)=2+\frac{(L,L)}{2}$ so $h^{2}(L)-h^{1}(L)\geq 2$. Now since $h^{0}(L)=h^{2}(L)^*$ we may just assume that $h^{0}(L)>0$ (by interchanging $L$ by $L^*$).

		\item Here's the only place where we use that $C$ is a curve. Let $D\in |L|:=\text{zero divisor sections of $L$} $. So dimension 1 implies (it think) that $D$ is irreducible. No we have that $L=\mathcal{O}(D)$. We can construct the short exact sequence
			\[\begin{tikzcd}
				0\arrow[r]&\mathcal{O}_M\arrow[r]&L\arrow[r]&L|_{D}\arrow[r]&0
			\end{tikzcd}\]
			Which gives
			\[\begin{tikzcd}[column sep=small]
				\cdots\arrow[r]&H^{0}(L)\arrow[r]&H^{0}(L|_{D})\arrow[r]&H^{1}(\mathcal{O}_M)=0\arrow[r]&\arrow[r]&\arrow[r]&\cdots
			\end{tikzcd}\]
			so the restriction map is surjective, and every section $L|_{D}$ extends to $M$.

		\item $L|_{D}$ is globally generated. ({\color{4}Slides:} The bundle $L|_{D}$ is base point free by Corollary 2.
	\end{enumerate}

	\begin{remark}\leavevmode
		Let $\pi:(\tilde{M},\tilde{D})\longrightarrow (M,D)$ a resolution of singularities. Since $\pi^* L=K_{\tilde{D}}$, then the restriction $\pi^* L |_{\tilde{D}}$ is very ample if $\tilde{D}$ is not hyperelliptic.
	\end{remark}
\end{proof}

\subsection{The same theorem but a little different}

\begin{thm}\leavevmode
	$M$ K3 with $\operatorname{Pic}(M) =\mathbb{Z}$ and $L$ a generator of $\operatorname{Pic}(M)$ such that $(L,L)${\color{6}$>2$}. Then $L$ or $L^*$ is ample, base point free and the map $\psi:M\longrightarrow \mathbb{P}H^{0}(M,L)^*$ is an embedding or a ramified covering.
\end{thm}

\begin{proof}\leavevmode
	Now we use Corollary 1, for ampleness.
\end{proof}

\subsection{Hyperelliptic curves}

First let's count the fixed point of the involution.

\begin{lemma}\leavevmode
	If $\tau:C\to C$ is the hyperelliptic involution, then $\tau$ has $2g$ fixed points.
\end{lemma}

\begin{proof}\leavevmode
	Let $f$ be the number of fixed points. So it acts on tangent send point to minus. So simple fixed points. Here $e$ is Euler characteristic.
	\[2-2g=e(C)=e(\mathbb{C}P^{1})-f=2-f\]
	{\color{3}So where do the fixed points appear?}
\end{proof}

\begin{prop}\leavevmode
	All curves of genus 2 are hyperelliptic.
\end{prop}

\begin{proof}\leavevmode
	(uses Serre duality)
\end{proof}

\subsection{A third variation of that theorem}

\begin{thm}\leavevmode
	$M$ K3 with $\operatorname{Pic}(M) =\mathbb{Z}$ and $L$ a generator of $\operatorname{Pic}(M)$ such that $(L,L)${\color{6}$>2$}. Then the map $\psi:M\longrightarrow \mathbb{P}H^{0}(M,L)^*$ is a two sheeted ramified cover if $(L,L)=2$ and an emedding otherwise;  $M$ is a 2-sheeted covering of $\mathbb{C}P^{2}$ or a sextic (when $(L,L)^2=2$).
\end{thm}

\begin{proof}\leavevmode
\begin{enumerate}[label=\textbf{Step \arabic*}]
	\item \textbf{The case $(L,L)>0$, when we have an embedding.}\hspace{0.5em} Again, assume that $H^{0}(L)\neq 0$, so $L$ is ample, globally generated and with a general smooth $D\in |L|$. Looks like we computed $(L,L)=2$.

	\item  \textbf{The hyperelliptic case.}\hspace{0.5em} $(L,L)=2$. We use Kodaira-Nakano, so  $H^{1}(L)=0$, and $\chi(L)=2+\frac{(L,L)}{2}=3$, so $\dim H^{0}(L)\overset{?}{=}0$ and we get the map $\psi:M\to \mathbb{C}P^{2}$, a ramified covering.

	\item \textbf{It is a sextic.} \hspace{0.5em} $\psi$ is ramified in a sextic. We used $R\subset M$ ramification divisor to show
		\[K_M=\psi^* K_{\mathbb{C}P^{2}}\otimes \mathcal{O}(R)=\mathcal{O}(M)\]
		Then $\psi^* (\mathcal{O}(-3))=L^{\otimes-3}$ gives $L^{\otimes-3}\otimes \mathcal{O}(R)=\mathcal{O}_M$, $[R]=3c_1(L)$.
		
		$R_0=\pi(R)$, we get
		\[ [R_0] \cap [\text{hyperplane section}]=[R]\cap [D]=3(L,L)=6 \]
		by computing
		\[\int \pi_* x\wedge y=\int x\wedge \pi^*y\]
		where $x=[R]$ and  $y=[H]$.
\end{enumerate}

\begin{remark}[Check!]\leavevmode
	K3 ramified over $\mathbb{C}P^{2}$ and sextics are in correspondence.
\end{remark}

\begin{idea4}{Corollary 3}\leavevmode
	K3, $\operatorname{Pic}(M)= \left<L\right> $, $(L,L)>2$, then  $L$ or $L^*$ is very ample.
\end{idea4}

\begin{prop}\leavevmode
	K3 surface isomorphic to quartic if and only if $\operatorname{Pic}(M)$ contains a very ample bundle $L\in\operatorname{Pic}(M)$ with $(L,L)=4$. 
\end{prop}

\begin{proof}\leavevmode
	\begin{enumerate}[label=\textbf{Step \arabic*}]
		\item Let $\varphi:M\hookrightarrow \mathbb{C}P^{3}$ be the embedding and $L:=\phi^*(\mathcal{O}(1))$. We have
			\[(L,L)=\int_{M}c_1(L)\wedge c_1(L)=\int_{\mathbb{C}P^{3}}[M]\wedge [H]\wedge [H]=4\]
			\item $L$ very ample $(L,L)=4$, RR $h^{0}(L)=\chi(L)=2+\frac{(L,L)}{2}=4$
	\end{enumerate}
\end{proof}

\begin{idea6}{Corollary 4}\leavevmode
	$M$ K3 surface with $\operatorname{Pic}(M) =\mathbb{Z}$ and $L$ generating  $\operatorname{Pic}(M)$ and $(L,L)=4$. Then  $M$ is isomorphic to a quartic.
\end{idea6}
\end{proof}

\section{Class 11: Density of quartics deduced from Ratner theory}

\subsection{A result from XX century}

\begin{idea1}{Result}\leavevmode
	$M$ K3, there exists an integer vector $X\in H^{2}(M,\mathbb{Z})$ with zero intersection number, i.e., $(X,X)=0$.
\end{idea1}

This is nonelementary.

\subsection{Today}
Consider $R$, the set of all integer vectors  $X\in H^{2}(M,\mathbb{Z})$ with $(X,X)=4$,  $V(R)\subset \operatorname{Gr}_{++}$ sec of 2-planes $L \subset H^{2}(M,\mathbb{R})$. $L\perp x$, some $x\in R$. Then $V(R)$ is dense in $\operatorname{Gr}_{+ +}$

Remember that
\[\operatorname{Gr}_{++}(H^{2}(M,\mathbb{R}))=\text{positively oriented 2-planes in }H^{2}(M,\mathbb{R}) \]

Today we will give a proof and next lecture another. And you should find another. One proof is that Corollaries 3 and 4 imply it.

\subsection{Noether-Lefschetz locus in the period space}

\begin{defn}\leavevmode
	Let $D\in H^{2}(M,\mathbb{R})$, and consider $\operatorname{Teich}_\eta$ the set of all complex structures such that over $\eta\in H^i(M,I,\mathbb{R})$. Then
	 \[\mathbb{P}\operatorname{er}_\eta=\{W\in\operatorname{Gr}_{++}:W\perp \eta\}\]
	 is the \textit{\textbf{Noether-Lefschetz locus}}.
\end{defn}

\begin{remark}\leavevmode
	$\mathbb{P}\operatorname{er}_\eta$ is the intersection of $\mathbb{P}\operatorname{er}$ and a complex hyperplane $\mathbb{P}\eta^\perp \subset\mathbb{P}H^{2}(M,\mathbb{C})$.
\end{remark}

\begin{claim}\leavevmode
	$\operatorname{Teich}\longrightarrow \mathbb{P}\operatorname{er}_\eta$ is a local difeomorphism
\end{claim}

\begin{defn}[$\mathbb{P}\operatorname{er}\left( V \right) $]\leavevmode
	$V \subset H^{2}(M,\mathbb{R})$, $\mathbb{P}\operatorname{er}\cap \mathbb{P}(V\otimes C)=\mathbb{P}\operatorname{er}(V)$ complex analytic in $\mathbb{P}\operatorname{er}$ […] This is also called the \textit{\textbf{Noether-Lefschetz locus}}.
\end{defn}

\subsection{The set of quartics with Picard rank 1}

\begin{remark}\leavevmode
	$H^{11}(M,I)=\mathbb{P}\operatorname{er}(I)^\perp$
\end{remark}

\begin{defn}\leavevmode
	$\mathbb{P}\operatorname{er}^0_\eta=\{v\in\mathbb{P}\operatorname{er}:\operatorname{Pic}(M,\mathbb{Z}) =\mathbb{Z}\}$
\end{defn}

\begin{claim}\leavevmode
$\mathbb{P}\operatorname{er}^0_\eta$ is dense in $\mathbb{P}\operatorname{er}_\eta$, $\eta\in H^{2}(M,\mathbb{Z})$.
\end{claim}

\begin{proof}\leavevmode
	$\mathfrak{S}$ set of rank 2 subgroups in $H^{2}(M,\mathbb{Z})$ containing $\eta$ […See slides]
\end{proof}

\subsection{Set of all quartics is dense}

\begin{thm}\leavevmode
	\[\bigcup_{\eta|(\eta,\eta)=4} \operatorname{Teich}_\eta\text{ is dense in $\mathbb{P}\operatorname{er}$} \]
\end{thm}

\begin{proof}\leavevmode
	Later today
\end{proof}

\begin{idea4}{Corollary 2}\leavevmode
	The set $\mathfrak{D}$ of K3 with Picard group of rank 1 generated by vectors $x$ such that  $(x,x)=4$, then $\mathfrak{D}$ is dense in $\operatorname{Teich}$.
\end{idea4}

\begin{remark}\leavevmode
	By corollary 1 (?), all such $x$ correspond to quartics, therefore, corollary 2 implies that quartics are dense in $\operatorname{Teich}$.
\end{remark}

\begin{coro}\leavevmode
	Every K3 is diffeomorphic to a smooth quartic.
\end{coro}

\subsection{Ergodic measures}

\begin{defn}\leavevmode
	Let $(M,\mu)$ be a space with a measure and $G$ a group acting on $M$ preserving $\mu$. This action is \textit{\textbf{ergodic}} if all $G$-invariant measurable subsets $M' \subset M $ satisfy $\mu(M')=0$ or $\mu(M\setminus M')=0$.
\end{defn}

\begin{remark}\leavevmode
	Ergodic measures are extremal rays in the cone of all $G$-invairiant measures.
\end{remark}

\begin{remark}\leavevmode
	Any $G$-invariant measure on $M$ is expressed as an average of a certain set of ergodic measures (Choquet's theorem). Therefore, $G$-invariant ergodic measures always exist.
\end{remark}

\begin{claim}\leavevmode
	$M$ manifold, $\mu$ Lebesgue measure, $G$ a group acting on $(M,\mu)$ ergodically. The set of non-dense orbits has measure 0.
\end{claim}

\begin{proof}\leavevmode
	Done in class [see slides], should be simple.
\end{proof}

\subsection{Ratner theory (lattices)}

\begin{defn}\leavevmode
	Let $G$ be a connected Lie group with a Haar measure. A \textit{\textbf{lattice}} $\Gamma \subset G$ is a discrete subgroup of finite covolume, that is, $G/\Gamma$ has finite volume.
\end{defn}

\begin{example}\leavevmode
	By Borel and Harish-Chandra theorem, any integer lattice in a simple Lie group has finite covolume.
\end{example}

\begin{thm}[Moore]\leavevmode
	If you have $G/H$, $G$ simple, $G$ with finite center, $H$ is noncompact, $\Gamma$ is a lattice. Then $\Gamma$ acts on $G/H$ ergodically. That is, for all $\Gamma$-invariant measurable subsets $Z \subset G/H$, either $Z$ has measure zero or  $G\setminus H/Z$ has measure 0.
\end{thm} 

\begin{thm}[Ratner]\leavevmode	
$\Gamma=G_\mathbb{Z}$ integer lattice, and $H \subset G$ gnerated by unipotents.

$x\in G/H$, $\overline{\Gamma x}$ (closure of orbit) is an orbit of the smallest rational subgroup $U\subset S\subset G$. Then $S\cap \Gamma^x$ is a lattice in $S$.
\end{thm}

\subsection{Oppenheim conjecture}

\begin{defn}\leavevmode
	An \textit{\textbf{irrational}} quadratic form is  $q:V=Z_\mathbb{Z}\otimes_\mathbb{Z}\mathbb{R}\longrightarrow \mathbb{R}$ indefinite, non proportional to an integer.
\end{defn}

\begin{conjecture}[Oppenheimer 29', Margulis proved in 87, and there's another proof in early 90's] $q$ irrational quadratic form in $\mathbb{R}^{n}$, $S_q=q(\mathbb{Z})$, then $S_q$ is dense in $\mathbb{R}$.
\end{conjecture}

\begin{proof}\leavevmode
	We go to Ratner theorem.

	\begin{enumerate}[label=\textbf{Step \arabic*}]
		\item Consider $G=\mathsf{SL}(n,\mathbb{R})$, which has a lattice, and $\mathsf{SO}(a,b)$. Let $G/H$ be the set of all quadratic forms of sign $(a,b)$.

			[Content missing…]
	\end{enumerate}
\end{proof}

\subsection{An exercise and a theorem using Ratner theory}

\begin{exercise}[Classify intermediate subgroups]\leavevmode
	Let $G=\mathsf{SO}(a,b)$ and $H\subset G$ the stabilizer of a point in $W\in\operatorname{Gr}_{++}(\mathbb{R}^{a,b})$, then there is only one type of intermediate subgroups between $G$ and $H$.
\end{exercise}

Therefore, Ratner theorem implies

\begin{prop}\leavevmode
	$G=\mathsf{SO}(3,19)$, $H=\mathsf{SO}(1,19)=$ stabilizer of $W\in\operatorname{Gr}_{++}$. Then $H\subseteq H_1\subseteq G$ so $H_1$ is the stabilizar of a vector in $W$. $\mathsf{SO}(H^{2}(M,\mathbb{Z})) \cdot W$-dense in $W\cap H^{2}(M,\mathbb{Z})$ 

	\begin{idea4}{Need}\leavevmode
		$\mathsf{SO}(H^{2}(M,\mathbb{Z})) \cdot \mathbb{P}\operatorname{er}$ is dense in $\mathbb{P}\operatorname{er}$.
	\end{idea4}
	Any $W\in\mathbb{P}\operatorname{er}$ with $W \not\in$ (?)…
\end{prop}

\section{Class 12: density of quartics, a more elementary proof}

\subsection{The idea}
is that complex structures are dense. Specifically

\begin{idea5}{Theorem 2}\leavevmode
	The set of all vectors with $(x,x)=4$ in  $H^{2}(M,\mathbb{Z})$, called $R$. And the set of all 2-planes orthogonal to some $x\in R$, called $Z(R)\subset \operatorname{Gr}_{++}$. \textbf{Then $Z(R)$ is dense in  $\operatorname{Gr}_{++}$.} 
\end{idea5}

\subsection{Quadratic lattices}

\begin{defn}\leavevmode
	\textit{\textbf{Integer quadratic lattice}} is a lattice with integer-valued scalar product.

	The lattice  $\Lambda,q$ \textit{\textbf{represents}}  $k$ if there is $x\in\Lambda$ with $q(x)=k$.
\end{defn}

\begin{idea3}{Theorem 1}[About lattices]\leavevmode
	$(V_{\mathbb{Z}},\Lambda$ lattice … then $Z(\mathfrak{R})$ is dense
\end{idea3}

\begin{idea4}{Lemma 1}\leavevmode
	Taking $Z$ is compatible with closure: $Z(\bar{A})=\overline{Z(A)}$.
\end{idea4}

\subsection{The null quadric}
\begin{defn}\leavevmode
	$\operatorname{Null}$ is the quadric $q(x,x)=0$.
\end{defn}

\begin{remark}\leavevmode
	I think that for every positive plane there is a point in the null quadric orthogonal to it. In slides: $Z(\operatorname{Nu ll}=\operatorname{Gr}_{++}$.
\end{remark}

Our objective of today is reduced to

\begin{thm}[3]\leavevmode
	$\overline{\mathbb{P}R}\supset \operatorname{Nu ll}(V_{\mathbb{R}})$
\end{thm}

\begin{remark}[Homework hint]\leavevmode
	"They are the same sets"
\end{remark}

\subsection{Extending isometries of a lattice}

Before lattices were of the same rank, now they're different ranks.

\begin{idea4}{Corollary 2}\leavevmode
	One lattice inside another, then the isometris of the smaller that can be extended to the larger is of finite index I think the isometries of the smaller.
\end{idea4}

\section{Class 13: limit points of orbits of $\mathsf{SO}_{\mathbb{Z}}(p,q)$}

\subsection{Summary of last lecture}

Running assumtions
\begin{itemize}
\item $(V_{\mathbb{Z}},q)$ quadratic lattice, signature   $\geq (1,3)$
\end{itemize}

A great question for me is how to relate theorem 3 to theorem 1. Is is related to the orbits. But how does density follow?

\subsection{Quadratic form representing 0}
\
\begin{defn}\leavevmode
	 $(\Lambda,q)$, we say $q$ \textit{\textbf{represents}} $n$ if there is $x\in\Lambda$ such that $q(x,x)=n$.

	  $x$ is  \textit{\textbf{primitive}} if it is not divisible by a number (so product o number times vector is  $x$?)
\end{defn}

\begin{remark}\leavevmode
	$x$ is primitive iff $\Lambda/\left<x\right> $ is torsion free.
\end{remark}
\begin{proof}\leavevmode
	Very easy.
\end{proof}

\begin{remark}\leavevmode
	$x$ is primitive iff $\exists \eta\in\Lambda^*$ such that $\left<\eta,x\right> =1$.
\end{remark}

\begin{thm}[Meyer, 1888]\leavevmode
	Is you have an indefinite lattice of rank greater o equal to five, then $\Lambda$ represents 0.
\end{thm}

\begin{proof}[Idea of proof]\leavevmode
	Go to $p$-adic numbers. Legandre symbols, Hilbert symbols, Hasse principle. See \textit{Ueber eien Satz von Dirichlet} by A. Meyer. 
\end{proof}

\subsection{Quadratic forms representing 4}

We saw that
\begin{itemize}
\item The \textit{\textbf{hyperbolic lattice}}  $U_2$ represents 4. (So there is an alement nameley  $2x+y$ that  quadatic form evaluates 4.
\item Any unimodular even quadratic lattice that represents 0 contains $U_2$.
\end{itemize}
Which allows to prove easily that
\begin{thm}\leavevmode
	In every K3 there is a vector such that $(v,v)=4$. Whis means that the intersection lattice  $H^{2}(M,\mathbb{Z})$ represents 4.
\end{thm}
\begin{proof}\leavevmode
	Because we know that $H^{2}(M,\mathbb{Z})=3U_2\oplus 2E_{-8}$ which contains $U_2$. Also you may use Meyer to see $\Lambda$ is unimodular, so represents 0 so it contains $U_2$ (I think). Notice we had seen that this lattice is even using Riemann-Roch.
\end{proof}

\subsection{Reminder on discriminant of lattices}

\begin{prop}\leavevmode
	$\Lambda_1\subset \Lambda$ quadratic lattices of the same rank. Then $\mathsf{SO}(\Lambda_1)$ and $\mathsf{SO}(\Lambda)$ are commesurable.
\end{prop}

\begin{proof}\leavevmode
	We will use that there only finitely many choices of intermediate lattices between  $\Lambda_1$ and $\Lambda_1^*$ measured by the discriminant (claim 1).

	$\Gamma_2=\mathsf{SO}(\Lambda)\cap \mathsf{SO}(\Lambda_1)$. So $\Gamma_1$ acts on the set of intermediate groups. This makes it have finite index because of the stabilizer.

	Now let $\Gamma_3=\mathsf{SO}(\Lambda)$. Then we show that $\Gamma_2$ is finite index in $\Gamma_1$: take $N$ such that $N\Lambda\subset \Lambda_1$ so $\mathsf{SO}(N\Lambda)\cap \mathsf{SO}(\Lambda_1)$ is finite index $\mathsf{SO}(N\Lambda)=\mathsf{SO}(\Lambda)$. {\color{4}What is going on, the groups are isomorphic after multipliyng by a constant?}
\end{proof}

\begin{idea2}{Most important corollary in all this}\leavevmode
	$(A,q)$ nondegenerate quadratic lattice,  $B \supset A$ superlattice of the {\color{8}smaller} rank. denote $\Lambda_A \subset \mathsf{SO}(A)$ the group of isometries of $A$ that can be extended to $B$. {\color{4}Then $\Gamma_A$ is of finite index in $\mathsf{SO}(A)$}.
\end{idea2}

\begin{proof}\leavevmode
	Consider the lattice $B_1:=A \oplus A^\perp \subset B$ because (if you tensor it by $\mathbb{Q}$ ) it will have the same rank. So use previous corollary and $\mathsf{SO}(B_1)\cap \mathsf{SO}(B)$ is finite index in $\mathsf{SO}(B)$. This means every $\gamma\in\mathsf{SO}(A)$ can be extended to $\mathsf{SO}(B_1)$. Now look at stabilizers $\operatorname{S t }_A\mathsf{SO}(B) \cap \operatorname{S t}_A\mathsf{SO}(B_1)$ and project to $\mathsf{SO}(A)$ so the projection is of finite index since the others were already of finite index.
\end{proof}

{\color{8}\bfseries What is the point of this.}\hspace{.5em}That $\Lambda_2$ rank 2 of signature $(1,1)$ is a sublattice of  $\mathsf{SO}(\Lambda)$, so $\Lambda_2$ ??? acts? on $\mathsf{SO}(\Lambda_2)=\mathbb{Z}$.

\subsection{Pell's equation}

\begin{defn}\leavevmode
	An integer is called \textit{\textbf{square-free}} if it is not divisible by a square of some number not 1.
\end{defn}

\begin{remark}\leavevmode
	The unit sphere of $\mathbb{Z}\oplus \mathbb{Z}\sqrt{w} $ where the norm is  $N(a+b\sqrt{w} =a^2-wb^2$ is a multiplicative group. Solution of Pell equation is norm 1.
\end{remark}

\begin{thm}[Pell, Dirichlet second time]\leavevmode
	Let $w\in\mathbb{Z}_{> 0}$ that group is isomorphic to $\mathbb{Z}$ {\color{3}up to sign}.
\end{thm}

 \begin{remark}\leavevmode
	 Norm is a quadratic form in this ring $\mathcal{O}_K$. Solution of Pell is in $\mathsf{SO}(O_K,N)$
\end{remark}

\begin{idea8}{Pell did not solve Pell equation}\leavevmode
	This is a mistake by Euler. Pell barely translated the solution by /..?
\end{idea8}

\begin{proof}[Of Pell theorem version second time]\leavevmode
	It suffices to show that Pell equation has a non trivial solution.
\end{proof}

\begin{thm}[Lagrange]\leavevmode
	 Pell equation has a non-trivial solution
\end{thm}
 We need a lemma to prove this theorem.

\begin{lemma}\leavevmode
	There exists infinitely many solutions $x,y$ such that  $|x-\sqrt{w} y|<1/y$.
\end{lemma}

\begin{proof}[Proof of lemma]\leavevmode
	Partition the interval $[0,1[$ into  $m$ little intervals starting with $[0,1/m[$ and then  $[1/m,2/m[$ and so on. Then pigeon principle!! Because if  $a,b\in [0,m]$ then their fractional parts are in the same little interval.
\end{proof}

\begin{proof}[Proof of Lagrange]\leavevmode
	\begin{enumerate}[label=\textbf{Step \arabic*}]
		\item By the lemma, the equation $x^2+y^2w=M$ has infinitely many solutions. Because there is a certain value in a certain sequence that appears infinitely many times.

		\item Let $M$ be an integer for with  $x^2-wy^2=M$ has infinitely many solutions.  This means there exist two numbers $z_1,z_2\in\mathbb{Z}+\mathbb{Z}\sqrt{w} $ such that $z_1\equiv z_2\operatorname{mod}M$ and that $N(z_1)=N(z_2)$, {\color{6}why?} After more computations we see that $M=M\cdot N(z)$ and $N(z)=1$.  $z$ is  $z_3\sigma(z_2)+1$. {\color{7}what is $\sigma$?}
	\end{enumerate}
\end{proof}




\begin{thm}[Pell, Dirichlet first time]\leavevmode
	Let $w\in\mathbb{Z}_{> 0}$Integer solutions $\big(a,b\big)$ of $a^2-wb^2=1$ are isomorphic to $\mathbb{Z}$.
\end{thm}

\begin{proof}\leavevmode
	There should be an elementary proof but we can't remember.
\end{proof}

\begin{thm}[The main application of Pell equation]\leavevmode
	$\Lambda$ lattice of signature $(1,1)$ which does not represent 0. Then  $\mathsf{PSO}(\Lambda)=\mathbb{Z}$ (\textit{$\mathsf{PSO}$ is trivial}.
\end{thm}


\section{Class 14}


\end{document}
