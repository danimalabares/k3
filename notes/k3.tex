\input{/Users/daniel/github/config/preamble.sty}

\begin{document}

{\Huge k3}

\section{Class 1}

The most important invariant of a k3 surface is \href{https://en.wikipedia.org/wiki/Intersection_form_of_a_4-manifold}{intersection form}.

There are three classes of manifolds
\begin{enumerate}
	\item Smooth manifolds
\end{enumerate}

\[\begin{tikzcd}
	\text{smooth manifolds}\arrow[r,"\text{forgetful functor} "]&\text{PL manifold}\arrow[r]&\text{Topological manifolds}   
\end{tikzcd}\]

Donaldson: contiunally many non-equivalent smooth structures on $\mathbb{R}^{4}$. K3 surfaces has countably many smooth structures and only one of them is compatible with complex structure.

\begin{defn}
	Intersection form. Given a quadratic form on a lattice $V_{\mathbb{Z}}=\mathbb{Z}^n$, so 
	 \[q:V_{\mathbb{Z}}\times V_{\mathbb{Z}}\to \mathbb{Z}\]
	 is \textit{\textbf{unimodular}} if 
	 \[V_{\mathbb{Z}}\overset{q}{\longrightarrow} \operatorname{Hom}(V_{\mathbb{Z}},\mathbb{Z})\]
	 is an isomorphism.
\end{defn}

\begin{thm}[Universal coefficients formula]\leavevmode
	\[H_{n-1}(M,\mathbb{Z})=\mathbb{Z}^{b_{n-1}(M)}\oplus T_{n-1}(M)\]
	\[h^n(M,\mathbb{Z})=\mathbb{Z}^{b_{n}(M)}\oplus T_{n-1}(M)\]
\end{thm}

\begin{coro}
	$H^2(X,\mathbb{Z})$ is torsion free if $\pi_{1}(X) =0$ because 
\end{coro}

\begin{defn}
	\textit{\textbf{Signature}} is $m-n$ if $q$ has signature $(m,n)$.
\end{defn}

\begin{thm}[Rokhlm-Wu?]\leavevmode
	Signature is divisible by 16 for simply-connected (something else).
\end{thm}

\begin{remark}
	The methods used in surgery break down in smooth case because strange topological objects like infinite sums of spheres arise.
\end{remark}

\begin{thm}[Freedman, 1982]
	There are as many 4-manifolds as there are intersection forms. $M$ simply connected 4 manifold homotopy class is uniquely determined by intersection dorm. Moreover, for every unimodular form there exists a unique $M$ with this intersection form.
\end{thm}

\begin{thm}[Donaldson, 1986]\leavevmode
	$M$ smooth compact manifold with positive definite odd intersection form $q$. Then
	\[\begin{pmatrix} 1&0&0\\0&1&0\\0&0&0&1 \end{pmatrix} \]
\end{thm}

\begin{defn}
	Bilinear symmetric form is \textit{\textbf{indefinite}} if it is not positive definite nor negative definite.
\end{defn}

\begin{thm}[Classification of unimodular symmetric bilinear forms]\leavevmode
	Odd are diagonalizable, while even are related to special Lie group $E_8$.
\end{thm}

\begin{defn}
	A \textit{\textbf{K3 surface}} is a K\"ahler complex surface $M$ with $b_1=0$ (simply connected) and $c_1(M,\mathbb{Z})=0$.
\end{defn}

Kodaira did what Andr\'e Weil couldn'g classify.

 \begin{thm}
	K3 surfaces have trivial canonical bundle $K_{M}=\Lambda^2(\Omega^1M)$.
\end{thm}

\section{Class 2}

$G$ topological group. \textit{\textbf{Principal $G$ bundle}} is a space with free $G$-action such that the quotient $E/G$ is Housdorff. There are several conditions that make this work. And then you have $\operatorname{Homo t o py}(X,BG)=$equivalence classes of $G$-bundles.
Vector bundles of a manifold are the same as maps from $X$ to $B\operatorname{U}(n)$.

Vector bundles up to stable equivalence are classified basically by Chern classes, so by the cohomology in $H^\bullet(B\operatorname{U})=Q[c_1,c_2,\ldots,c_n$.

Now look at the loop space of $X$. Then $H^{\bullet}(\Omega X)$ is a free graded commutative algebra. Loop space has the interesting property that $\Omega \operatorname{U}=B\operatorname{U}$ and $\Omega B\operatorname{U}=\operatorname{U}$.

\subsection{Bialgebras}

Let $A$ be a superalgebra (graded with antisymmetric product). Then we ask the axiom of coassociativity and that .

\begin{example}
	$G$ group, and $C(G)$ the ring of  $k$-valued functions $C(G\times G)=C(G)\times C(G)$ so
	\begin{align*}
		G\times G &\longrightarrow G \\
		C(G) &\longmapsto C(G)\otimes C(G)
	\end{align*}
	
\end{example}

\subsection{H-spaces}

\begin{defn}
	$H$-space is a space $M$ with a map $\mu:M\times M\\to M$that is homotopy associative,
	\[\begin{tikzcd}
		M\times M\times M\arrow[r,"\mu\times \operatorname{id}"]\arrow[d,"\operatorname{id}\times \mu"]&M\times M\arrow[d,"\mu"]\\
		M\times M\arrow[r,"\mu"]&M
	\end{tikzcd}\]
	which is homotopy commutative. And with homotopy unit.
\end{defn}

So it's like a homotopy algebra?

\begin{example}
	The loop space.
\end{example}

\subsection{Bialgebras of finite type}

\begin{defn}
	A bialgebra $A$ is of \textit{\textbf{finite type}} if it is the direct sum of $A=\bigoplus_{i\geq 0} A^i $ supercommutative and each $A^1$ is finite dimensional.
\end{defn}

\begin{remark}
	Free commutative algebra is polynomial algebra
\end{remark}

\begin{defn}
	$A=\mathbb{C}[x_1,\ldots,x_n,\ldots]\otimes \Lambda^\bullet(a_1,\ldots,a_n,\ldots)$ is a graded commutative free algebra. In the slides: it is $\operatorname{Sym}_{\operatorname{gr}}V^*$ where $V^*$ is a graded vector space.
\end{defn}

\begin{thm}[Hopf]\leavevmode
	A graded commutative bialgebra of finite type over $k$ of 0 characteristic is free graded commutative as a $k$ algebra.
\end{thm}

\subsection{The cohomology algebra of U(n)}

\begin{claim}
	The cohomology algebra $H^{*}(\operatorname{U}(n),\mathbb{Q})$ is a free graded commutative algebra with generators in degrees $1,3,5,\ldots,2n-1$.
\end{claim}

\begin{proof}[Demostra\c c\~ao]
	Induction.  $\operatorname{U}(1)$ is clear because it is a circle. Then do Serre spectral sequence. Differentials vanish on the second page because there's only nonzero groups on even degrees! And we get that $E_2^{p_1}=H^{p}(S^{2n-1})\otimes H^{q}(\operatorname{U}(n-1))$. And then the sequence converges to that of the total space which is $\operatorname{U}(n)$.
\end{proof}

\subsection{Grassman manifolds}

\begin{defn}
	The \textit{\textbf{fundamental bundle}} $B_{\operatorname{fun}}$ is a rank $n$ vector bundle over $\operatorname{Gr}(n,m)$.
\end{defn}

\begin{claim}
	$B$, $B'$ vector bundles of rank $n$, $m-n$, $B\oplus B'$
	\[\varphi:X\to \operatorname{Gr}(m,n)\]
	\[\varphi (x)=B_x\subset B_x\oplus B_x'=\mathbb{K}^m\]
	then $B=\varphi^* B_{\operatorname{fun}}$.
\end{claim}

\begin{thm}\leavevmode
	If you have $B$ as a bundle on a manifold $X$ then $B\oplus B'$ is trivial for some bundle $B'$.
\end{thm}

\begin{proof}[Demostra\c c\~ao]
	Embed the total space in a large enough euclidean space.
\end{proof}

\begin{defn}
	$\operatorname{Gr}(n,\infty)=\operatorname{Gr}(n)$ is $\bigcup_{m=n_1}^\infty\operatorname{Gr}(n,m)=\operatorname{Gr}(n) $
\end{defn}

\begin{coro}
	For every bundle  $B$ of rank $n$ there is a function $\varphi:X\to \operatorname{Gr}(n)$ such that $B=\varphi^*B_{\operatorname{fun}}$.
\end{coro}

Take a bundle $E\to X$ and $G$ acts freely on $E$ so $E$ principal $G$ bundle. Classifying space  $BG$

\begin{thm}[Atiyah-Bott]\leavevmode
	Classifying space is unique up to homotopy equivalence.
\end{thm}

\subsubsection{The fundamental bundle}

In class 4 I finally understood that

\begin{defn}
	The \textit{\textbf{fundamental bundle}} on the Grassmanian  $\operatorname{Gr}(n)$ (the Grassmanian is this space where points are linear spaces) is the vector bundle such that the fiber of one point (which is a vector space) is the vector space that is the point. It's very tautological.
\end{defn}

\begin{thm}[Did we prove this?]\leavevmode
	Let $B$ be a vector bundle of rank $n$ on a cellular space $X$. Then there exists a continuous map $\varphi:X\to \operatorname{Gr}(n)$ such that $ B$ is isomorphic to the pullback $\varphi^*B_{\operatorname{fun}}$ of the fundamental bundle.
\end{thm}

\begin{remark}
	In fact $\operatorname{Gr}(n)$ is the classifying space of vector bundles of rank $n$, in the sense that isomorphism classes of vector bundles of maps $\varphi:X\to \operatorname{Gr}(n)$.
\end{remark}

\subsubsection{The canonical bundle}

When doing homework 3 I found this very nice on Hatcher, Vector bundles and K3:

\begin{defn}
	The \textit{\textbf{canonical bundle}} $p:E\to \mathbb{R}P^{n}$ has as its total space $E$ the subspace of $\mathbb{R}P^{n}\times \mathbb{R}^{n+1}$ consisting of pairs $(\ell,v$ with $v\in\ell$ and $p(\ell,v)=\ell$. There is also an infinite-dimensional projective space $\mathbb{R}P^{\infty}$ which is the union of the finite-dimensional projective spaces $\mathbb{R}P^{n}$ under the inclusions $\mathbb{R}P^{n}\hookrightarrow \mathbb{R}P^{n+1}$ coming from natural inclusions $\mathbb{R}^{n+1}\hookrightarrow \mathbb{R}^{n+2}$. The inclusions $\mathbb{R}P^{n}\hookrightarrow \mathbb{R}P^{n+1}$ induce corresponding inclusions of canonical line bundles, and the union of all these is a canonical line bundle over $\mathbb{R}P^{\infty}$.

	A natural generalization is the Grassmanian $\mathcal{Gr}(k,n)$ along with a canonical $k$-dimensional vector bundle over it consisting of pairs $(\ell,v)$ where $\ell$ is a poin the Grassmanian and  $v$ is a vector in $\ell$.
\end{defn}

\subsubsection{What this classification space should mean}

Remember that
\begin{defn}[Representable functor]
	Let $\mathcal{C}$ be a category. A functor $F:\mathcal{C}^{\operatorname{op}}\to \operatorname{Sets}$ is called \textit{\textbf{representable}} if there exists an object $B=B_{F}$ in $\mathcal{C}$ with the property that there is a \textit{natural} isomorphism of functors
	\[\varphi:\mathcal{C}(-,B_{F})\to F\]
	where $\mathcal{C}(-,B_{F})$ is the set of arrows from $-$ to $B_{F}$.

	One ussually expresses the naturality condition for a map $f:X\to Y$ with the following diagram:

	\[\begin{tikzcd}
		\mathcal{C}(X,B)\arrow[r,"\varphi_{X}"]\arrow[d,"f^{*}"]&F(X)\arrow[d,"f^{*}"]\\
		\mathcal{C}(Y,B)\arrow[r,"\varphi_{Y}"]&F(Y)
	\end{tikzcd}\]
\end{defn}

And in homotopy theory I have studied that

\begin{thm}[Brown representability theorem]
	Let $F$ be a contravariant functor from the homotopy category of parallel connected CW-complexes to pointed sets. If $F$ satisfies conditions (i) and (ii) above (for any pointed connected CW-complexes $X_{i}$, $A$, $B$, $C$), then $F$ is representable.
\end{thm}

\begin{remark}[So what is a classifying space?]
	The theorem says that there is a space $B=B_{F}$ (itself a pointed CW-complex) for which there is a natural isomorphism
	\[\varphi :[X,B_{F}]_{*}\overset{\cong }{\longrightarrow}F(X)\]
	for any pointed CW-complex $X$. This space $B_{F}$ is called a \textit{\textbf{classifying space}} for $F$. Recall also that when such $\varphi$ exists, it is completely determined by a \textit{generic} element $\gamma\in F(B_{F})$.

	The classifying space together with the genereic element is unique up to homotopy.
\end{remark}

\begin{remark}
	$H^{n}(-,G)$ is represented by  $K(G,n)$ together with a chosen element in $H^{n}(K(G,n),G)$
\end{remark}

But anyway. We see in \href{https://en.wikipedia.org/wiki/Classifying_space_for_U(n)#Case_of_line_bundles}{wiki} that for the case of homework 3 bundle  $S^1\to S^\infty\to \mathbb{C}P^\infty$ we get that the base space $BU(1)=\mathbb{C}P^\infty$, Thus, the set of isomorphism classes of circle bundles over a manifold $M$ are in one-to-one correspondence with the homotopy classes of maps from $M$ to $\mathbb{C}P^\infty$.

So what is the functor that we are representing? I think is K. Because the maps are isomorphic to  $K(S^1)$…? Circle bundles?

\subsection{Stiefel spaces}

\begin{defn}
	$\mathbb{K}^\infty$ is the direct limit of $\mathbb{K}^n$ so its just the direct sum $\bigoplus_{i=n}^\infty\mathbb{K} $. Stiefel space is the space of orthonormal $n$-frames.
\end{defn}

If we prove that Stiefel is contractible we obtain our classifying space so let's prove that. We have a fibration

\[\operatorname{U}(n)\hookrightarrow \operatorname{St}(n,\infty)\to \operatorname{Gr}(n,\infty)\]

\begin{thm}
	$\operatorname{St}(n)$ is contractible.
\end{thm}

\begin{proof}[Demostra\c c\~ao]
	\begin{enumerate}[label=\textbf{Step \arabic*}]
		\item Locally trivial fibration with contractible fiber and base $Y\to X$ then $Y$ is contractible, this is so trivial.

		\item Fibration $ \operatorname{St}(n)\to \operatorname{St}(n-1)$ with fiber $S^\infty$ 

		\item Show that $S^\infty$ is contractible.

		\item And then some map $\mathbb{R}$ that is not surjective, and construct homotopy of identity to a constant map.
	\end{enumerate}
\end{proof}

\begin{exercise}
	If $X_{\infty}=\bigcup X_{i} $ is the inductive limit of contractible cellular spaces then it is contractible. Use Whitehead theorem.
\end{exercise}

\begin{thm}[Important]\leavevmode
	$\operatorname{Gr}(\infty)=B\operatorname{U}$.
\end{thm}

\subsection{Stable equivalence}

\begin{defn}
	Vector bundles $V$,  $W$ are stable equivalent if  $V\oplus A\cong W\cong B$ for trivial vector bundles  $A$ and $B$.
\end{defn}

Homotopy classes of equivalent vector bundles are in coorespondance with…

\begin{thm}
	$B\operatorname{U}$ is $H$-space.
\end{thm}

\begin{coro}
	$H^* (B\operatorname{U},\mathbb{Q})$ is a free supercommutative algebra.
\end{coro}

\begin{claim}
	$H^*(B\operatorname{U}$ is a free polynomial algebra generated by classes $c_1,c_2,\ldots$ in all even degrees.
\end{claim}

\section{Class 3}

\subsection{Reminder}

\begin{defn}
	\textit{\textbf{Bialgebra}} is an algebra that is  equipped with comultiplication, counit…
\end{defn}

\begin{remark}
	It is when the dual space also has an algebra structure, but we prefer to use the tensor notation.
\end{remark}

Let $\sum_{i \geq 0}A^i$ with $\dim A^i<\infty$. \textit{\textbf{Free commutative algebra}} is a polynomial algebra.  \textit{\textbf{Free graded commutative algebra}} is
\[\widetilde{\operatorname{Sym}}^\bullet(W^\bullet \oplus V^\bullet):=\operatorname{Sym}^\bullet(W^\bullet)\otimes \Lambda^\bullet(V^\bullet)\]
where
\[W=\bigoplus_{i} W^{\operatorname{even}} \qquad V=\bigoplus_{i}V^{\operatorname{odd}}. \]

\subsection{Hopf algebra}

\begin{defn}
	A bialgebra is a \textit{\textbf{Hopf algebra}} when it is also equipped with an antipode map ($S$) such that the following diagram commutes
\[\begin{tikzcd}
	&H\otimes H\arrow[rr,"S\otimes \operatorname{id}"]&&H\otimes H\arrow[rd,"m"]\\
	H\arrow[ur,"\Delta"]\arrow[dr,swap,"\Delta"]\arrow[rr,"\eta"]&&\mathbb{C}\arrow[rr,"u"]&&H\\
								     &H\otimes H\arrow[rr,"\operatorname{id}\otimes S"]&&H\otimes H\arrow[ur]
\end{tikzcd}\]
[diagram from quantum group minicourse notes]

	\end{defn}

\begin{example}
	The cohomology of the loop space, $H^{\bullet}(\Omega X)$.
\end{example}

\subsection{Primitive elements in a bialgebra}

\begin{defn}
	An element of a bialgebra $x\in A$ is \textit{\textbf{primitive}} if $\Delta(x)=x\otimes 1+1\otimes x$.
\end{defn}

\begin{align*}
	\Delta(xy)&=\Delta(x)\Delta(y)\\
	&=(1\otimes x+x\cdot 1)(y\otimes 1+y\otimes y)\\
	&=1\otimes xy+xy\otimes 1+x\otimes y+y\otimes x.
\end{align*}

\begin{remark}
	We trying to show that Hopf algebras? bialgebras? are generated by primitive elements?
\end{remark}

\begin{defn}
	$A^\bullet$ bialgebra, $\mathcal{P}^\bullet\subset A^\bullet$ space of primitive, and the natural embedding
	\[\operatorname{Sym}_{gr}(\mathcal{P}^\bullet)\to A\]
	We say that $A$ is \textit{\textbf{free up to defree $k$}} if
	\[\bigoplus_{i\leq k} \operatorname{Sym}_{\operatorname{gr}}^i(P)\overset{\psi}{\longrightarrow} A\]
	is an embedding.
\end{defn}

\begin{lemma}
	Let $A^\bullet$ be a bialgebra which is free up to degree $k$. Then $A^\bullet$ is free up to degree $k+1$.
\end{lemma}

\begin{proof}\leavevmode 
	\begin{enumerate}[label=\textbf{Step \arabic*}]
		\item Choose a basis of $P$, $\{x_i\}$. Chose a polynomial condition $Q(x_1,\ldots,x_n)=0$ of degree $k+1$. Write this as
			\[Q=Q_mx_1^m+Q_{m-1}x_1^{m-1}+\ldots+Q_0.\]
			that is
			\[Q=\sum_{i=0}^mQ_ix_1^i\]
			with $Q_i$ invariant somehow. Then we apply comutiplication to obtain
			\[\Delta(Q)=Q\otimes 1+1\otimes Q+R\]
			where $R$ is some sort of reminder with bounded degree:
			\[R\in\mathfrak{U}:=\bigoplus_{i\leq k} \operatorname{Sym}^i_{\operatorname{gr}}(P)\otimes \bigoplus_{i\leq k} \operatorname{Sym}_{\operatorname{gr}}^i(P)  \]
			which follows from a similar computation of that which we did after defining primitive elements.

			\item Project to drop the terms that have $Q\otimes 1+1\otimes Q$:
				\[\Pi:\mathfrak{U}\to x_1\otimes \bigoplus_{i\leq k}   \operatorname{Sym}^i_{\operatorname{gr}}(P)\]
				since the $x_i$ are primitive, i.e. $\Delta(x_i)=x_i\otimes 1+1\otimes x_i$, one has
				\[\Delta(x_1^m)=(x_1\otimes 1+1 \otimes x_1)^m\]
				we get that
				\[\Pi(\Delta(x_1^m))=mx_1\otimes x_1^{m-1}\]
				while on the board it is written that
				\[\Pi(\Delta(x_1^m))=\Pi((x_1\otimes 1+1\otimes x_1)^m)\]

			\item Let $\Pi(R):=x_1\otimes R_0$. Since $Q=0$ in $A$, its component $R_0$ is also equal to 0. So $\Pi(\Delta(Q))=0$. Then
				\begin{align*}
					0&=\Pi \left( \Delta \left( \sum_{m}x_1^m\cdot Q_m \right)  \right)\\
					 & =\sum_{m}x_1\otimes x_1^{m-1}Q_m+\Pi(mx_1\otimes x_1^{m-1}\cdot  \Delta(Q_m))\\
					& =\sum_{m}x_1\otimes x_1^{m-1}Q_m
				\end{align*}
				so that 
				\begin{align*}
					x_1\otimes x_1^{m-1}Q_m&=0\\
					\implies x_1^{m-1}Q_m&=0
				\end{align*}
				So we conclude that
				\[Q_m=0\]
		\end{enumerate}
	\end{proof}

\begin{remark}
	We just proved that for any subalgebra generated by finite elements, we didn't use that it is free.
\end{remark}

\subsection{Algebras with filtration}

\begin{defn}
	A \textit{\textbf{filtration on algebra}} is
	\[A^\bullet  \supset  F_1A^\bullet\supset F_2A^\bullet\supset\ldots\]
	such that
	 \[F_i A^\bullet F_j\subset F_{i+j}A^\bullet\]
\end{defn}

\begin{defn}
	\textit{\textbf{Associated graded}} to a filtered algebra is
\[A^\bullet_{\operatorname{gr}}=\bigoplus_{i=0}^\infty \dfrac{F^1A^\bullet}{F^{i+1}A^\bullet} \]
\[F^0A^\bullet=A^\bullet\]
\end{defn}

\begin{defn}
	$I\subset A$ ideal then \textit{\textbf{$I$-adic filtration}} is the filtration by the degrees of the ideal
	\[A\supset I\supset  I^2 \supset  I^3\ldots\]
\end{defn}

\begin{lemma}
	Choose an $I$-adic filtration. Then $A_{\operatorname{gr}}$ is generated by its first and second graded components $A/I\oplus I/I^2$.
\end{lemma}

\begin{proof}[Demostra\c c\~ao]
	Indeed, $I^k/^{k+1}$ is generated by products of $k$ elements in $(I/I^2)$.
\end{proof}

\begin{defn}
	A \textit{\textbf{augmentation ideal}} in a bialgebra is the kernel of the counit homomorphism $\varepsilon:A\to k$. We denote it by $Z=\ker A$
\end{defn}

\begin{remark}
\[\Delta(x)=1\otimes x+x\otimes 1 \operatorname{mod}Z\otimes Z\]
Why? Because
\begin{align*}
	x&=\varepsilon \otimes \operatorname{id}(\Delta(x))\qquad \text{up to $Z\otimes A$}\\
	\Delta(x)& =1\otimes x\qquad \text{up to $A\otimes X$} \\
	\Delta(x)&=x\otimes 1
\end{align*}
\end{remark}

Ok, now we can prove Hopf theorem.

\begin{thm}[Hopf theorem]\leavevmode
	A finite type bialgebra is generated by primitive elements.

	In slides: Let $A$ be a graded bialgebra of finite type over a field $k$ of characteristic 0. Then $A$ is a free graded commutative $k$-algebra.
\end{thm}

\begin{proof}\leavevmode 
	\begin{enumerate}[label=\textbf{Step \arabic*}]
		\item I think this is the computation above.

		\item $A_{\operatorname{gr}}$ is a bialgebra.

		\item $A_{\operatorname{gr}}$ is multiplicative generated by $Z^1/Z^2$. All elements $Z^1/Z_2$ are primitive, so this algrebra $A_{\operatorname{gr}}$ is generated by primitive elements.

		\item Let $\{x_i\}$ be a basis of primitive elements of $A_{\operatorname{gr}}$. Then lifts of $A$ have no relations because $A_{\operatorname{gr}}$ is already generated by primitive elements. Then there are no relations also for elements in $A^\bullet$ (I think).
	\end{enumerate}
\end{proof}

\subsection{Grassmanians (Reminder)}

$B$ vector bundle of rank $n$ on $X$ then there exists a map (essentialy unique) $\varphi:X\to \operatorname{Gr}(n)$ such that
\[\varphi^* (B_{\operatorname{fun}}=B\]
which makes the Grassmanian a classifying space, and $\operatorname{Gr}(1)=B\operatorname{U}(n)$.

The infinite Grassmanian is important.

\subsection{BU as an H-space (Reminder)}

Bott periodicity identifies the space of loops on $\operatorname{U}$ and $B\operatorname{U}$.

\begin{prop}
	Embed $\mathbb{C}^{\infty}\times \mathbb{C}^\infty$ into $\mathbb{C}^\infty$ taking the basis vectors of the first copy to the even basis vectors and the basis of the second copy to the odd. Then for $L_1\subset \mathbb{C}^\infty$, $L_2\subset \mathbb{C}^\infty$, the map
	\[L,L'\mapsto S(L,L')\]
	defines a structure of $H$-space on the infinite Grassmanian $B\operatorname{U}$.
\end{prop}

\begin{proof}
	Just show that $H$-associatity up to homotopy.
\end{proof}

\begin{coro}
	$H^\bullet(B\operatorname{U},\mathbb{Q})$ is a free supercommutative algebra.
\end{coro}
\begin{proof}
	Follows from Hopf theorem.
\end{proof}

\subsection{Cohomology of BU}

\begin{claim}
	$H^\bullet(B\operatorname{U},\mathbb{Q})$ is a free polynomial algebra generated by classes $c_1,c_2,\ldots$ in all even degrees.
\end{claim}

\begin{proof}[Demostra\c c\~ao]
	Leray-Serre spectral sequence.
\end{proof}

\subsection{Chern classes: axiomatic definition}

\begin{defn}
	\textit{\textbf{Chern classes}} are classes  $c_i(B)\in H^{2i}(X)$ for $i=0,1,2,\ldots$

	\textit{\textbf{Chern classes}} are  $c_i(B)\in H^{2i}(X)$ for a complex vector bundle $B$ over $X$ with axioms

	\begin{enumerate}[label=\alph*.]
		\item $c_0(B)=1$

		\item Functoriality (commutes with bullbacks): for $\varphi:X\to Y$ with $B$ bundle on $Y$, 
			 \[\varphi^*(c_i(B))=c_1(\varphi^*(B))\]

		\item Define \textit{\textbf{total Chern class}}  $c_{*}:=\sum_{i}c_i(B)$ then
			\[c_i(B)\cdot c_i(B')=c_*(B\oplus B')\qquad \text{(Whitney)} \]

		\item $\mathcal{O}(1)$ on $\mathbb{C}P^{n}$,
			\[c_i(\mathcal{O}(1)=1+[H]\]
			where $[H]$ is the fundamental class of a hyperplane section.
	\end{enumerate}
\end{defn}

\begin{remark}[Once and for all]
	$\mathcal{O}(-1)$ is the \textit{\textbf{tautological line bundle}} on a Grassmanian, defined as  $\{(\ell,v)\in\mathcal{Gr}(k,n)\times \mathbb{C}^n:v\in\ell\}$.

	$\mathcal{O}(1)$ is the \textit{\textbf{hyperplane bundle}} which is the dual of that so $\{(\ell,v^* )\in\mathcal{Gr}(k,n)\times (\mathbb{C}^n)^*:v^*\in\ell^*\}$
\end{remark}

Suppose we have a class $a\in H^{\bullet}(B\operatorname{U})$. Then for all $B$ on $X$ 
\[\varphi:X\to B\operatorname{U}\]
so
\[B\cong \varphi^* (B_{\operatorname{fun}})\]
and so
\[\varphi^*_B(c_*)=c_*(B).\]





\section{Class 4}

\subsection{Reminder}

For each rank $n$ bundle $B$ on $X$ there exists $\varphi_B:X\to \operatorname{Gr}(n,\infty)=B\operatorname{U}(n)$ such that $\varphi^* _B(B_{\operatorname{f un}}=B$.

The infinite grassmanian is classifying space for (?) stable bundles.

Some more review about $H$-space structure, primitive elements, a comment on last exercise of homework 2.

Chern classes of $\mathcal{O}(1)$ are hyperplane sections: $c_i(\mathcal{O}(1))=1+[H]$.

\subsection{The splitting principle}

\begin{exercise}
	Prove that $BU(1)=\mathbb{C}P^\infty$.
\end{exercise}

\begin{proof}[Solution]\leavevmode
	Hopf fibration on $S^\infty$? It's easier, take $n=1$, it's just by  definition.
\end{proof}

\begin{defn}
	The \textit{\textbf{fundamental bundle}} on $BU(1)^n$ has fiber
	\[\ell_1\oplus \ell_2\oplus \ldots\ell_n\]
	where $\ell_i\in BU(1)$ are product $\ell_1\times \ell_2\times \ldots\times \ell_n$.
\end{defn}

\begin{remark}
	Chern classes of $B_{\operatorname{f un}}$ are uniquely determined by axioms, because every factor has Chern classes, and fibers are just sums, and pullbacks preserve sums…
	\[B_{\operatorname{f un}}=\bigoplus_{i} \pi_i \mathcal{O}(1) \]
	where
	\[pi_i:BU(1)^n\to BU(1)\]
	is a projection.
\end{remark}

\begin{remark}
	$H^{\bullet}(BU(1))^n=\mathbb{Z}[z_1,\ldots,zn$ 
	Here at least I remember that the cohomology of $\mathbb{C}P^{\infty}$ is just polynomials so it looks reasonable that the $n$-th power is polynomials in more cariables.
\end{remark}

\begin{thm}[Splitting principle]\leavevmode
	Let $\varphi_{\operatorname{f un}}:BU(1)^n\to BU$, the \textit{\textbf{fundamental map}}, it induces embedding on cohomology up to degree $2n$. For all primer generator $\sigma_i\in H^{2}(BU)$, $\varphi_{\operatorname{f un}}(\sigma_1)=\lambda \sum_{i}z_i^k$ with $\lambda\neq 0$.

	So
	\[\begin{tikzcd}
		BU(1)^n \arrow[r]& BU\\
		X\arrow[u]\arrow[ur]
	\end{tikzcd}\]
	
\end{thm}

\begin{remark}
	\href{https://en.wikipedia.org/wiki/Classifying_space_for_U(n)}{Wiki} Thus, the set of isomorphism classes of circle bundles over a manifold $M$ are in one-to-one correspondence with the homotopy classes of maps from $M$ to $\mathbb{C}P^{\infty}$
\end{remark}

\begin{thm}
	Chern classes are unique (uniquely determined by axioms).
\end{thm}

\begin{proof}\leavevmode 
	\begin{enumerate}[label=\textbf{Step \arabic*}]
		\item Every bundle is obtained as pullback of the fundamental bundle. So for $A\in H^{\bullet}(BU)$ and $B$ bundle on $X$, $A(B)=\varphi^*_{B}(A)\subset H^{\bullet}(X)$ so $c_i(B)$ are obtained as pullbacks of $ c$ in the fundamental bundle.

		\item \[BU(1)^\infty \overset{\varphi_{\operatorname{f un}}}{\longrightarrow}BU\]
			pullback of fundamental bundle is fundamental. (This map is defined from the former by induction).
			\[\varphi^*_{\operatorname{fu n}}(c_i(B_{\operatorname{f un}})=c_i(B_{\operatorname{f un}}\text{ on $BU$} )\]
			The Chern classes of the fundamental bundle are already known. Since $\varphi^*_{\operatorname{f un}}$ is injective by the splitting principle we are done.
	\end{enumerate}
\end{proof}

\subsection{Primitive generators of $H^{*}(BU)$}

Recall the $H$-space multiplication:
\begin{align*}
	BU\times BU  &\longrightarrow BU \\
	L_1\times L_2 &\longmapsto L_1\oplus L_2
\end{align*}
and the comultiplication
\[\Delta :H^{\bullet}(BU)\to H^{\bullet}(BU)\]
Generators of $H^{\bullet}(BU)$ are $c_{h_1},c_{h_2},\ldots$ with $c_{h_i}\in H^{2i}(BU)$ and we have the comultiplication $\Delta(c_{h_i})=c_{h_i}\otimes 1+1\otimes c_{h_i}$.

\begin{remark}
	\[\varphi=(\varphi_1,\varphi_2):X\to BU\times BU\]
	and we can compose so we have
	\[\varphi\circ \mu:X\to BU\]
	what does this map do?
	\begin{align*}
		\varphi\circ \mu: X &\longrightarrow BU \\
		\varphi^* (B_{\operatorname{fun}} &\longmapsto B_1\\
		(\varphi\circ \mu)^* (B_{\operatorname{ fu n}})&=B_1\oplus B_2
	\end{align*}
	So then we have
	\begin{align*}
		\varphi^*&:H^{\bullet}(BU)\otimes H^{\bullet}(BU)\to H^{\bullet}(X)\\
		\Delta &:H^{\bullet}(BU)\to H^{\bullet}(BU)\otimes H^{\bullet}(BU)\\
		\Delta \circ \varphi^*&:H^{\bullet}(BU)\to H^{\bullet}(X)
	\end{align*}
\end{remark}

\begin{coro}
		For every $x \in H^{\bullet}(BU)$
		\[X(B_1\oplus B_2)=\Delta(x)(B_1,B_2)\]
	\end{coro}

	\begin{coro}
		If $x\in H^{*}(BU)$ is primitive, then $x(B_1\oplus B_2)=x(B_1)\oplus X(B_2)$.
	\end{coro}

	\begin{proof}
		$\Delta(x)=x\otimes 1+1\otimes x$ so $\Delta(x)$ evaluated on $(B_1,B_2)$
	\end{proof}

	\begin{remark}
	We will construct the full Chern class $c_*(B)$ as a pullback of a class $C\in H^{*}(BU)$.
\end{remark}

\begin{remark}
Then take exponential. Let $\chi_i\in H^{2i}(BU)$ be a primitive generator. Use Hopf theorem to see that it is unique by a constant. Snce $\chi_i(B_1\oplus B_2)=\chi_i(B_1)+\chi_i(B_2)$, the class $C=e^{\sum_{i}a_i\chi_i}=1+\ldots+\frac{\chi_n}{n!}+\ldots$ satisfies the Whitney formula.

To construct Chern classes satisfying the axioms it remains to arrange the coefficients $a_i$ in such a way that $C(\mathcal{O}(1))=1+[H]$ I think this means hyperplane section.
\end{remark}

\begin{lemma}
An embedding
	\[BU(1)\overset{\varphi}{\hookrightarrow}BU\]
	with $\chi_i\in H^{2i}(BU)$ primitive generator. Then $\varphi^*(\chi_i)\neq 0$
\end{lemma}

\begin{proof}
	$H^{\bullet}(BU)=$ symmetric polynomials in $H^{i}(BU(1))^n$, $\varphi_{\operatorname{fun}}(x_N)=x\sum_{i=1}^nz_i^k$ so $\varphi (x_k)=\lambda x_1^k$.
\end{proof}

\begin{remark}
	$\varphi^* (c_i(B_{\operatorname{fun}})=c_i(\mathcal{O}(1)=1+[H]$
\end{remark}

\begin{thm}
	Choose generators $\chi_i\in H^{2}(BU)$ primitive. Then $\varphi^* (\sum_{i}\chi_i=\log(1+[H])$ where the logarith is a formal power series, namely $\sum_{i=1}^\infty\frac{H^n}{n!}(-1)^n$.

	That means $c(B_{\operatorname{fun}})=\operatorname{exp}\left(\sum_{\chi_i}\right)$.
\end{thm}
	
	
\section{Class 5}

We want to study the space of line bundles on a surface.

\subsection{K-group}

\begin{defn}
	Let $V$ be the set of equivalence classes of vector bundles on $X$. Consider the free module generated by $V$ (it's just $V$ copies of $Z$):
	\[\mathbb{Z}\left<V\right> =\bigoplus_{V} \mathbb{Z} \]
	And now consider
	\[\frac{\mathbb{Z}\left<V\right> }{[F_1]-[F_1]-[F_3]}\]
	for all exact sequences
	\[\begin{tikzcd}
		0\arrow[r]&F_1\arrow[r]&F_2\arrow[r]&F_3\arrow[r]&0
	\end{tikzcd}\]
	Equivalently, the relation is $[F_1]+[F_3]=[F_2]$.
\end{defn}

\begin{remark}
	We may give an $H$-structure to the set of homotopy classes of maps $X\to BU$ as follows
	$\varphi_1,\varphi_2:X\to B\operatorname{U}$ 
	\[B_1=\varphi^* (B_{\operatorname{fun}})\]
	define the $H$-product
	\[\varphi:=\varphi_1\circ \varphi_2\]
	such that
	\[\varphi^* (B_{\operatorname{fun}}=B_1\oplus B_2\]
\end{remark}

And then we have an isomorphism (that we are not going to use):
\[K\left(X \right) \overset{\operatorname{hom}}{\longrightarrow}\text{group of homotopy classes of maps from $X$ to $BU$} \]
This is because every bundle on $X$ is the pullback of the fundamental bundle by some map. We need to check that the image of trivial bundle is trivial map (homotopic to constant?) and that it preserves the product.

\begin{remark}
	The important thing of today is that that sum corresponds to addition
\end{remark}

\begin{remark}
	I guess I should first understand how is it that every bundle is the pullback of the fundamental bundle.
\end{remark}

So for example for injectivity we need to show that if a map $\varphi$ pulls back the fundamental bundle to the trivial bundle then $\varphi$ is homotopic to identity. This is not obvious though.

The point is that that map is a bijection.

\begin{claim}
	Chern classes are defined on $K(X)$ and satisfy Whitney formula (meaning Chern classes they pass to the quotient, right?)
\end{claim}

\begin{proof}
	Let $B$ be a bundle on $X$ so that $B=\varphi^* (B_{\operatorname{fun}}$. We showed last time that there is a $c_{\cdot}\in H^{0}(BU)$ such that $c_{\cdot}(B)=\varphi^* (c_{\cdot}$. In fact we proved that $c_{\cdot}= \operatorname{exp}(\text{additive} )$, but its actually Chern character, $c_{\cdot}=\operatorname{exp}(\operatorname{Ch}_{\cdot})$, in fact $\operatorname{Ch}_{\cdot}(B_1+B_2\operatorname{Ch}(B_1)+\operatorname{Ch}_{\cdot}(B_2)$.
\end{proof}

\subsection{Coherent sheaves}

\begin{defn}
	Let $M$ be a complex manifold and $\mathcal{O}_{M}$ its structure sheaf (of holomorphic functions). A \textit{\textbf{coherent sheaf}} is a sheaf of $\mathcal{O}_{M}$-modules, locally isomorphic to a quotient of a free sheaf $\mathcal{O}^n_{M}$ by a finitely generated  $\mathcal{O}_{M}$-invariant subsheaf.

	A \textit{\textbf{coherent sheaf}} on a projective manifold. A \textit{\textbf{projective manifold}} is  $\operatorname{Proj}(A^\bullet) $ where $A^\bullet$ is a graded ring. \textit{\textbf{Coherent sheafes}} are sheaves of graded $A^\bullet$-modules.
\end{defn}

\begin{exercise}
	Let $M$ be a projective manifold. Prove that any coherent sheaf $F$ has a (projective) resolution
	\[\begin{tikzcd}[column sep=small]
		0\arrow[r]&B_{n}\arrow[r]&B_{n-1}\arrow[r]&\cdots \arrow[r]&B_0\arrow[r]&F\arrow[r]&0
	\end{tikzcd}\]
	where $B_i$ are vector bundles. This is called the \textit{\textbf{syzygy resolution}}
\end{exercise}

\begin{proof}[Solution]\leavevmode
	Every module has a projective resolution called \textit{\textbf{Koszul resolution}}. So what is Koszul resolution. First you have a resolution of a maximal ideal. For a maximal ideal it is clear since … (Herieta? and) Eisenbud or even Bourbaki Homological algebra.
\end{proof}

\subsection{Coherent sheaves and their Chern classes}


So there's actually two K-groups. One is generated by bundles and the other by sheaves. For bundles, it is an algebra. For sheaves, it is a module over the other one. For Groethendick one was $K^\bullet$ and the other $K_{\bullet}$ but we don't know which is which.

\begin{remark}
	After this is done, it's possible to prove that the $K$-group of coherent sheaves on a projective manifold is equal to the K-group generated by holomorphic vector bundles.
\end{remark}

\begin{defn}
	The \textit{\textbf{Chern class}} of a coherent sheaf is the Chern class of the corresponding element in the K-group.
\end{defn}

\begin{remark}[about singularities, see slides]	
Suppose we do resolution of a manifold and pullback a bundle
\[\begin{tikzcd}
\tilde{M}\arrow[d,"\pi"]\\
M
\end{tikzcd}\qquad \begin{tikzcd}
\pi^*F\arrow[d]\\
F
\end{tikzcd}\]
\end{remark}

\subsection{Euler characteristic of a coherent sheaf}

\begin{defn}
	Let $F$ be a coherent sheaf. Its \textit{\textbf{Euler characteristic}} is
	\[\chi(F)=\sum_{i}(-1)^i \dim H^{i}(F)\]
	{\color{magenta}But what is that cohomology? What is the space?}
\end{defn}

\begin{claim}
	For any exact sequence
	\[\begin{tikzcd}
		0\arrow[r]&F_1\arrow[r]&F_2\arrow[r]&F_3\arrow[r]&0
	\end{tikzcd}\]
	we have
	\[\chi(F_2)=\chi(F_1)+\chi(F_3)\]
\end{claim}

\begin{proof}
	Should be possible…
\end{proof}

Then
\[\chi:K(M)\to \mathbb{Z}\]
is a homomorphism.

\subsection{Chern character}


OK so last class we defined an homomorphism called $\chi$ that was additive. Now let's call it
\[\operatorname{c}_{\cdot}=\operatorname{exp}(\operatorname{Ch}_{\cdot})\]
and it was additive
\[\operatorname{Ch}_{\cdot}(B_1\oplus B_2)=\operatorname{Ch}_{\cdot}(B_1)+\operatorname{Ch}_{\cdot}(B_2)\]

So the textbook definition is that \textit{\textbf{Chern character}} on line bundles is
 \[\operatorname{exp}(c_{\bullet}(L))\]
 So $c_1$ is additive and if you pass to the exponent it will be multiplicative:
 \begin{align*}
 	c_1(L_1\otimes L_2)&=c_1(L_1)+c_1(L_2)\\
	\operatorname{Ch}_{\cdot}(L_2\otimes L_2)&=\operatorname{Ch}_{\cdot}(L_1)\cdot \operatorname{Ch}_{\cdot}(L_2)
 \end{align*}

\subsection{Riemann-Roch-Hirzebruch theorem}

\begin{thm}[RRH]\leavevmode
	Let $F$ be a coherent sheaf on a complex compact manifold $M$. Then $\chi(F)$ can be expressed through Chern classes of $F$ and $M$ as follows:
	\[\chi(F)=\int_{X}\operatorname{Ch}_{\cdot}(F)\wedge \operatorname{Td}_{\cdot}(TM),\]
	where $\operatorname{Td}_{\cdot}(TM)$ mdenotes the \textit{\textbf{total Todd class of the tangent bundle $TM$}}, which is a sum of Chern classes.
	\[\operatorname{Td}_{\cdot}=1+\frac{c_1}{2}+\frac{c_1^2+c_1}{12}+\frac{c_1c_2}{24}+\frac{-c_1^4+4c_1^2c_2+c_1c_3+3c_2^2-c_4}{720}+\ldots\]
\end{thm}

\subsection{K-group for complex curves}

\begin{lemma}
	$K$-group for complex curves is generated by line bundles.
\end{lemma}

\begin{proof}\leavevmode 

	\begin{enumerate}[label=\textbf{Step \arabic*}]
		\item For each $F$ coherent sheaf, $L^n\otimes F$ has a section. So there is a monomorphism $L^{-N}\hookrightarrow F$.

		\item The consider the localization to produce a short exact sequence
			\[\begin{tikzcd}
				0\arrow[r]&F_1\arrow[r]&F_2\arrow[r]&F_3\arrow[r]&0
			\end{tikzcd}\]
			since $F=\bigoplus_{i} F_i $ for $F_i=\mathcal{O}_{M}/\mathfrak{m}_{X}^{a_i}$ so
			\[\begin{tikzcd}
				0\arrow[r]&(\mathfrak{m}_{X}^{a_1}\arrow[r]&\mathcal{O}_{X}\arrow[r]&F_1\arrow[r]&0
			\end{tikzcd}\]
	\end{enumerate}
\end{proof}

\subsection{Riemann-Roch for complex curves}

\begin{thm}[Riemann-Roch for complex curves]\leavevmode
	Let $F$ be a coherent sheaf on a compact complex curve of genus $g$. Then
	\[\chi(F)=c_1(F)+\operatorname{rk}(F)(1-g)\]
\end{thm}

\begin{proof}\leavevmode 
 We want to see
			\[c_1(L)=\operatorname{deg}(L)\]

	\begin{enumerate}[label=\textbf{Step \arabic*}]
		\item It suffices to prove for line bundles by the lemma.

		\item For degree 0 its easy beacuse $c_1(k_{x})=1$. For structure sheaf $\mathcal{O}_X$ we have rank is 1.

		\item Now let $ L$ be a line bundle. We have
			\[\begin{tikzcd}
				0\arrow[r]&\mathcal{O}_M\arrow[r]&L\arrow[r]&F\arrow[r]&0
			\end{tikzcd}\]
			\[\begin{tikzcd}
				0\arrow[r]&L\arrow[r]&\mathcal{O}_M\arrow[r]&F\otimes L=F\arrow[r]&0
			\end{tikzcd}\]
			\[\begin{tikzcd}
				0\arrow[r]&\mathcal{O}_{M}\arrow[r]&L_1^N\otimes L\arrow[r]&F\arrow[r]&0
			\end{tikzcd}\]
			\[\begin{tikzcd}
				0\arrow[r]&L_1^{-N}\arrow[r]&L\arrow[r]&F\arrow[r]&0
			\end{tikzcd}\]
		and the point is that many things "have sections". What does it mean to have sections.
	\end{enumerate}
\end{proof}

\subsection{Riemann-Roch-Hirzebruch for line bundles on complex surfaces}

\begin{defn}
	A \textit{\textbf{complex surface}} is a compact complex manifold of dimension 2.
\end{defn}

\paragraph{Notation}
\[(L_1,L_2)=c_1(L_1)\wedge c_1(L_2)\]
and if $D$ is a divisor we write
\[(D,L)=\operatorname{deg}_DL=\int_{M}[D]\wedge c_1(L)\]

\begin{thm}[RRH for surfaces]\leavevmode
	$L$ line bundle on surface and $K_{X}=\Omega^{2}(X)$ its canonical bundle. Then
	\begin{equation*}\label{eq:rrh}
\chi(L)=\chi(\mathcal{O}_{X})+\frac{(L-K_X,L)}{2}
	\end{equation*}
	where $(A,B)$ denotes the intersection form applied to cohomology classes on $X$.
\end{thm}

\begin{proof}\leavevmode 
	
	\begin{enumerate}[label=\textbf{Step \arabic*}]
		\item Let $D$ a smooth curve of genus $g$ and $L_1,L_2$ line bundles that fit in an exact sequence
			\[\begin{tikzcd}
				0\arrow[r]&L_2\arrow[r]&L_2\arrow[r]&L_2|_{D}\arrow[r]&0
			\end{tikzcd}\]
Then we use Rieman-Roch for curves gives
\[\chi(L_1)=\chi(L_2)+(L_2,D)+(1-g)\]

\item Let $N D$ denote the normal bundle on $D$. The adjunction formula gives $K_D=K_X|_{D}\otimes KD$. Since $g-1=\operatorname{deg}K_D/2$, we obtain $1-g=-(K_X+D,D)/2$.

\item The next step goes as before, with  Rieman-Roch in one dimension. Let $\chi'(L)$ be the RHS of \cref{eq:rrh}, namely  $\chi'(L)=\chi(\mathcal{O}_{X})+\frac{L-K_X,L)}{2}$. In step 1 we have $c_1(L_2)=c_1(L_1)+D$. Then
	\begin{align*}
		\chi'(L_2)-\chi'(L_1)&=\frac{1}{2}\left[ (L_2-K_X,L_2)-(L_2-K_X-D,L_2-D) \right] \\
		&=(L_2,D)-(K_X+D,D)/2
	\end{align*}

	\item Comparing Step 2 and Step 3, we get
		\[\chi'(L_2)-\chi'(L_1)=\chi(L_2)-\chi(L_1)\]
		Therefore, \cref{eq:rrh} is equivalent for $L_2$ and for  $L_1$. We just need to manipulate bundles to reduce a bundle to… by building exact sequences.

	\item So suppose you have a smooth section of a bundle. Take an ample bundle $A$ and do
		\[\begin{tikzcd}
			0\arrow[r]&\mathcal{O}_X\arrow[r]&A^N\arrow[r]&A^N|_{D}\arrow[r]&0
		\end{tikzcd}\]
		\[\begin{tikzcd}
			0\arrow[r]&L\arrow[r]&A^N\otimes L\arrow[r]&A^N\otimes L|_{D}\arrow[r]&0
		\end{tikzcd}\]
	and then by step 4 we just need to deal with $A^N\otimes L$.

	\item It's very ample, it has many sections, including some that are smooth. Now we just assume $L$ is $A^N\otimes L$. So
		\[\begin{tikzcd}
			0\arrow[r]&\mathcal{O}_X\arrow[r]&L\arrow[r]&L|_{D}\arrow[r]&0
		\end{tikzcd}\]
		so for bundles that have smooth sections the statement is free.
	\end{enumerate}
\end{proof}

\subsection{Applying the general formula to the curve case}

We have
\[\operatorname{Ch}_{\cdot}(L)=1+c_1(L)+\frac{c_1^2(L)}{2}\]
\[\operatorname{Td}_{\cdot}(L)=1+\frac{c_1(TM)}{2}+\frac{c_1^2(M)+c_2}{12}\]
Now
\[\chi(L)-\chi(\mathcal{O})=-\frac{(K_1(L),K)}{2}+\frac{c_1(L)^2}{2}=\frac{(L,K-L)}{2}\]





\section{Class 6: Local Torelli theorem and its applications}

\subsection{Exponential exact sequence}

The exponential exact sequence is
\[\begin{tikzcd}
	0\arrow[r]&\mathbb{Z}_M\arrow[r]&\mathcal{O}_M\arrow[r]&\mathcal{O}_M^*\arrow[r]&0
\end{tikzcd}\]
and it gives a long exact sequence
\[\begin{tikzcd}[column sep=small]
	\cdots\arrow[r]&H^{1}(\mathcal{O}_M)\arrow[r]&H^{1}(\mathcal{O}^*_M)=\operatorname{Pic}\arrow[r,"c_1"]&H^{2}(M,\mathbb{Z})\arrow[r,"\alpha"]&H^{2}(\mathcal{O}_M)\arrow[r]&\arrow[r]&\cdots
\end{tikzcd}\]
$\alpha$ is just forgetful map, a projection, to the  $H^{0,2}(M)$ part of a form

The group $H^{2}(\mathcal{O}_M)$ is identified with $H^{0,2}(M)$ which is Dolbeault cohomology, hence the kernel of $\alpha$ is $H^{2}(M,\mathbb{Z})\cap H^{1,1}(M)$.

\begin{prop}
	$c_1$ holomorphic line bundle on compact K\"ahler manifold belongs to intersection $H^{2}(M,\mathbb{Z})\cap H^{1,1}(M)$ and every element of this group can be realised as $c_1(L)$.
\end{prop}

\subsection{K3 surfaces are holomorphically symplectic}

\begin{defn}
	A \textit{\textbf{complex surface}} is a compact, complex manifold of complex dimension 2.
\end{defn}

\begin{defn}
	A \textit{\textbf{K3 surface}} is a (K\"ahler, can drop this assumption) complex surface $M$ with $b_1=0$ and $c_1(M,\mathbb{Z})=0$
\end{defn}

\begin{remark}
	The hypothesis that $c_1=0$ implies that $c_1(K_M)=0$ and thus $K_M=0_M$ (it is trivial). This is beacuase  $H^{1}(\mathcal{O}_M)=0$, which follows from Hodge theory.
\end{remark}

\subsection{Hodge diamond of a K3 surface}

\[\begin{array}{c c c c c }
	&&1\\
	&0&&0\\
	1&&20&&1\\
	&0&&0\\
	&&1
\end{array}\]
since the cohomology groups
\[\begin{array}{c c c c c }
	&&\mathbb{C}\\
	&0&&0\\
	\text{sections of }K_X= H^{2,0}=\mathbb{C}&&?&&\text{Hodge (Serre?) duality}\implies  H^{0,2}=\mathbb{C}\\
	&0&&0\\
	&&H^{1,1}=\mathbb{C}
\end{array}\]
For the missing one, we comput $\chi(\mathcal{O}_M)$ using Riemann-Roch, which gives $c_2$ and from that we comput $b_2$.

\subsection{Geometric structures (the story of Teichm\"uler space)}

\begin{defn}
	\textit{\textbf{Geometric structure}} on a manifold is reduction of structure group to  $G\subset \operatorname{GL}()$.
\end{defn}

\subsection{Fr\'echet spaces}

\begin{defn}
	A \textit{\textbf{seminorm}} on a vector space $V$ is a function $\nu:V\to \mathbb{R}^{\geq 0}$ such that
	\begin{itemize}
	\item $v(\lambda x)=|\lambda| \nu(x)$ 
	\item triangle inequality.
	\end{itemize}
\end{defn}

\begin{defn}
	Define a topology using a family of seminorms generated by the open balls of all seminorms.
\end{defn}

\begin{defn}
	$V$ infinite dimensional vector space, $\nu_{\alpha}$ collection of seminorms. Sequence of vectors $z_i$ is \textit{\textbf{Cauchy}} if  $z_i$ is Cauchy for each $\nu_j$. If all Cauchy sequences converge it is called \textit{\textbf{Fr\'echet space}}.
\end{defn}

We can also define Fr\'echet space to with the distance $d(x,y)=\sum_{k=1}\frac{1}{2^k}\max(\nu_k(x-y),1)$ 

\begin{defn}
	The \textit{\textbf{topology $C^k$}} on a Riemannian manifold on the space $C^\infty_{c}(M)$ is
	\[|\varphi|_{C^k}:=\operatorname{sup}\sum_{i=0}^k|\nabla^i\varphi|\]
	where $\nabla^i$ is the iterated connection $\nabla^i:\mathcal{C}^\infty(M)\to \Lambda^{1}(M)^{\otimes i}$
\end{defn}

\begin{defn}
	Of tensor field, section of $TM^{\otimes i}\otimes T^*M^{\otimes j}$.
\end{defn}

\subsection{$C^0$ topology on group of diffeomorphisms}

\paragraph{Idea} To interpret diffeomorphisms as sections of a bundle. 

\begin{defn}
	On  $\operatorname{Dif}(M)$, riemannian manifold,
	\[d(f_1,f_2)=\operatorname{sup}_{x\in M}d(f_1(x),f_2(x))\]
\end{defn}

\subsection{$\mathcal{C}^\infty$-topology on group of diffeomorphisms}

It has more sets (is stronger) than the $C^0$ topology, 


\begin{defn}
	Fix $\mathcal{U}$ small neightbourdhoos of $\operatorname{id}$ in $\operatorname{Dif}(M)$. Choose an atlas of $U_i\subset V_i$ such that $U_i$ is relatively compact. There exists a neighbourhood of identity in $\operatorname{Dif}$ such that diffeomorphisms (sufficiently close to identity) they map $\tau (U_i)\subset V_i$. To find this neighbourhood use that closure of $U_i$ is compact in $V_i$.

	Now define the $C^\infty$ topology on $\mathcal{U}$ as $C^\infty$ convergence on maps from $U_i\subset \mathbb{R}^n$ to $V_i\subset \mathbb{R}^{n}$ using usual derivatives.
\end{defn}

Anyways, the idea is that we only need a \textit{uniform structure}  which is a partially ordered set to define Cauchy sequences.

\subsection{Teichm\"uler space of geometric structures}

Let $\mathcal{C}$ be the set of all geometric structures of a given type equipped with $C^\infty$ topology. The \textit{\textbf{Teichm\"uller space}} is  $\mathcal{C}/\operatorname{Diff}_0$, where $\operatorname{Diff}_{0}$ is the connected component of the identity. The group $\operatorname{Diff}(M)/ \operatorname{Diff}_0(M)$ is the \textit{\textbf{mapping class group}}, we are not going to use it.

\subsection{Teichm\"uler space of symplectic structures}

$\operatorname{Symp}\subset \Gamma(\Lambda^{2}(M)$. It is not Housdorff and we don't even know how much Housdorff it is. Maybe for four dimensional manifolds,…

\subsection{Moser's theorem}

\begin{thm}[Moser, 1965]\leavevmode
	The Teichm\"uler space is a manifold, and the preiod map
		\begin{align*}
		\operatorname{Per}:\operatorname{Teich}_{s}&\longrightarrow H^{2}(M,\mathbb{R})\\
		w &\longmapsto [w]
		\end{align*}
\end{thm}

It is very beautiful but semi-elementary if you know Moser's lema.

\subsection{The kernel of a differential form}

If $\Omega$ is a differential form on $M$, its \textit{\textbf{kernel}} is the space of all vectors  $X\in TM$ such that $i_X(\Omega)=0$.

 \begin{prop}
	 $[\ker \Omega,\ker \Omega]\subset \ker \Omega$.
\end{prop}

\begin{coro}
	If  $(M,I)$ almost complex and  $\Omega\in\Lambda^{2,0}(M)$ non-degenerate and closed, then $I$ is integrable.
\end{coro}

\begin{proof}
	$T^{0,1}=\ker \Omega$.
\end{proof}

\subsection{C-symplectic structures}

\begin{defn}
	$\Omega\in\Lambda^{2}(M,\mathbb{C})$, $M$ $4n$-dimensional manifold. Suppose that  $\Omega^{n+1}=0$ and $\Omega^n\wedge \overline{\Omega}^n$ is nowhere zero. Then $\ker \Omega\oplus \overline{\ker \Omega}=TM\otimes \mathbb{C}$.

	This is a \textit{\textbf{C-symplectic}} manifold.
\end{defn}

These manifolds have a nice Teichm\"uler space.

\begin{thm}[Moser-Koebe?]\leavevmode
	$(M,I_+,\Omega_+($ family of  $C$-symplectic forms, $[\Omega_t]=$constant, $H^{0,1}(M_t)=0$ then all $\Omega_t$ are related by a diffeomorphism. (This is Moser's trick!)
\end{thm}

Notice that for $n=1$ we have that the condition $\Omega^2=0$ and $\Omega\wedge \overline{\Omega}$ volume mean


\begin{thm}\leavevmode
	$\operatorname{CTeich}$ Teichm\"uler space of C-symplectic structures on K3 surface. Consider the
	\[\operatorname{Per}:\Omega\to [\Omega]\in H^{2}(M,\mathbb{C})\]
	Then the image $\operatorname{Per}(\operatorname{Teich})=\{pQ]:\int u\wedge  u=0, \int u\wedge  \bar{u}>0\}$ is a quadric.

	This is a local diffeomorphism.
	
\end{thm}

\subsection{The period space of complex structures}

Now take $\operatorname{CTeich}/\mathbb{C}^*$ because the Teichm\"uler space of complex structures has a free $\mathbb{C}^*$ action.


\begin{prop}[Local Torelli theorem for complex structures]
	Teichm\"uler space of complex strcutres on K3.
	\[\mathbb{P}\operatorname{er}=\{v\in\mathbb{P}H^{2}(M,\mathbb{C}):(v,v)=0,(v,\bar{v})>0\}\]
	so
	\[\frac{\operatorname{CTeich}}{\mathbb{C}^*}\longrightarrow \frac{Q}{\mathbb{C}^*}\subset \mathbb{P}H^{2}(M,\mathbb{C})\]
\end{prop}

\subsection{The period space of complex structures is a Grassmanian}

Lets define
\[\operatorname{Gr}_{++}(H^{2}(M,\mathbb{R}))=\text{positively oriented 2-planes in }H^{2}(M,\mathbb{R}) \]
Where positively oriented means the form is ? Then
\begin{align*}
\mathbb{P}\operatorname{er}&=\operatorname{Gr}_{++}(H^{2})=\dfrac{\operatorname{SO}(3,1\mathcal{o})}{\operatorname{SO}(2) \times \operatorname{SO}(1,1\mathcal{o})}
\end{align*}

\section{Class 7}

\subsection{Reminder on local Torelli theorem}

\[\operatorname{CTeich}=\text{Teichmüller space of hol sympl structures} \]
\begin{align*}
	\operatorname{CTeich} &\longrightarrow H^{2}(M,\mathbb{C}) \\
	\Omega &\longmapsto [\Omega ]
\end{align*}

Then this map is a local differomorphism to the period space.

\subsection{Hodge index theorem (without slides)}

\begin{thm}[Hodge index theorem]\leavevmode
	Consider a signature of intersection form on complex K\"ahlerler surface is positive on real part of $\operatorname{Re}H^{2,0}(M)$, $(1,0)$ on  $H^{1,1}(M,\mathbb{R})$, negative on 
	\begin{align*}
		\ker L:H^{1,1}(M,\mathbb{R})  &\longrightarrow H^{4}(M) =\mathbb{R}\\
		X &\longmapsto [X\wedge \omega ]
	\end{align*}
	with $\omega$ Kähler form.
	
\end{thm}

\begin{proof}
	$\Omega^{2,0}$ 1 dimensional, $\operatorname{Re}\Omega^{2,0}$ 2-dimensional (at most) in $\Lambda^2(M,\mathbb{R})$.
	\begin{align*}
		\Omega&=\omega_1+r-1\omega_2\\
		\Omega \wedge  \overline{\Omega}&=\omega_1^2+\omega_2^2>0\\
		\Omega \wedge \Omega &=0=\omega_1^2=\omega^2_2\\
		\iff \omega_1\wedge \omega_2&=0\\
		\omega_1^2&=\omega_2^2\\
		\implies \omega_1\perp \omega_2\\
		\omega_1^2&=\omega_2^2>0
	\end{align*}

	Then 
	\[\Lambda^{1,1}&=\ker L\oplus \omega\]
	That is a 4-dimensional bundle that is given by multiples of the K\"ahler form plus the primitive part. Then
	\[\ker^\perp=\left<\operatorname{Re} \Omega,\operatorname{Im}\Omega,\omega\right> \]
	Then consider Hodge star operator $*:\Lambda^2\to \Lambda^2$, $*^2=1$ and its complementary, this interchanges eigenvalues, negative positive, …
\end{proof}

\begin{coro}
	Signature of K3 surface is $(3,19)$
\end{coro}

\subsection{The period space of complex structures is Grassmanian}

\begin{claim}
\[\mathbb{P}\operatorname{er}=\dfrac{\operatorname{SO}(3,b_2-3)}{\operatorname{SO}(1,b_2-3) \times \operatorname{SO}(2)}=\operatorname{Gr}_{++}(h^{2}(M,\mathbb{R})\]
\end{claim}

\begin{remark}
	$(V,q)$ real vector space signature $q$ is $(m,n)$, $m\geq 2$ then
\[\operatorname{Gr}_{++}(V,q)=\{\ell\in\mathbb{P}V_{\mathbb{C}}:q(\ell,\ell)=0,q(\ell,\bar{\ell})>0\} \]
Recall that
$T_p\operatorname{Gr}_{++}=\operatorname{Hom}(?)$
\end{remark}

\begin{proof}[Of the claim]\leavevmode 
	\begin{enumerate}[label=\textbf{Step \arabic*}]
		\item $\ell\in\mathbb{P}\operatorname{er}$
		\begin{align*}
			q(\operatorname{Re}(\Omega),\operatorname{Im}(\Omega))&=0\\
			q(\operatorname{Re}(\Omega),\operatorname{Re}(\Omega))=q(\operatorname{Im}(\Omega),\operatorname{Im}(\Omega))>0
		\end{align*}
		What is going on
		\begin{align*}
			\omega_1&=\operatorname{Re}(\Omega),\qquad \omega_2=\operatorname{Im}(\Omega),\qquad \Omega \in\ell\\
			q(\omega_1+\sqrt{-1} \omega_2,\omega_1+\sqrt{-1}\omega_2&=0\\
			&=q(\omega_1,\omega_1)-q(\omega_2,\omega_2)+\sqrt{-1} 2q(\omega_1,\omega_2)
		\end{align*}
		and also 
		\begin{align*}
			q(\Omega,\overline{\Omega})&>0\\
			=q(\omega_1-\sqrt{-1}\omega_2,\omega_1+\sqrt{-1}\omega_2)&=q(\omega_1,\omega_1)+q(\omega_2,\omega_2)>0
		\end{align*}
		so we have obtained from a line in Period a positive definite plane

		\item $p\in\operatorname{Gr}_{++}$. Project, obtain a quadric form on $\mathbb{C}^2$.
			There exist two lines in $P_{\mathbb{C}}, $ $ \ell,\overline{ \ell}$ such tat
			\[q(\ell,\ell)=0, \qquad q(\bar{\ell},\bar{\ell})=0  \]
			\[q(x,y)=xy\]
	\end{enumerate}
\end{proof}

\begin{coro}
	$U\subset \operatorname{Teich}$, $V\subset H^{2}(M,\mathbb{R})$ set of all nonzero $(1,1)$-classes on $H^{2}(M,I)$ for some $I\in U$ Then $V\subset H^{2}(M,\mathbb{R})$ is open.
\end{coro}

\begin{proof}
Idea: take a 2 dimensional space and move it a bit everywhere, consider orthogonal complement. Deform $P$ by taking a $y$ and considering its orthogonal complement. $y$ is chosen close to $x$.

\[\begin{tikzcd}
	X\in P^\perp \text{is a class of type (1,1)}\\
	y \text{near }x\\
	P \text{project to }y^\perp \\
	\text{so you have an open set in Grassmanian} 
\end{tikzcd}\]
\[U_x\overset{\phi}{\longrightarrow}\operatorname{Gr}_{++}\qquad \qquad \phi^{-1}(U)\]


	\begin{enumerate}[label=\textbf{Step \arabic*}]
		\item 	Take a complex structure $I\in\operatorname{Teich}$, $P\subset H^{2}(M,\mathbb{R})$, then $H^{11}(M,I)=P^\perp$.

			\item Teichmüler is locally diffeomorphic to $\operatorname{Gr}_{++}$. Suffices to show in a nieghbourhood $U_1 \ni P$ in $\operatorname{Gr}_{++}$ that $\bigcup_{P_1\in U_1}P_1^+ $ is open.

		\item $y\in H^{2}(M,\mathbb{R})$, $y\in U_x$, nonzero in a neighb of $x\in P^\perp$. $P_y$ projection from $P$ to $y^\perp$
	\end{enumerate}
\end{proof}

\subsection{Intersection form on a K3 surface}

\begin{lemma}[Of linear algebra]
	Consider bilinear symmetric form on $V_\mathbb{Z}$
	\begin{align*}
		\pi:V_R\setminus 0  &\longrightarrow \mathbb{P}V_Q \\
		 &\longmapsto 
	\end{align*}
	where $R$ is the set of odd vectors and $Q$ rational vectors. Then  $p(\text{odd vectors} )$ is dense on $\mathbb{P}V_Q$.
	
\end{lemma}

\begin{proof}\leavevmode 
	\begin{enumerate}[label=\textbf{Step \arabic*}]
		\item Construct a sequence of odd vectors converging to any element $s\in V_{\mathbb{Z}}\setminus 0$.
			\[\lim_n\pi(r_0+2ns)=\pi(s)\]
	\end{enumerate}
\end{proof}

\begin{thm}
	Intersection form of K3 is even.
\end{thm}

\begin{proof}
	\begin{enumerate}[label=\textbf{Step \arabic*}]
		\item suppose it is odd. Coro 1 lema 1 imply complex structure $I$ and odd vector $r\in H^{1}(M,I)$. The point is that the set of vectors of type 11 is open negithboudood of any class. But then the odd classes are dense.

		\item Involves Riemann-Roch. For each class $r\in H^{11}(M)\cap H^{2}(M,\mathbb{Z})$ we have a holomorphic line bundle by the exponential sequence, $c_1|L|$, $L$ is hol line bundle:
			\[\begin{tikzcd}
				0\arrow[r]&\mathbb{Z}_M\arrow[r,"\sqrt{-1}2\pi]&\mathcal{O}_M\arrow[r,"\operatorname{exp}"]&\mathcal{O}^*_M\arrow[r]&0
			\end{tikzcd}\]
			and
			\[\begin{tikzcd}[column sep=small]
				\cdots\arrow[r]&0=H^{1}(\mathcal{O}_M)\arrow[r]&\operatorname{Pic}= H^{1}(\mathcal{O}^*_M)\arrow[r]&H^{2}(M,\mathbb{Z})\arrow[r,"\operatorname{proj}"]&H^{02}(M)=H^{2}(\mathcal{O}_M)\arrow[r]&\arrow[r]&\cdots
			\end{tikzcd}\]
			Now Riemann-Roch:
			\[\chi(L)=\chi(\mathcal{O}_M-\frac{L(K-L)}{2}=2-(L,L)/2\]
			because canonical bundle is trivial, so that cannot be odd.
	\end{enumerate}
\end{proof}

\subsection{Smooth quartics}

\begin{defn}
	A \textit{\textbf{smooth quartic}} is a smooth quartic hypersurface in  $\mathbb{P}^3$. So a solution of a quartic equation.
\end{defn}

\begin{remark}
	Adjunction formula. Canonical bundle of quartic is canonical bundl eof $\mathbb{C}P^{3}$ restrictesd to quartic times normal bundle:
	\[K_Q=K_{\mathbb{C}P^{3}}|_{Q}\otimes N(Q)\]
	But $N(Q)$ is degree four so it is just $\mathcal{O}(4)=N(Q)$.

	and canonical bundle $K_{\mathbb{C}P^{3}}$ of $\mathbb{C}P^{3}$ is $\mathcal{O}(-4)$ by Euler formula.

	So $K_Q=0_Q$---quartic has trivial canonical bundle.
\end{remark}

\[V:\mathbb{C}P^{3}\hookrightarrow \mathbb{C}P^{34}\]
What is this map. It is associated to $\mathcal{O}(4)$, with the line system
\[\mathbb{C}P^{34}=\mathbb{P}H^{0}(\mathcal{O}4)^*\]
and it is called Veronese map.

\begin{claim}
	Smooth quartic is a hyperplane section of $V(\mathbb{C}P^{3})$.
\end{claim}

So any hyperplane on $\mathbb{C}P^{34}$ is a hyuperplane on $\mathbb{C}P^{3}$.

…So the zeroes of this restriction are quadrics.

\subsection{Smooth quartics and Lefschetz hyperplance section theorem}

\begin{thm}[Lefshetz hyperplane]\leavevmode
	$ \pi(H\cap V)=\pi_1(V)$
\end{thm}





\section{Class 8}


\end{document}



